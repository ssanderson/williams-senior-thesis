\documentclass[c]{beamer}
\usetheme{default}
\title{Generalizing Nondeterminism for Algebraic Computation Machines}
\subtitle{Possible Subtitle}
\author{Scott Sanderson}
\institute{Department of Mathematics\\Williams College}
\date{\today}
\usepackage{tikz}
\usepackage{beamerthesiscommands}
%  \usepackage{enumitem}
\usetikzlibrary{calc,arrows,shapes,positioning}

\begin{document}

\theoremstyle{definition}
\newtheorem{proposition}{Proposition}
\newtheorem{proofidea}{Proof Idea}

% \begin{frame}
%   \titlepage
% \end{frame}

% \begin{frame}{Outline}

%   \begin{itemize}
%   \item Incredibly Brief Introduction to Complexity Theory and Machine Models \pause
%   \item Turing Machines \pause
%   \item BSS Machines \pause
%   \item \ndet Machines \pause
%   \item Results \pause
%   \end{itemize}
  
% \end{frame}

% \begin{frame}{A Simple Puzzle}
  
%   \begin{itemize}
%   \item Does the set of numbers: $\set{1,4,7,-5,-3,18,-6,12}$ contain
%     a subset adding up to 10? \pause
%   \item Yes: $18 - 3 - 6 + 1 = 10$
%   \end{itemize}
  
% \end{frame}


% \begin{frame}{\subsum{}}

%   \begin{definition}[\subsum{}]
    
%     An \textbf{instance} of \subsum{} is a pair $(S, t)$, where:

%     \begin{itemize}
%     \item $S = \set{s_1, s_2, \ldots, s_n}$ is a set of integers.
%     \item $t$ is a single integer, called the \textbf{target} for $S$.
%     \end{itemize}\pause

%     \vspace{\baselineskip}

%     An instance of \subsum{} is said to be a \textbf{Yes Instance} if
%     there exists some $X \subseteq S$ such that $\sum_X = T$.\\ \pause
    
%     Otherwise it is said to be a \textbf{No Instance}.

%   \end{definition}

% \end{frame}

% \begin{frame}{A Less Simple Puzzle}

%   \textbf{Task:} Given an instance $(S, T)$ of \subsum{}, determine
%   whether $(S, T)$ is a Yes Instance or a No Instance.
  
% \end{frame}

% \begin{frame}{Computational Complexity Theory}

%   \begin{itemize}
%   \item \subsum{} is an example of what is known as a \textbf{Decision
%       Problem}.\pause
%   \item Very generally, a Decision Problem $D$ is a set of strings,
%     partitioned into Yes Instances and No Instances, $D_{yes}$ and
%     $D_{no}$.\pause
%   \item The task associated with a Decision Problem is find a
%     procedure for telling whether a given member of $D$ falls into
%     $D_{yes}$ or $D_{no}$.\pause
%   \item Computational Complexity Theory seeks to study the intrinsic
%     \textbf{difficulty} of solving these tasks.
%   \end{itemize}

% \end{frame}

\begin{frame}{Machine Models}
  
  \begin{itemize}
  \item We formalize the notion of ``procedure for performing a task''
    by developing models of computing machines.
  \item In classical theory of computation, the de facto machine model
    is the Turing Machine.\pause
  \end{itemize}

\end{frame}

\begin{frame}{Turing Machine (Formal Definition)}

  \begin{definition}[Turing Machine (TM)]

    A \textbf{Turing Machine} is a 7-tuple, $(Q, \Sigma, \Gamma,
    \delta, q_0, q_{accept}, q_{reject})$, where:

    \begin{itemize}
    \item $Q$ is a set of machine states.
    \item $\Sigma$ is an alphabet of input characters, not containing
      the blank symbol, $\blank$.
    \item $\Gamma$ is the tape alphabet, with $\blank \in \Gamma$, and
      $\Sigma \subset \Gamma$.
    \item \functype{\delta}{(Q \times \Gamma)}{(Q \times \Gamma \times
        \set{L, R})} is the machine's
      \textbf{transition function}.
    \item $q_0$ is the initial state of the machine.
    \item $q_{accept}$ is the machine's accept state.
    \item $q_{reject}$ is the machine's reject state.
    \end{itemize}
  \end{definition}
  
\end{frame}

\begin{frame}[Turing Machine (Less Scary Definition)]
  
  \begin{itemize}
  \item Essentially, a Turing Machine is an infinite, one-way tape
    divided into cells, along with a tape head that rests on a
    single cell.  
  \item The transition function $\delta$ encodes instructions
    which tell the tape head, at each step of the computation,
    write a symbol from $\Gamma$ to its current position, then
    move either left or right.
  \item The specific instruction that gets executed depends on the
    state of the machine and the symbol currently under the tape
    head.
  \end{itemize}
\end{frame}

\begin{frame}[An Example Turing Machine Computation]
  \begin{columns}[c]
    
    \small

    \column{2in}

    \begin{itemize}
    \item $Q = \set{q_0, q_{accept}, q_{reject}}$
    \item $\Gamma = \set{\blank, 0, 1}$
    \item $\Sigma = \set{0,1}$
    \item $\delta(q_0, 0) = (q_0, 1, \rightarrow)$
    \item $\delta(q_0, 1) = (q_0, 0, \rightarrow)$
    \item $\delta(q_0, \blank) = (q_{accept}, \blank, \leftarrow)$
    \end{itemize}

    \column{2in}
    
    \begin{figure}[p]
      
      PLACEHOLDER
      
      %   \tape[1]{$q_0$}{{1,0,0,1,$\blank$,$\cdots$}} \vspace{1mm}
      %   \tape[2]{$q_0$}{{0,0,0,1,$\blank$,$\cdots$}} \vspace{1mm}
      %   \tape[3]{$q_0$}{{0,1,0,1,$\blank$,$\cdots$}} \vspace{1mm}
      %   \tape[4]{$q_0$}{{0,1,1,1,$\blank$,$\cdots$}} \vspace{1mm}
      %   \tape[5]{$q_0$}{{0,1,1,0,$\blank$,$\cdots$}} \vspace{1mm}
      %   \tape[4]{$q_{accept}$}{{0,1,1,0,$\blank$, $\cdots$}} \vspace{1mm}
      % \caption{An Example Turing Machine Computation}
    \end{figure}
    
  \end{columns}
\end{frame}

\end{document}