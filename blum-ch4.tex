\documentclass[twoside]{article}
\setlength{\oddsidemargin}{0.25 in}
\setlength{\evensidemargin}{-0.25 in}
\setlength{\topmargin}{-0.6 in}
\setlength{\textwidth}{6.5 in}
\setlength{\textheight}{8.5 in}
\setlength{\headsep}{0.75 in}
\setlength{\parindent}{0 in}
\setlength{\parskip}{0.1 in}

\setcounter{secnumdepth}{2}

%
% ADD PACKAGES here:
%
\usepackage{amsmath,amsfonts,graphicx}

\renewcommand{\thepage}{\arabic{page}}
\renewcommand{\thesection}{\arabic{section}}
\renewcommand{\theequation}{\arabic{equation}}
\renewcommand{\thefigure}{\arabic{figure}}
\renewcommand{\thetable}{\arabic{table}}

%
% The following macro is used to generate the header.
%
\newcommand{\head}[2]{
   \pagestyle{myheadings}
   \thispagestyle{plain}
   \newpage
   \noindent
   \begin{center}
   \framebox{
      \vbox{\vspace{2mm}
    \hbox to 6.28in { {\bf Scott Sanderson
		\hfill Source: \cite{#1} }}
       \vspace{4mm}
       \hbox to 6.28in { {\Large \hfill #2 \hfill} }
       \vspace{2mm}
       \hbox to 6.28in { { \hfill Date: \today} }
      \vspace{2mm}}
   }
   \end{center}
   
   \vspace*{4mm}
}

% Thesis Commands
\usepackage{thesiscommands}

% Convention for citations is authors' initials followed by the year.
% For example, to cite a paper by Leighton and Maggs you would type
% \cite{LM89}, and to cite a paper by Strassen you would type \cite{S69}.
% (To avoid bibliography problems, for now we redefine the \cite command.)
% Also commands that create a suitable format for the reference list.
\renewcommand{\cite}[1]{[#1]}
\def\beginrefs{\begin{list}%
        {[\arabic{equation}]}{\usecounter{equation}
         \setlength{\leftmargin}{2.0truecm}\setlength{\labelsep}{0.4truecm}%
         \setlength{\labelwidth}{1.6truecm}}}
\def\endrefs{\end{list}}
\def\bibentry#1{\item[\hbox{[#1]}]}


%Use this command for a figure; it puts a figure in wherever you want it.
%usage: \fig{NUMBER}{SPACE-IN-INCHES}{CAPTION}
\newcommand{\fig}[3]{
			\vspace{#2}
			\begin{center}
			Figure \thelecnum.#1:~#3
			\end{center}
	}
% Use these for theorems, lemmas, proofs, etc.
\newtheorem{theorem}{Theorem}[section]
\newtheorem{lemma}{Lemma}[section]
\newtheorem{proposition}{Proposition}[section]
\newtheorem{claim}{Claim}[section]
\newtheorem{corollary}{Corollary}[section]
\newtheorem{definition}{Definition}[section]
\newenvironment{proof}{{\bf Proof:}}{\hfill\rule{2mm}{2mm}}
\newenvironment{proofsketch}{{\bf Proof Sketch:}}{\hfill\rule{2mm}{2mm}}

\begin{document}
%FILL IN THE RIGHT INFO.
%\head{**Title**}{**Date**}
\head{Blum et al.}{Chap 4 - Decision Problems and Complexity over a Ring}

\section{Decision Problems} 

  The treatment of complexity questions in our model is much the same
  as in classical complexity theory.  We frame problems as languages,
  ie, as subsets of our input domain.  In the classical theory, this
  means binary encodings of strings -- for us, it means subsets of
  $R^\infty$.  The complexity/decidability of a problem then becomes a
  question about the complexity/decidability of testing an input for
  membership in a language.

  \subsection{Definitions}
  
  \begin{definition}{\textbf{Decision Problem}}
      
    A \textbf{Decision Problem} over a ring $R$ is a set, $S \subseteq
    R^\infty$.  A \textbf{Structured Decision Problem} is a set $S$,
    partitioned into subsets, $S_{yes}$ and $S_{no}$, respectively
    called the \textbf{Yes Instances} and \textbf{No Instances} of 
    $S$.
 
  \end{definition}

  \begin{definition}{\textbf{Characteristic Function}}

    Let $S = S_{yes} \cup S_{no} \subseteq R^{\infty}$.  The
    characteristic function,
    \functype{\charfunc_S}{S}{\set{0,1}}, is given by:

    $$\charfunc_S(x) = \twopartdef{1}{x \in S_{yes}}{0}{x \in S_{no}}$$
  \end{definition}

  \subsection{Notable Decision Problems}

  \begin{itemize}
    \subsubsection{Decision Problems over Arbitrary Rings}
    \bolditem{HN} - Are there solutions to a finite system of polynomial equations?
    \bolditem{QUAD} - Does a finite system of quadratic equations have a solution?
    \bolditem{4-FEAS} - Does a finite system of degree-4 polynomials have a solution?
    \bolditem{QA-FEAS} - Does a set of quasi-algebraic equations describe the empty set?  
    \bolditem{KP} - Knapsack Problem
    
    \subsubsection{Decision Problems over Ordered Rings}
    \bolditem{SA-FEAS} - Does a set of semi-algebraic equations describe the empty set?
    \bolditem{LPF} - Feasibility of Linear Programming Problems
    \bolditem{TSP} - Travelling Salesman Problem
    
    \subsubsection{Decision Problems over $\integers$}
    \bolditem{IPF} - Feasibility of Integer Programming Problems

    \subsubsection{Decision Problems over $\integers_2$}
    \bolditem{SAT} - Boolean formula satisfiability
  \end{itemize}

  \subsection{Known Decidability Results}

  All of the above are known to be decidable (when well-defined) over
  $\complexes$, $\reals$, and $\integers_2$.  Over $\integers$,
  \textbf{IPF}, \textbf{TSP}, and \textbf{KP} are decidable, but
  \textbf{4-FEAS}, \textbf{QUAD}, \textbf{HN}, \textbf{QA-FEAS}, and
  \textbf{SA-FEAS} are not.  Over $\rationals$, \textbf{LPF},
  \textbf{TSP}, and \textbf{KP} are decidable.  It is an open problem
  whether \textbf{QUAD} and \textbf{HN} are decidable over $\rationals$.

\section{The Class P}

The complexity class $P$ is intended to abstract a notion of problems
that are computationally tractable.  In general the class $P$ refers
to those decision problems which can be computable in some number of
steps that is polynomial in some relevant measure of the size of the
problem's input.  Depending on context and interpretation of inputs,
the relevant measure of size varies. In the classical theory of
computation, size refers to the length of the machine's binary input.
Since the binary representation length of an integer is logarithmic in
its magnitude, this definition is equivalent to the set of problems
that can be computed in time polylogarithmic in the magnitude of the
machine's input interpreted as an integer.  In other cases, we find it
more appropriate to interpret the input to an algorithm as a set of
distinct inputs, each of which is bounded in size, rather than a
single input.  (This is how we generally concepualize the inputs to
sorting algorithms, for example).  In such a case, the relevant
measure of cost is taken to be the number of inputs, rather than the
magnitude of any single input.  In more rare cases, we may care about
both magnitude and number of inputs, in which case our relevant measure
of size may involve some product of the two measures.\\

We extend these intuitions to arbitrary rings by defining two
different measures of the magnitude of an input to a machine: the
\emph{length} (corresponding to the number of distinct inputs to a
machine) and the \emph{height} corresponding to the magnitude of each
input.  The overall size of an input is then taken to be product of
the length of an input with the maximum height of any element of the
input.  This naturally abstracts the case where we interpret our input
as a single value with unbounded magnitude ($length = 1$, $height =
n$) as well as the case where we regard the input as an array of
bounded values ($length = n$, $height = 1$).

\subsection{Definitions}

\begin{definition}{\textbf{Height Function} over $R$}

  We define a \textbf{Height Function} on a ring $R$ to be a map:
  $$\mfunctype{ht_R}{R^\infty}{\reals}$$
\end{definition}

\note{As we shall see, the height function on a ring effectively
  defines what we consider to the the cost of performing a computation
  on an element of the ring.  Of particular note are the \emph{unit
    height} function, $ht_R(x) = 1$, and the \emph{bit height}
  function, $ht_R(x) = \log(|x| + 1)$.  We use the unit height
  function in cases where we are concerned with the algebraic
  complexity of an algorithm (ie, the number of algebraic operations
  required to compute the algorithm, independent of the values on
  which the operations are performed).  We use the bit height function
  when we want to capture the classical theory of computation over
  $\integers$.  Over $\rationals$, we give an alternative definition
  of bit height, $ht_\rationals(x) = (max(ht_\integers(p),
  ht_\integers(q))$, where $x = \frac{p}{q}$, $p$ and $q$ relatively
  prime.  This captures the bit cost of representing a rational $x$ as
  a pair of integers.  Unless noted otherwise, we generally use bit
  cost for computation over $\integers$ and $\rationals$, and unit
  cost otherwise.}

\begin{definition}{\textbf{Input Length}}
  
  The \textbf{length} of an input $x \in R^n$ to a machine over $R$ is given by:
  $$length(x) = n$$
\end{definition}

\begin{definition}{\textbf{Input Size}}

  The \textbf{size} of an input $x = (x_1, x_2, \ldots, x_n)$ to a
  machine over a ring, $R$ with height function $ht_R$ is given by:
  $$length(x) * max(ht_R(x_i))$$.
\end{definition}
\note{It should be clear from the above that, in the case that $ht_R$
  is the unit height function, the size of the input to a machine
  reduces to just the length of the input.}

Now that we have a definition of input size, we want also to consider
the cost of performing a computation on a given machine, which we
define as the halting time of

\begin{definition}{\textbf{Computation Cost}}
  
  Let $x \in R^n$ be an input to a machine $M$ over $R$ with height
  function $ht_R$ defined.  The \textbf{cost}, $cost_M$ for $M$ to
  compute $\computefn_M(x)$ is given by:
  $$cost_M(x) = T(x) * ht_{max}(x)$$ where $ht_{max}(x)$ is given by \todo{Decipher page 87 of Blum et al}.
  
\end{definition}

\begin{definition}{\textbf{Polynomial Time Machine}}
  A machine $M$ over $R$ is a \textbf{polynomial time} machine on $X
  \subset \inspace_M$ if there exist positive integers $c$ and $q$
  such that:
  $$cost_M(x) \leq c(size(x))^q \mid \forall x \in X$$
\end{definition}
\begin{definition}{The Complexity Class $P$ over $R$}
  
  A decision problem $S \subseteq R^\infty$ is said to be in the class
  $P$ if there exists a polynmial-time machine $M$ that computes the
  characteristic function of S, $\charfunc_S$.
\end{definition}

\todo{Poly-time reductions}

\section*{References}
\beginrefs

% Sample citation, to be cited at by calling \cite{CW87}

\bibentry{B98}{\sc Blum et al.},
``Complexity and Real Computation,''
{\it Springer-Verlag New York, Inc.}
1998, pp. 37-69
\endrefs

% **** THIS ENDS THE EXAMPLES. DON'T DELETE THE FOLLOWING LINE:

\end{document}





