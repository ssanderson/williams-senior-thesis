\documentclass[twoside]{article}
\setlength{\oddsidemargin}{0.25 in}
\setlength{\evensidemargin}{-0.25 in}
\setlength{\topmargin}{-0.6 in}
\setlength{\textwidth}{6.5 in}
\setlength{\textheight}{8.5 in}
\setlength{\headsep}{0.75 in}
\setlength{\parindent}{0 in}
\setlength{\parskip}{0.1 in}

\setcounter{secnumdepth}{2}

%
% ADD PACKAGES here:
%
\usepackage{amsmath,amsfonts,graphicx}

\renewcommand{\thepage}{\arabic{page}}
\renewcommand{\thesection}{\arabic{section}}
\renewcommand{\theequation}{\arabic{equation}}
\renewcommand{\thefigure}{\arabic{figure}}
\renewcommand{\thetable}{\arabic{table}}

%
% The following macro is used to generate the header.
%
\newcommand{\head}[2]{
   \pagestyle{myheadings}
   \thispagestyle{plain}
   \newpage
   \noindent
   \begin{center}
   \framebox{
      \vbox{\vspace{2mm}
    \hbox to 6.28in { {\bf Scott Sanderson
		\hfill Source: \cite{#1} }}
       \vspace{4mm}
       \hbox to 6.28in { {\Large \hfill #2 \hfill} }
       \vspace{2mm}
       \hbox to 6.28in { { \hfill Date: \today} }
      \vspace{2mm}}
   }
   \end{center}
   
   \vspace*{4mm}
}

% Thesis Commands
\usepackage{thesiscommands}

% Convention for citations is authors' initials followed by the year.
% For example, to cite a paper by Leighton and Maggs you would type
% \cite{LM89}, and to cite a paper by Strassen you would type \cite{S69}.
% (To avoid bibliography problems, for now we redefine the \cite command.)
% Also commands that create a suitable format for the reference list.
\renewcommand{\cite}[1]{[#1]}
\def\beginrefs{\begin{list}%
        {[\arabic{equation}]}{\usecounter{equation}
         \setlength{\leftmargin}{2.0truecm}\setlength{\labelsep}{0.4truecm}%
         \setlength{\labelwidth}{1.6truecm}}}
\def\endrefs{\end{list}}
\def\bibentry#1{\item[\hbox{[#1]}]}


%Use this command for a figure; it puts a figure in wherever you want it.
%usage: \fig{NUMBER}{SPACE-IN-INCHES}{CAPTION}
\newcommand{\fig}[3]{
			\vspace{#2}
			\begin{center}
			Figure \thelecnum.#1:~#3
			\end{center}
	}
% Use these for theorems, lemmas, proofs, etc.
\newtheorem{theorem}{Theorem}[section]
\newtheorem{lemma}{Lemma}[section]
\newtheorem{proposition}{Proposition}[section]
\newtheorem{claim}{Claim}[section]
\newtheorem{corollary}{Corollary}[section]
\newtheorem{definition}{Definition}[section]
\newenvironment{proof}{{\bf Proof:}}{\hfill\rule{2mm}{2mm}}
\newenvironment{proofsketch}{{\bf Proof Sketch:}}{\hfill\rule{2mm}{2mm}}

\begin{document}
%FILL IN THE RIGHT INFO.
%\head{**Title**}{**Date**}
\head{Blum et al.}{Chap 4 - Decision Problems and Complexity over a Ring}

\begin{section}{Decision Problems} 

  The treatment of complexity questions in our model is much the same
  as in classical complexity theory.  We frame problems as languages,
  ie, as subsets of our input domain.  In the classical theory, this
  means binary encodings of strings -- for us, it means subsets of
  $R^\infty$.  The complexity/decidability of a problem then becomes a
  question about the complexity/decidability of testing an input for
  membership in a language.

  \subsection{Definitions}
  
  \begin{definition}{\textbf{Decision Problem}}
      
    A \textbf{Decision Problem} over a ring $R$ is a set, $S \subseteq
    R^\infty$.  A \textbf{Structured Decision Problem} is a set $S$,
    partitioned into subsets, $S_{yes}$ and $S_{no}$, respectively
    called the \textbf{Yes Instances} and \textbf{No Instances} of 
    $S$.
 
  \end{definition}

  \begin{definition}{\textbf{Characteristic Function}}

    Let $S = S_{yes} \cup S_{no} \subseteq R^{\infty}$.  The
    characteristic function,
    \functype{\charfunc_S}{S}{\set{0,1}}, is given by:

    $$\charfunc_S(x) = \twopartdef{1}{x \in S_{yes}}{0}{x \in S_{no}}$$
  \end{definition}

  \subsection{Notable Decision Problems}

  \begin{itemize}
    \subsubsection{Decision Problems over Arbitrary Rings}
    \bolditem{HN} - Are there solutions to a finite system of polynomial equations?
    \bolditem{QUAD} - Does a finite system of quadratic equations have a solution?
    \bolditem{4-FEAS} - Does a finite system of degree-4 polynomials have a solution?
    \bolditem{QA-FEAS} - Does a set of quasi-algebraic equations describe the empty set?  
    \bolditem{KP} - Knapsack Problem
    
    \subsubsection{Decision Problems over Ordered Rings}
    \bolditem{SA-FEAS} - Does a set of semi-algebraic equations describe the empty set?
    \bolditem{LPF} - Feasibility of Linear Programming Problems
    \bolditem{TSP} - Travelling Salesman Problem
    
    \subsubsection{Decision Problems over $\integers$}
    \bolditem{IPF} - Feasibility of Integer Programming Problems

    \subsubsection{Decision Problems over $\integers_2$}
    \bolditem{SAT} - Boolean formula satisfiability
  \end{itemize}

  \subsection{Known Decidability Results}

  All of the above are known to be decidable (when well-defined) over
  $\complexes$, $\reals$, and $\integers_2$.  

  Over $\integers$, \textbf{IPF}, \textbf{TSP}, and \textbf{KP} are
  decidable, but \textbf{4-FEAS}, \textbf{QUAD}, \textbf{HN},
  \textbf{QA-FEAS}, and \textbf{SA-FEAS} are not.  Over $\rationals$

\end{section}



\section*{References}
\beginrefs

% Sample citation, to be cited at by calling \cite{CW87}

\bibentry{B98}{\sc Blum et al.},
``Complexity and Real Computation,''
{\it Springer-Verlag New York, Inc.}
1998, pp. 37-69
\endrefs

% **** THIS ENDS THE EXAMPLES. DON'T DELETE THE FOLLOWING LINE:

\end{document}





