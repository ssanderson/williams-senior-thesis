\documentclass[twoside]{article}
\setlength{\oddsidemargin}{0.25 in}
\setlength{\evensidemargin}{-0.25 in}
\setlength{\topmargin}{-0.6 in}
\setlength{\textwidth}{6.5 in}
\setlength{\textheight}{8.5 in}
\setlength{\headsep}{0.75 in}
\setlength{\parindent}{0 in}
\setlength{\parskip}{0.1 in}

\setcounter{secnumdepth}{2}

%
% ADD PACKAGES here:
%
\usepackage{amsmath,amsfonts,graphicx}

\renewcommand{\thepage}{\arabic{page}}
\renewcommand{\thesection}{\arabic{section}}
\renewcommand{\theequation}{\arabic{equation}}
\renewcommand{\thefigure}{\arabic{figure}}
\renewcommand{\thetable}{\arabic{table}}

%
% The following macro is used to generate the header.
%
\newcommand{\head}[2]{
   \pagestyle{myheadings}
   \thispagestyle{plain}
   \newpage
   \noindent
   \begin{center}
   \framebox{
      \vbox{\vspace{2mm}
    \hbox to 6.28in { {\bf Scott Sanderson
		\hfill Source: \cite{#1} }}
       \vspace{4mm}
       \hbox to 6.28in { {\Large \hfill #2 \hfill} }
       \vspace{2mm}
       \hbox to 6.28in { { \hfill Date: \today} }
      \vspace{2mm}}
   }
   \end{center}
   
   \vspace*{4mm}
}

% Thesis Commands
\usepackage{thesiscommands}

%Use this command for a figure; it puts a figure in wherever you want it.
%usage: \fig{NUMBER}{SPACE-IN-INCHES}{CAPTION}
\newcommand{\fig}[3]{
			\vspace{#2}
			\begin{center}
			Figure \thelecnum.#1:~#3
			\end{center}
	}
% Use these for theorems, lemmas, proofs, etc.
\newtheorem{theorem}{Theorem}[section]
\newtheorem{lemma}{Lemma}[section]
\newtheorem{proposition}{Proposition}[section]
\newtheorem{claim}{Claim}[section]
\newtheorem{corollary}{Corollary}[section]
\newtheorem{definition}{Definition}[section]
\newenvironment{proof}{{\bf Proof:}}{\hfill\rule{2mm}{2mm}}
\newenvironment{proofsketch}{{\bf Proof Sketch:}}{\hfill\rule{2mm}{2mm}}

\begin{document}
%FILL IN THE RIGHT INFO.
%\head{**Title**}{**Date**}

\head{S12}{NDET-Machines over R}

\section{Definition}

Here we present a model of nondeterministic computation that naturally
generalizes from the Turing Machine model to the Blum model of
computation.  The basic idea is that each branch node is allowed to
have multiple \textbf{Yes} and \textbf{No} nodes, and an accepting
computation is any computational path through the machine that
respects the branching functions.

\begin{definition} NDET-Machine over R

  An NDET-Machine over a ring R is a finite, directed, connected graph
  with two types of nodes: \emph{computation} and \emph{branch}, along
  with distinguished \emph{input} and \emph{accept} nodes.

  Associated with each computation node, $\eta$, is a
  \emph{next-node}, $\beta_{\eta}$, and a computation map,
  \functype{g_\eta}{R^\infty}{R^\infty}

  Associated with each branch node, $\eta$, is a nonzero polynomial
  function \functype{g_\eta}{R^\infty}{R}, and a set $\set{\beta^+_1,
    \beta^+_2, \ldots \beta^+_i}$ of \emph{Yes Nodes} and a set
  $\set{\beta^-_1, \beta^-_2, \ldots \beta^-_j}$ of \emph{No Nodes}.
  We refer to the max of $i$ and $j$ as the \textbf{degree} of $\eta$.
\end{definition}

The computation of an NDET-machine is defined analogously to that of a
Blum machine, with the caveat that we now accept any input for which
there exists some accepting path through the machine.  More formally,
for each configuration pair $z = (\eta, x)$, we define the
\textbf{successor set} $H(z)$, to be the set of all pairs,
$\set{(\eta', x')}$, such that:

\begin{enumerate}
\item If $\eta$ is a computation node, then $\eta' = \beta_\eta$, and
  $x' = g(x)$.
\item If $\eta$ is a branch node, $x' = x$ and if $g(x) > 0)$, $\eta'$
  is a yes-node of $\eta$.  If $g(x) < 0$, $\eta'$ is a no-node of
  $\eta$.
\end{enumerate}

We say that an NDET-machine $M$ accepts an input $w$ if there exists a
sequence of configurations, $z_1 = (\eta_1, w), z_2 = (\eta_2, x_2),
\ldots z_k = (\eta_k, x_k)$, such that $z_1$ is the input node of $M$,
$\eta_k$ is the accept node of $M$, and $z_{i+1} \in H(z_i) \forall 1
\leq i \leq k$.  We say that $M$ accepts $w$ in time $T$ if there
exists such a sequence with $k \leq T$.

\begin{definition} $\ndetp$
  A language, $S$ is in $\ndetp$ if there exists a machine $M$ and a
  constant $p$ such that, $\forall w \in S$, $M$ accepts w in time
  $O(size(w)^p)$.
\end{definition}

It should be immediately clear that NDETP$_{\integers_2}$ is the
classical NP when taken with unit cost, much in the same way that Blum
et al's verifiability definition of NP collapses to the classical case
over $\integers_2$.  As we shall see, however, it is less clear
whether the equivalence of $\ndetp$ and NP holds for arbitrary rings.

\end{document}