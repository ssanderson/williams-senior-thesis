\documentclass[twoside]{article}
\setlength{\oddsidemargin}{0.25 in}
\setlength{\evensidemargin}{-0.25 in}
\setlength{\topmargin}{-0.6 in}
\setlength{\textwidth}{6.5 in}
\setlength{\textheight}{8.5 in}
\setlength{\headsep}{0.75 in}
\setlength{\parindent}{0 in}
\setlength{\parskip}{0.1 in}

\setcounter{secnumdepth}{2}

%
% ADD PACKAGES here:
%
\usepackage{amsmath,amsfonts,graphicx}

\renewcommand{\thepage}{\arabic{page}}
\renewcommand{\thesection}{\arabic{section}}
\renewcommand{\theequation}{\arabic{equation}}
\renewcommand{\thefigure}{\arabic{figure}}
\renewcommand{\thetable}{\arabic{table}}

%
% The following macro is used to generate the header.
%
\newcommand{\head}[2]{
   \pagestyle{myheadings}
   \thispagestyle{plain}
   \newpage
   \noindent
   \begin{center}
   \framebox{
      \vbox{\vspace{2mm}
    \hbox to 6.28in { {\bf Scott Sanderson
		\hfill Source: \cite{#1} }}
       \vspace{4mm}
       \hbox to 6.28in { {\Large \hfill #2 \hfill} }
       \vspace{2mm}
       \hbox to 6.28in { { \hfill Date: \today} }
      \vspace{2mm}}
   }
   \end{center}
   
   \vspace*{4mm}
}

% Thesis Commands
\usepackage{thesiscommands}

%
% Convention for citations is authors' initials followed by the year.
% For example, to cite a paper by Leighton and Maggs you would type
% \cite{LM89}, and to cite a paper by Strassen you would type \cite{S69}.
% (To avoid bibliography problems, for now we redefine the \cite command.)
% Also commands that create a suitable format for the reference list.
\renewcommand{\cite}[1]{[#1]}
\def\beginrefs{\begin{list}%
        {[\arabic{equation}]}{\usecounter{equation}
         \setlength{\leftmargin}{2.0truecm}\setlength{\labelsep}{0.4truecm}%
         \setlength{\labelwidth}{1.6truecm}}}
\def\endrefs{\end{list}}
\def\bibentry#1{\item[\hbox{[#1]}]}


%Use this command for a figure; it puts a figure in wherever you want it.
%usage: \fig{NUMBER}{SPACE-IN-INCHES}{CAPTION}
\newcommand{\fig}[3]{
			\vspace{#2}
			\begin{center}
			Figure \thelecnum.#1:~#3
			\end{center}
	}
% Use these for theorems, lemmas, proofs, etc.
\newtheorem{theorem}{Theorem}[section]
\newtheorem{lemma}{Lemma}[section]
\newtheorem{proposition}{Proposition}[section]
\newtheorem{claim}{Claim}[section]
\newtheorem{corollary}{Corollary}[section]
\newtheorem{definition}{Definition}[section]
\newenvironment{proof}{{\bf Proof:}}{\hfill\rule{2mm}{2mm}}
\newenvironment{proofsketch}{{\bf Proof Sketch:}}{\hfill\rule{2mm}{2mm}}

\begin{document}
%FILL IN THE RIGHT INFO.
%\head{**Title**}{**Date**}
\head{Blum et al.}{Chap 5 - Class NP and NP-Complete Problems}

\section{Definitions}

The definition of NP over R given by Blum et al is a generalization of
the verifiability definition of NP given in classical complexity theory.

\begin{definition}{\textbf{Class NP} over $R$ \textbf{(Witness Definition)}}

  A decision problem $S \subseteq R^\infty$ is in the class NP over a
  ring $R$ if there exist positive integers $c$ and $q$ and a a
  machine $M$ over R with $\inspace_M = R^\infty \times R^\infty$ such
  that:

  \begin{enumerate}
  \item if $x \in S$ then there exists a \emph{witness}, $w
      \in R^\infty$, such that $\computefn_M(x,w) = 1$ and
      $cost_M(x,w) \leq c(size(x))^q$.
  \item if $x \notin S$, then there is no such $w$.
  \end{enumerate}
\end{definition}

The classical theory admits a number of different characterizations of
NP.  Though the verifiability definition is often the simplest to
conceptualize and work with, another important definition is given in
terms of the power of \emph{non-determinism}, which in this context
means allowing a machine to be in multiple states simulataneously.  In
the classical theory, it turns out that the problems solvable in
polynomial time by a non-deterministic Turing Machine are precisely
the problems verifiable in polynomial time by a standard TM.  \\

Recall our earlier definition of a machine, $M$ as a tuple:

\todo{incorporate shift nodes from ch. 3}

$$M = (\inspace, \statespace, \outspace, \nodes, H, H_{in}, H_{out})$$
where 
$$\nodes = \set{\nodein, \nodeout} \cup \set{\eta_1, \eta_2, \ldots \eta_i} \cup
\set{b_1, b_2, \ldots b_j}$$

We defined the computing endomorphism $H$ as a function from
$\fullstate \rightarrow \fullstate$, representing our intuition that
at each computation step the machine transitions to a new node and
(possibly) mutates its currently stored state.  What we want to do now
is modify this model in order to abstract a notion of being able to
``simultaneously'' take multiple computational paths.  We do so as
follows:

\begin{definition}{\textbf{Non-deterministic machine} over a ring $R$}

\end{definition}


\section*{References}

\renewcommand{\cite}[1]{[#1]}
\def\beginrefs{\begin{list}%
        {[\arabic{equation}]}{\usecounter{equation}
          \setlength{\leftmargin}{2.0truecm}\setlength{\labelsep}{0.4truecm}%
          \setlength{\labelwidth}{1.6truecm}}}
\def\endrefs{\end{list}}
\def\bibentry#1{\item[\hbox{[#1]}]}

\beginrefs

\bibentry{B98}{\sc Blum et al.},
``Complexity and Real Computation,''
{\it Springer-Verlag New York, Inc.}
1998, pp. 37-69
\endrefs

% **** THIS ENDS THE EXAMPLES. DON'T DELETE THE FOLLOWING LINE:

\end{document}





