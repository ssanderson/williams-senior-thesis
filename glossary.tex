\chapter{Glossary of Symbols and Definitions}

In the following, let $M$ be a BSS Machine (finite or
infinite-dimensional), let $R$ be a ring, and let $x$ be an element of
$R^\infty$. In general, for any machine property we may omit the
subscript $_M$ if the intention is clear from context.

\section{Defining Properties of BSS Machines}
\begin{itemize}
\item $\inspace_M$ (\textbf{Input Space}): The space of values in
  $R^\infty$ that serve as valid inputs to $M$.
\item $\outspace_M$ (\textbf{Output Space}): The space of balues in
  $R^\infty$ that serve as valid outputs from $M$.
\item $\statespace_M$ (\textbf{State Space}): ``Location'' where $M$
  stores the results of partial computations. BSS analog of the
  classical Turing Machine tape.
\item $\nodes_M$ (\textbf{Nodes}): BSS analog to the states of a
  classical Turing Machine.
\item $\eta$ : Symbol used to denote a particular node of $M$.
  Distinguished nodes used in some machine definitions include
  $\nodein$ (\textbf{Input Node}), $\nodeout$ (\textbf{Output
    Node}), $\accept$ \textbf{Accept Node}, and $\reject$
  (\textbf{Reject Node}).
\item $\beta$ : Symbol used to denote a particular node of $M$.
  Often used to refer to the successor of a particular $\eta$.  In
  such cases we may write $\beta_\eta$.  In the case that $\eta$ is
  a branch node, we write $\beta_\eta^+$ to distinguish the node
  associated with $h_\eta(x) > 0$ and $\beta_\eta^-$ to distinguish
  the node associated with $h_\eta(x) < 0$. (For unordered rings,
  replace the above conditions with $h_\eta = 0$ and $h_\eta \neq
  0$, respectively.)
\item $\fullstatem{M}$ (\textbf{Configuration Space}):
\item $z$ : Symbol used to refer to a machine \textbf{configuration}
  (i.e. a pair in $\fullstate$).  Completely determines the remainder
  of any computation performed by $M$.
\end{itemize}
\section{Properties of Computations}
\begin{itemize}
\item $\computefn_M$ (\textbf{Computation Function}): Function from
  $\inspace_M \rightarrow \outspace_M$ defined by $M$.
\item $\compute_M(x)$ (\textbf{Computation}): Sequence of
  configurations in $\fullstatem{M}$ defined by the computation of $M$
  on input $x$.  For NDET machines, $\compute_M(x)$ is a set of such
  paths.
\item $\computeout{M}(x)$ (\textbf{Computation Output}): Final value
  of a halting computation.
\item $\pathprojection$: Projection from $\fullstate{M}$ onto the
  first coordinate.
\item $\stateprojection$: Projection from $\fullstatem{M}$ onto the
  second coordinate.
\item $\computepath_M(x)$ (\textbf{Computation Path}): The projection
  of $\compute_M(x)$ under $\pathprojection$, (i.e., the sequence of
  nodes traversed by $M$ while computing $\compute_M(x)$.
\item $\coincidence{\computepath}$ (\textbf{Coincidence Set}): The set
  of values in $\inspace_M$ whose initial computation paths coincide
  with $\computepath$
\item $\halting_M$ (\textbf{Halting Set}): Set of values in
  $\inspace_M$ for which the $\compute_M(x)$ is finite.  When
  parameterized as $\halting_M^T$, denotes the \textbf{time-T Halting
    Set} of $M$, defined by the set of values in $\inspace_M$ for
  which $\compute_M(x)$ has length less than or equal to $T$.
\item $\allpaths$ (\textbf{Path Set}): Symbol used to refer to a set
  of paths.  Notable examples include $\allpaths(M, x)$, the set of
  valid paths for an NDET Machine $M$ on input $x$; $\accpaths(M, x)$,
  the set of accepting paths for $M$ on input $x$; and $\rejpaths(M,
  x)$, the set of rejecting paths for $M$ on input $x$.
\item $\charfunc_S$ (\textbf{Characteristic Function}): Function
  defined on some decision problem $S$ such that $$\charfunc_S(x) =
  \twopartdef{1}{x \in S_{yes}}{0}{x \in S_{no}}$$
\end{itemize}
