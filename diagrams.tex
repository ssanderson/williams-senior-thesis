\documentclass[landscape]{article}
\usepackage{thesiscommands}
\usepackage{makecell}
\usepackage{setspace}
\usepackage{mydiagrams}

\begin{document}

% Newton's Method - High Level 
% {\Huge Newton's Method - High Level}\\

% \begin{tikzpicture}[node distance=4cm]

%   \node (init)   [input, draw] {Input: $p \in \reals[x]$, $z \in \reals$, $\epsilon \in \reals$};
%   \node (comp)   [compute, draw, below of=init]   {$z = z - \frac{p(z)}{p'(z)}$};
%   \node (check)  [branch, draw, below of=comp]    {$|p(x)| - \epsilon < 0$};
%   \node (output) [accept, draw, below of=check]   {Output $z$};
  
%   \draw [->, thick] (init) --                                      (comp);
%   \draw [->, thick] (comp) --                                      (check);
%   \draw [->, thick] (check) to [out=0, in=0] node[auto, swap] {No} (comp);
%   \draw [->, thick] (check) to node[auto] {Yes}                    (output);

% \end{tikzpicture}

\newpage

% Newton's Method - Lower Level

% Newton Machine for 5th-degree polynomials over $\reals$ considered as
% an ordered field.  $\inspace_M = \statespace_M = \reals^8$.
% $\outspace_M = \reals$ The coordinate $x_1$ represents our current
% point $z$, $x_2$ represents $\epsilon$, and the rest of the
% coordinates represent the coefficients of the polynomial. In the
% computation node, we compute $z = \frac{p(x)}{p'(x)}$, and we branch
% on the condition $x_0^2 < \epsilon^2$, which is equivalent to $|x_0| <
% \epsilon$.

% \begin{figure}[p]
%   \centering
  
%   \begin{tikzpicture}[node distance=4cm, label distance=.5cm]

%     \node(init) [input, draw, label=left:\Large $\eta_1$] {Input: $x = (x_1, x_2, x_3, x_4, x_5, x_6, x_7, x_8)$;};
%     \node(comp) [compute, draw, label=left:\Large $\eta_2$, below of=init] {
%       $g(x) = (x_1 - 
%       \frac{x_3x_1^5 + x_4x_1^4+ x_5x_1^3 + x_6x_1^2 + x_7x_1 + x_8}
%       {5x_3x_1^4 + 4x_4x_1^3 + 3x_5x_1^2 + 2x_6x_1 + x_7}, 
%       x_2, x_3, x_4, x_5, x_6, x_7, x_8)$};
%     \node(check)[branch, draw, label=left:\Large $\eta_3$, below of=comp] {$h(x) = x_1^2 - x_2^2$};
%     \node(output)[accept, draw, label=left:\Large $\nodeout$,below of=check] {Output: $O(x) = x_0$};

%     \draw [->, thick] (init) --                                             (comp);
%     \draw [->, thick] (comp) --                                            (check);
%     \draw [->, thick] (check) to [out=0, in=0] node[auto, swap] {$h(x) > 0$}(comp);
%     \draw [->, thick] (check) to node[auto] {$h(x) < 0$}                  (output);

%     \node (legend) [rectangle, draw, fill=black!10, right of=output, node distance=6cm] {
%       \makecell[l]{
%         $\nodes = \set{\eta_1, \eta_2, \eta_3, \nodeout}$\\
%         $\inspace_M = \statespace_M = \outspace_M = \reals^8$\\
%         Computation Nodes: $\set{\eta_2}$\\
%         Branch Nodes: $\set{\eta_3}$
%       }
%     };
    
%   \end{tikzpicture}

%   \caption{Newton Machine - Low Level}
%   \label{fig:newton-low}
% \end{figure}

% \newpage
% \centering
% \begin{tikzpicture}[node distance=4.8cm]

%   \node(init) [input, draw, label=left:\Large $\eta_1$] 
%   {Input: x = $(x_1, x_2, x_3)$};
%   \node(test) [branch, draw, below of=init, label=45:\Large $\eta_2$] 
%   {$h(x) = x_1 - \frac{x_2}{x_3}$};
%   \node(flip) [compute, draw, left of=test, label=left:\Large $\eta_3$,]
%   {$g(x)= (x_1, 1, x_2+1)$};
%   \node(fliptest) [branch, draw, below of=flip, label=left:\Large $\eta_4$] 
%   {$h(x) = x_3$};
%   \node(inc)  [compute, draw, below of=fliptest, label=left:\Large $\eta_5$,] 
%   {$g(x) = (x_1, x_2+1, x_3-1)$};
%   \node(output)   [accept, draw, below of=test, right of=test, label=left:\Large $\nodeout$] 
%   {Output: $O(x) = (x_2, x_3)$};

%   \draw[->, thick] (init) -- (test);
%   \draw[->, thick] (flip) -- (test);
%   \draw[->, thick] (test) to [out=270, in=0] node [auto] {$h(x) \neq 0$} (inc);
%   \draw[->, thick] (test) to [out=0, in=90] node[auto] {$h(x)=0$} (output);
%   \draw[->, thick] (fliptest) to [out=0, in= 225] node[auto] {$h(x) = 0$} (test);
%   \draw[->, thick] (fliptest) to [out=90, in= 270] node[auto] {$h(x) \neq 0$}(flip);
%   \draw[->, thick] (inc) -- (fliptest);

%   \node (legend) [rectangle, draw, fill=black!10, below of=output] {
%     \makecell[l]{
%       $\nodes = \set{\eta_1, \eta_2, \eta_3, \eta_4, \eta_5, \nodeout}$\\
%       $\inspace_M = \reals, \statespace_M = \reals^3, \outspace_M = \reals^2$\\
%       Computation Nodes: $\set{\eta_3, \eta_5}$\\
%       Branch Nodes: $\set{\eta_2, \eta_4}$\\
%       $\halting_M = \rationals^+$\\
%       $\computefn_M(x) = (p,q) \mid x = \frac{p}{q}$
%     }
%   };

% \end{tikzpicture}

\mandelrec{}

\end{document}

