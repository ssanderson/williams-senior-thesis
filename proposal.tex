\documentclass{article}
% Change "article" to "report" to get rid of page number on title page
\usepackage{amsmath,amsfonts,amsthm,amssymb}
\usepackage{setspace}
\usepackage{fancyhdr}
\usepackage{lastpage}
\usepackage{extramarks}
\usepackage{chngpage}
\usepackage{soul}
\usepackage[usenames,dvipsnames]{color}
\usepackage{graphicx,float,wrapfig}
\usepackage{ifthen}
\usepackage{listings}
\usepackage{courier}

\definecolor{MyDarkGreen}{rgb}{0.0,0.4,0.0}

% For faster processing, load Matlab syntax for listings
\lstloadlanguages{Matlab}%
\lstset{language=Matlab,
        frame=single,
        basicstyle=\small\ttfamily,
        keywordstyle=[1]\color{Blue}\bf,
        keywordstyle=[2]\color{Purple},
        keywordstyle=[3]\color{Blue}\underbar,
        identifierstyle=,
        commentstyle=\usefont{T1}{pcr}{m}{sl}\color{MyDarkGreen}\small,
        stringstyle=\color{Purple},
        showstringspaces=false,
        tabsize=5,
        % Put standard MATLAB functions not included in the default
        % language here
        morekeywords={xlim,ylim,var,alpha,factorial,poissrnd,normpdf,normcdf},
        % Put MATLAB function parameters here
        morekeywords=[2]{on, off, interp},
        % Put user defined functions here
        morekeywords=[3]{FindESS},
        morecomment=[l][\color{Blue}]{...},
        numbers=left,
        firstnumber=1,
        numberstyle=\tiny\color{Blue},
        stepnumber=5
        }

% In case you need to adjust margins:
\topmargin=-0.45in      %
\evensidemargin=0in     %
\oddsidemargin=0in      %
\textwidth=6.5in        %
\textheight=9.0in       %
\headsep=0.25in         %

% Homework Specific Information
\newcommand{\hmwkTitle}{}
\newcommand{\hmwkSubTitle}{}
\newcommand{\hmwkDueDate}{}
\newcommand{\hmwkClass}{Thesis Proposal}
\newcommand{\hmwkClassTime}{}
\newcommand{\hmwkClassInstructor}{}
\newcommand{\hmwkAuthorName}{Scott Sanderson}

% Setup the header and footer
\pagestyle{fancy}                                                       %
\lhead{\hmwkAuthorName}                                                 %
\chead{\hmwkClass \hmwkTitle}                                           %
\rhead{Advisor: Brent Heeringa}                                         %
\lfoot{\lastxmark}                                                      %
\cfoot{}                                                                %
\rfoot{Page\ \thepage\ of\ \protect\pageref{LastPage}}                  %
\renewcommand\headrulewidth{0.4pt}                                      %
\renewcommand\footrulewidth{0.4pt}                                      %

%%%%%%%%%%%%%%%%%%%%%%%%%%%%%%%%%%%%%%%%%%%%%%%%%%%%%%%%%%%%%
% Make title
\title{\vspace{2in}\textmd{\textbf{\hmwkClass:\ \hmwkTitle\ifthenelse{\equal{\hmwkSubTitle}{}}{}{\\\hmwkSubTitle}}}\\\normalsize\vspace{0.1in}\small{Due\ on\ \hmwkDueDate}\\\vspace{0.1in}\large{\textit{\hmwkClassInstructor\ \hmwkClassTime}}\vspace{3in}}
\date{}
\author{\textbf{\hmwkAuthorName}}
%%%%%%%%%%%%%%%%%%%%%%%%%%%%%%%%%%%%%%%%%%%%%%%%%%%%%%%%%%%%%

% This is used to trace down (pin point) problems
% in latexing a document:
%\tracingall

%%%%%%%%%%%%%%%%%%%%%%%%%%%%%%%%%%%%%%%%%%%%%%%%%%%%%%%%%%%%%
% Some tools

% Use these to play with the headers and footers
\newcommand{\enterProblemHeader}[1]{\nobreak\extramarks{#1}{#1 continued on next page\ldots}\nobreak%
                                    \nobreak\extramarks{#1 (continued)}{#1 continued on next page\ldots}\nobreak}%
\newcommand{\exitProblemHeader}[1]{\nobreak\extramarks{#1 (continued)}{#1 continued on next page\ldots}\nobreak%
                                   \nobreak\extramarks{#1}{}\nobreak}%

\setcounter{secnumdepth}{0}
\newcommand{\homeworkProblem}[3]
 {\enterProblemHeader{#1}
   \ifthenelse{\equal{#2}{}}%
        {\subsection{#1}}%
        {\subsection{#1 (#2)}}%
               #3
   \exitProblemHeader{#1}}

%%% I think \captionwidth (commented out below) can go away
%%%
%% Edits the caption width
%\newcommand{\captionwidth}[1]{%
%  \dimen0=\columnwidth   \advance\dimen0 by-#1\relax
%  \divide\dimen0 by2
%  \advance\leftskip by\dimen0
%  \advance\rightskip by\dimen0
%}

% Includes a figure
% The first parameter is the label, which is also the name of the figure
%   with or without the extension (e.g., .eps, .fig, .png, .gif, etc.)
%   IF NO EXTENSION IS GIVEN, LaTeX will look for the most appropriate one.
%   This means that if a DVI (or PS) is being produced, it will look for
%   an eps. If a PDF is being produced, it will look for nearly anything
%   else (gif, jpg, png, et cetera). Because of this, when I generate figures
%   I typically generate an eps and a png to allow me the most flexibility
%   when rendering my document.
% The second parameter is the width of the figure normalized to column width
%   (e.g. 0.5 for half a column, 0.75 for 75% of the column)
% The third parameter is the caption.
\newcommand{\scalefig}[3]{
  \begin{figure}[ht!]
    % Requires \usepackage{graphicx}
    \centering
    \includegraphics[width=#2\columnwidth]{#1}
    %%% I think \captionwidth (see above) can go away as long as
    %%% \centering is above
    %\captionwidth{#2\columnwidth}%
    \caption{#3}
    \label{#1}
  \end{figure}}

% Includes a MATLAB script.
% The first parameter is the label, which also is the name of the script
%   without the .m.
% The second parameter is the optional caption.
\newcommand{\matlabscript}[2]
  {\begin{itemize}\item[]\lstinputlisting[caption=#2,label=#1]{#1.m}\end{itemize}}

%%%%%%%%%%%%%%%%%%%%%%%%%%%%%%%%%%%%%%%%%%%%%%%%%%%%%%%%%%%%%

%%%%%%%%%%%%%%%%%%%%%%%%%%%%%%%%%%%%%%%%%%%%%%%%%%%%%%%%%%%%%
%%%%%%%%%% The main document content
%%%%%%%%%%%%%%%%%%%%%%%%%%%%%%%%%%%%%%%%%%%%%%%%%%%%%%%%%%%%%

\begin{document}
\begin{spacing}{1.1}
%\maketitle
% Uncomment the \tableofcontents line to get a Contents on front page
% \tableofcontents
  \newpage
  
  \fontsize{12pt}{12pt}\selectfont

  \hspace{3pt} Classical Theory of Computation provides a
  framework in which we can pose and answer mathematically rigorous
  questions about the possibility and cost of solving problems by way
  of algorithmic procedures.  Central to this theory has been the
  Turing Machine model of computation, in which the memory of of an
  ideal computing machine is represented by an infinitely long tape
  containing finitely many symbols at any given point in time.  One
  shortcoming of this computing model is that it does not naturally
  allow us to analyze computational problems that are naturally
  defined over uncountable sets such $\mathbb{R}$ or $\mathbb{C}$.
  Examples of such problems include the expected runtime of Newton's
  Method, determining membership in the Mandelbrot Set, and the
  Hilbert Nullstellensatz, which asks whether a finite set of
  polynomials in $n$ variables share a common root.  \\

  In their 1989 paper, \emph{On a Theory of Computation and Complexity
    Over the Real Numbers: NP-Completeness, Recursive Functions, and
    Universal Machines}, Blum, Shub, and Smale introduce a model of
  computation defined over arbitrary rings.  Along with Felipe Cucker,
  the same authors further develop their theory in their 1998 book,
  \textit{Complexity and Real Computation}.  An important feature of
  the model due to Blum et al. is a characterization of the complexity
  classes P and NP \emph{relative to a ring $R$}.  Their definition
  serves as a natural generalization of the characterization of NP in
  the classical theory as the set of problems for which there exist
  efficient (ie, polynomial time) verifiers for candidate solutions.
  There exist, however, a great variety of alternative definitions of
  NP in the classical theory.  For many of these definitions, it is
  not immediately apparent that they define the same class of
  problems. For example, in the classical theory of computation, the
  class of efficiently verifiable problems is precisely the class of
  problems that can be efficiently computed by a nondeterministic
  Turing machine.  It is not clear whether this result generalizes
  to the arbitrary algebraic case. \\

  For my thesis, I intend to investigate the relationships between
  alternative characterizations of NP over arbitrary rings, using the
  model developed by Blum et al. as a jumping off point.  In
  particular, I will consider the class of problems efficiently
  solvable by Blum machines altered to allow for nondeterministic
  computation in a manner analogous to the standard definition of a
  nondeterministic Turing machine.  This problem will require either
  demonstrating the existence of problems which fall in one class and
  not the other, or a novel equivalence proof; the proof of
  equivalence of the analogous classes defined over standard Turing
  machines relies on an upper bound on the number of possible
  solutions that no longer holds for uncountably infinite inputs.

\end{spacing}
\end{document}

%%%%%%%%%%%%%%%%%%%%%%%%%%%%%%%%%%%%%%%%%%%%%%%%%%%%%%%%%%%%%

%----------------------------------------------------------------------%
% The following is copyright and licensing information for
% redistribution of this LaTeX source code; it also includes a liability
% statement. If this source code is not being redistributed to others,
% it may be omitted. It has no effect on the function of the above code.
%----------------------------------------------------------------------%
% Copyright (c) 2007, 2008, 2009, 2010, 2011 by Theodore P. Pavlic
%
% Unless otherwise expressly stated, this work is licensed under the
% Creative Commons Attribution-Noncommercial 3.0 United States License. To
% view a copy of this license, visit
% http://creativecommons.org/licenses/by-nc/3.0/us/ or send a letter to
% Creative Commons, 171 Second Street, Suite 300, San Francisco,
% California, 94105, USA.
%
% THE SOFTWARE IS PROVIDED "AS IS", WITHOUT WARRANTY OF ANY KIND, EXPRESS
% OR IMPLIED, INCLUDING BUT NOT LIMITED TO THE WARRANTIES OF
% MERCHANTABILITY, FITNESS FOR A PARTICULAR PURPOSE AND NONINFRINGEMENT.
% IN NO EVENT SHALL THE AUTHORS OR COPYRIGHT HOLDERS BE LIABLE FOR ANY
% CLAIM, DAMAGES OR OTHER LIABILITY, WHETHER IN AN ACTION OF CONTRACT,
% TORT OR OTHERWISE, ARISING FROM, OUT OF OR IN CONNECTION WITH THE
% SOFTWARE OR THE USE OR OTHER DEALINGS IN THE SOFTWARE.
%----------------------------------------------------------------------%