%% Macros and packages used in my thesis.
%% Include this file to typeset any portion...


\usepackage{setspace}

%% Special math fonts and symbols
\usepackage{amssymb}
\usepackage{amsfonts}
\usepackage{amsmath}
\usepackage{amsthm}
%% Rotate tables and figures
\usepackage{rotating}
%% Used for TODO items
\usepackage{color}
%% used for code listings.
\usepackage{float}
%% Used to replace LaTeX's ugly emptyset with diameter, which looks nicer.
\usepackage{wasysym}
%% Nicely formatted algorithms.
\usepackage{algorithmicx}
\usepackage[chapter]{algorithm}
\usepackage{algpseudocode}
%% Nicely formatted listings.
\usepackage{listings}
%% More kinds of arrow with stuff
\usepackage{empheq}
\usepackage{multicol}
\usepackage{subfigure}

%%%%%%%%%%%%%%%%%%%%%%%%%%%%%
%% Macros                  %%
%%%%%%%%%%%%%%%%%%%%%%%%%%%%%

%% The name of the commandline tool.
%%\newcommand{\HominyTool}{\texttt{hominy}}
\newcommand{\HominyTool}{Hominy}
\newcommand{\JG}{JG}


\definecolor{gray}{rgb}{0.5,0.5,0.5}

%% Todo items.
%\newcommand{\todo}[1]{\textcolor{red}{\textbf{\textsf{TODO:}}} \textsf{#1}}

%% Paragraph shorthand
% \newcommand{\p}[1]{\paragraph{#1}}

%% Reference a figure
\newcommand{\fig}[1]{Figure~\ref{#1}}

%% Reference a table
\newcommand{\tab}[1]{Table~\ref{#1}}

%% Code listings, algorithms
%% \floatstyle{ruled}
%% \newfloat{listing}{htbp}{lop}[chapter]
%% \floatname{listing}{Listing}
%% %\newfloat{algorithm}{htbp}{lop}
%% %\floatname{algorithm}{Algorithm}
%% \newfloat{example}{htbp}{lop}[chapter]
%% \floatname{example}{Example}


%% Reference a code listing
%% \newcommand{\code}[1]{Listing~\ref{#1}}

%% Reference an algorithm
\newcommand{\alg}[1]{Algorithm~\ref{#1}}

%% Reference an algorithm
%% \newcommand{\ex}[1]{Example~\ref{#1}}

%% Reference a chapter
\newcommand{\ch}[1]{Chapter~\ref{#1}}

%% A better empty set
%% \renewcommand{\emptyset}{\diameter
\renewcommand{\emptyset}{\varnothing}

%%%%%%%%%%%%%%%%%%%%%%%
%% Theorem environments
%%%%%%%%%%%%%%%%%%%%%%%

\theoremstyle{definition}

\newtheorem{theorem}{Theorem}[chapter]
\newtheorem{lemma}{Lemma}[chapter]
\newtheorem{proposition}{Proposition}[chapter]
\newtheorem{claim}{Claim}[chapter]
\newtheorem{corollary}{Corollary}[chapter]
\newtheorem{definition}{Definition}[section]
\renewenvironment{proof}{{\bf Proof:}}{\hfill\rule{2mm}{2mm}}
\newenvironment{proofsketch}{{\bf Proof Sketch:}}{\hfill\rule{2mm}{2mm}}

%% #1 => label, #2 => caption
\newenvironment{LFigure}[2]{
\begin{figure}
\hrule \smallskip
\newcommand{\Caption}{\caption{#2}\label{#1}}
\singlespacing
}{
\smallskip \hrule
\Caption
\end{figure}
}
\newenvironment{Figure}[2]{
\begin{LFigure}{#1}{#2}
\centering
}{\end{LFigure}}

\newenvironment{UFigure}[2]{
\begin{figure}
\newcommand{\Caption}{\caption{#2}\label{#1}}
\singlespacing
\centering
}{
\smallskip
\Caption
\end{figure}
}
\newenvironment{FigureHere}[2]{
\begin{figure}[h]
\hrule \smallskip
\newcommand{\Caption}{\caption{#2}\label{#1}}
\singlespacing
\centering
}{
\smallskip \hrule
\Caption
\end{figure}
}

\newenvironment{PFigure}[2]{
\begin{figure}[p]
\hrule \smallskip
\newcommand{\Caption}{\caption{#2}\label{#1}}
\singlespacing
\centering
}{
\smallskip \hrule
\Caption
\end{figure}
}

%% Lockset notation (for Eraser)
\newcommand{\ls}[1]{\textit{ls}(#1)}

%% happened before!
\newcommand{\hb}{\textit{happened before} }

%% Easier set notation
% \newcommand{\set}[1]{\{#1\}}

%% Vertical bar with space.
\newcommand{\vbar}{\ |\ }

\newcommand{\Vol}[1]{\hat{#1}}
\newcommand{\Forker}[1]{\ddot{#1}}
\newcommand{\Joiner}[1]{\mathring{#1}}

%% Specs
\newcommand{\Spec}[1]{\texttt{#1}}
\newcommand{\SpecWithArg}[2]{\Spec{#1~\ensuremath{#2}}}

\newcommand{\ThreadLocal}{\Spec{thread-local}}
\newcommand{\ReadShared}{\Spec{read-shared}}
\newcommand{\GuardedBy}[1]{\SpecWithArg{guarded-by}{#1}}
\newcommand{\GuardedBySpec}{\SpecWithArg{guarded-by}{\textrm{\Lock}}}
\newcommand{\InitTrans}{\Spec{init}}
\newcommand{\ForkTrans}{\Spec{fork}}
\newcommand{\JoinTrans}{\Spec{join}}
\newcommand{\VolBarrier}[1]{\SpecWithArg{vol}{\Vol{#1}}}
\newcommand{\VolBarrierSpec}{\SpecWithArg{vol}{\textrm{\VolVar}}}
%% \newcommand{\LockBarrier}[1]{\SpecWithArg{barrier}{#1}}
\newcommand{\MultiLock}[1]{\GuardedBy{\set{#1}}}
\newcommand{\Transition}[1]{\texttt{;\ [} #1 \texttt{]\ }}


%% #1: action name #2: action thread
\newcommand{\SimpleAction}[2]{\textit{#1}_{#2}}

\newcommand{\ReadOrWrite}{\textit{acc}}
\newcommand{\OrderedAccessAction}[2]{#2 \rhd #1}
\newcommand{\OrderedAction}[3]{\OrderedAccessAction{\SimpleAction{#1}{#2}}{#3}}
%% #1: thread  #2: order-set
\newcommand{\OrderedRead}[2]{\OrderedAction{rd}{#1}{#2}}
\newcommand{\OrderedWrite}[2]{\OrderedAction{wr}{#1}{#2}}
\newcommand{\OrderedAccess}[2]{\OrderedAction{\ReadOrWrite}{#1}{#2}}
%% \newcommand{\OrderedReadOrWrite}[2]{\OrderedAction{$(\textit{rd}\cdot\textit{wr})$}{#1}{#2}}

%% \newcommand{\OrderedReadBySet}[2]{\OrderedAction{rd}{#1}{#2}}
%% \newcommand{\OrderedWriteBySet}[2]{\OrderedAction{wr}{#1}{#2}}
%% \newcommand{\OrderedAccessBySet}[2]{\OrderedAction{$(\textit{rd}\cdot\textit{wr})$}{#1}{#2}}
%% \newcommand{\OrderedReadOrWriteBySet}[2]{\OrderedAction{$(\textit{rd}\cdot\textit{wr})$}{#1}{#2}}

%% #1: action name  #2: acting thread  #3: args
\newcommand{\Act}[3]{\ensuremath{\SimpleAction{#1}{#2}(#3)}}

%% #1: thread  #2: field
\newcommand{\Rd}[2]{\Act{rd}{#1}{#2, v}} %% v!
\newcommand{\Wr}[2]{\Act{wr}{#1}{#2, v}} %% v!
\newcommand{\RdValue}[3]{\Act{rd}{#1}{#2,#3}}
\newcommand{\WrValue}[3]{\Act{wr}{#1}{#2,#3}}
\newcommand{\Rdv}[2]{\Act{rd}{#1}{\Vol{#2},v}}
\newcommand{\Wrv}[2]{\Act{wr}{#1}{\Vol{#2},v}}
\newcommand{\RdvValue}[3]{\Act{rd}{#1}{\Vol{#2},#3}}
\newcommand{\WrvValue}[3]{\Act{wr}{#1}{\Vol{#2},#3}}

%% #1: thread  #2: sync device
\newcommand{\Acq}[2]{\Act{acq}{#1}{#2}}
\newcommand{\Rel}[2]{\Act{rel}{#1}{#2}}
\newcommand{\Wait}[2]{\Act{wait}{#1}{#2}}
\newcommand{\Notify}[2]{\Act{notify}{#1}{#2}}
\newcommand{\ForkThread}[2]{\Act{fork}{#1}{#2}}
\newcommand{\JoinThread}[2]{\Act{join}{#1}{#2}}

%% Standard program step
\newcommand{\ProgStep}[1]{\xrightarrow{\ #1\ }}

%% Instrumented step
\newcommand{\SimpleInsStep}[1]{\xRightarrow{\ #1\ }}
\newcommand{\InsStep}[1]{\xRightarrow{\makebox[8ex]{\scriptsize $#1$}}}

%%Events
\newcommand{\Ev}[3]{\langle #1 \gtrdot_{\textit{#2}} #3 \rangle}
%% \AccessEvent{thread}
\newcommand{\AccessEvent}[1]{\langle#1\gtrdot_{\ReadOrWrite}#1\rangle}
%% \WildcardEvent
\newcommand{\WildcardEvent}[1]{\Ev{?}{?}{#1}}
%% \AcqEvent{thread}{lock}
\newcommand{\AcqEvent}[2]{\Ev{#2}{acq}{#1}}
%% \RelEvent{thread}{lock}
\newcommand{\RelEvent}[2]{\Ev{#1}{rel}{#2}}
%% \WaitEvent{thread}{lock}
%% \newcommand{\WaitEvent}[2]{\Ev{#2}{wait}{#1}}
%% \NotifyEvent{thread}{lock}
%% \newcommand{\NotifyEvent}[2]{\Ev{#1}{notify}{#2}}
%% \ForkEvent{thread}{thread_forked}
\newcommand{\ForkEvent}[2]{\Ev{\Forker{#1}}{fork}{#2}}
%% \JoinEvent{thread}{thread_joined}
\newcommand{\JoinEvent}[2]{\Ev{\Joiner{#2}}{join}{#1}}
%% \RdvEvent{thread}{lock}
\newcommand{\RdvEvent}[2]{\Ev{\Vol{#2}}{rdv}{#1}}
%% \WrvEvent{thread}{lock}
\newcommand{\WrvEvent}[2]{\Ev{#1}{wrv}{\Vol{#2}}}
%% \WrEvent{thread}
\newcommand{\WrEvent}[1]{\Ev{#1}{wr}{#1}}
%% \RdEvent{thread}
\newcommand{\RdEvent}[1]{\Ev{#1}{rd}{#1}}


\newcommand{\Keyword}[1]{\textsf{\ #1\ }}

\newcommand{\SuchThat}{.\ }
\newcommand{\Forall}[1]{ \forall #1 \SuchThat }


\def\Fill{\ \ \hfill \ \ }
\def\RewriteTo{\leadsto}
\def\Union{\ \cup\ }
\def\Intersect{\ensuremath{\ \cap\ }}
\def\LocalTo{\mathtt{local\_to}\ }
\def\SeparatedBy{\mathtt{separated\_by}\ }
\def\Proves{\vdash}
\newcommand{\ProvesFor}[1]{\Proves_{#1}}

\newcommand{\FreeOf}[1]{\textsf{ is #1-free }}
\newcommand{\SyncThread}[1]{\underline{#1}}

\def\InsStore{\varphi}
\newcommand{\LastWrite}{\varphi_{wr}}
\newcommand{\LastReads}{\varphi_{rd}}

\newcommand{\FunctionName}[1]{\textit{#1}}
\newcommand{\Ancestors}{\FunctionName{ancestors}}
\newcommand{\AllAncestors}{\FunctionName{ancestors}}
\newcommand{\Choose}{\FunctionName{choose}}
\newcommand{\TypeName}[1]{\textit{#1}}

\newcommand{\ForkingTid}{\TypeName{ForkingTid}}
\newcommand{\JoiningTid}{\TypeName{JoinedTid}}
\def\Lock{\TypeName{Lock}}
\def\Tid{\TypeName{Tid}}
\def\Var{\TypeName{Variable}}
\def\VolVar{\TypeName{VolatileVariable}}
\def\GlobalStore{\TypeName{GlobalStore}}
\def\Value{\TypeName{Value}}
\def\LocalStore{\TypeName{LocalStore}}
\def\LocalStores{\TypeName{LocalStores}}
\def\Action{\TypeName{Action}}
\def\Op{\TypeName{Operation}}
\def\ProgramState{\TypeName{State}}
%\def\GoldiSet{\TypeName{GoldiSet}}
\def\EventSet{\TypeName{EventSet}}
\def\SyncDevice{\TypeName{SyncDevice}}
%\def\GoldiElem{\TypeName{GoldiElem}}
\def\Event{\TypeName{Event}}
\def\OrderSet{\TypeName{OrderSet}}
\def\InsStoreName{\TypeName{InsStore}}
\def\LastWriteName{\TypeName{LastWrite}}
\def\LastReadsName{\TypeName{LastReads}}
\def\ExtendedGoldiStore{\TypeName{ExtendedGoldiStore}}

\newcommand{\Select}[1]{\ \big|_{#1}}

\newcommand{\RuleName}[1]{\textsc{[#1]}}
%% Type Rule. Use \\ and \hline to sep rows.
\newenvironment{Rule}[1]
	       {
		 \noindent\RuleName{#1}\\[1ex]
		 \begin{math}
		   \begin{array}{c}
	       }
	       {
		   \end{array}
		 \end{math}\\[2ex]
	       }


\newcommand{\ExtendedUpdateRule}[5]
{
\begin{Rule}{#1}
  a \in \set{#2}\\
   \Forall{x} \LastWrite'(x) = \left\{\begin{array}{ll}
       \LastWrite(x) \Union \set{\Ev{#4}{#5}{#3}} & \Keyword{if} \WildcardEvent{#4} \in x\\
          \LastWrite(x) & \Keyword{otherwise}
	  \end{array}\right. \\
  \Forall{x, u} \LastReads'(x,u) = \left\{\begin{array}{ll}
      \LastReads(x,u) \Union \set{\Ev{#4}{#5}{#3}} & \Keyword{if} \WildcardEvent{#4} \in x\\
      \LastReads(x,u) & \Keyword{otherwise}
      \end{array}\right. \\
  \hline
  (\LastWrite, \LastReads) \SimpleInsStep{a} (\LastWrite', \LastReads')
\end{Rule}
}

\newcommand{\SingleExtendedUpdateRule}[5]
{
\begin{Rule}{#1}
   \Forall{x} \LastWrite'(x) = \left\{\begin{array}{ll}
       \LastWrite(x) \Union \set{\Ev{#4}{#5}{#3}} & \Keyword{if} \WildcardEvent{#4} \in x\\
          \LastWrite(x) & \Keyword{otherwise}
	  \end{array}\right. \\
	    \Forall{x, u} \LastReads'(x,u) = \left\{\begin{array}{ll}
	          \LastReads(x,u) \Union \set{\Ev{#4}{#5}{#3}} & \Keyword{if} \WildcardEvent{#4} \in x\\
		        \LastReads(x,u) & \Keyword{otherwise}
			      \end{array}\right. \\
  \hline		      
  (\LastWrite, \LastReads) \InsStep{#2} (\LastWrite', \LastReads')
\end{Rule}
}


%% #1: rule name, #2: event type, #3: thread, #4: dev
\newcommand{\SendUpdateRule}[4]{
\begin{Rule}{#1}
  \Forall{x} \InsStore'(x) = \begin{cases}
	  \InsStore(x) \Union \set{\Ev{#3}{#2}{#4}} & \Keyword{if} \WildcardEvent{#3} \in \InsStore(x)\\
	  \InsStore(x) & \Keyword{otherwise}
	  	  \end{cases} \\
  \hline
  \InsStore \InsStep{\Act{#2}{#3}{#4}} \InsStore'
\end{Rule}

}

%% #1: rule name, #2: event type, #3: thread, #4: dev, #5 link to...
\newcommand{\ReceiveUpdateRule}[5]{
\begin{Rule}{#1}
  \Forall{x} \InsStore'(x) = \left\{\begin{array}{ll}
	  \InsStore(x) \Union \set{\Ev{#4}{#2}{#3}} & \Keyword{if} \WildcardEvent{#4} \in \InsStore(x)\\ %%\Ev{?}{#5}{#4} \in x\\
	  \InsStore(x) & \Keyword{otherwise}
	  	  \end{array}\right. \\
  \hline
  \InsStore \InsStep{\Act{#2}{#3}{#4}} \InsStore'
\end{Rule}

}
