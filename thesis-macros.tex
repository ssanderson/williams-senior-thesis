%% Macros and packages used in my thesis.
%% Include this file to typeset any portion...


\usepackage{setspace}

%% Special math fonts and symbols
\usepackage{amssymb}
\usepackage{amsfonts}
\usepackage{amsmath}
\usepackage{amsthm}
%% Rotate tables and figures
\usepackage{rotating}
%% Used for TODO items
\usepackage{color}
%% used for code listings.
\usepackage{float}
%% Used to replace LaTeX's ugly emptyset with diameter, which looks nicer.
\usepackage{wasysym}
%% Nicely formatted algorithms.
\usepackage{algorithmicx}
\usepackage[chapter]{algorithm}
\usepackage{algpseudocode}
%% Nicely formatted listings.
\usepackage{listings}
%% More kinds of arrow with stuff
\usepackage{empheq}
\usepackage{multicol}
\usepackage{subfigure}

%%%%%%%%%%%%%%%%%%%%%%%%%%%%%
%% Macros                  %%
%%%%%%%%%%%%%%%%%%%%%%%%%%%%%

%% The name of the commandline tool.
%%\newcommand{\HominyTool}{\texttt{hominy}}
\newcommand{\HominyTool}{Hominy}
\newcommand{\JG}{JG}


\definecolor{gray}{rgb}{0.5,0.5,0.5}

%% Todo items.
%\newcommand{\todo}[1]{\textcolor{red}{\textbf{\textsf{TODO:}}} \textsf{#1}}

%% Paragraph shorthand
% \newcommand{\p}[1]{\paragraph{#1}}

%% Reference a figure
\newcommand{\fig}[1]{Figure~\ref{#1}}

%% Reference a table
\newcommand{\tab}[1]{Table~\ref{#1}}

%% Code listings, algorithms
%% \floatstyle{ruled}
%% \newfloat{listing}{htbp}{lop}[chapter]
%% \floatname{listing}{Listing}
%% %\newfloat{algorithm}{htbp}{lop}
%% %\floatname{algorithm}{Algorithm}
%% \newfloat{example}{htbp}{lop}[chapter]
%% \floatname{example}{Example}


%% Reference a code listing
%% \newcommand{\code}[1]{Listing~\ref{#1}}

%% Reference an algorithm
\newcommand{\alg}[1]{\Algorithm~\ref{#1}}

%% Reference an Example.
%% \newcommand{\ex}[1]{Example~\ref{#1}}

%% Reference a chapter
\newcommand{\ch}[1]{Chapter~\ref{#1}}

%% A better empty set
%% \renewcommand{\emptyset}{\diameter
\renewcommand{\emptyset}{\varnothing}
\renewcommand{\algorithmicrequire}{\textbf{Input:}}

%%%%%%%%%%%%%%%%%%%%%%%
%% Theorem environments
%%%%%%%%%%%%%%%%%%%%%%%

\theoremstyle{definition}

\newtheorem{theorem}{Theorem}[chapter]
\newtheorem{lemma}{Lemma}[section]
\newtheorem{proposition}{Proposition}[section]
\newtheorem{claim}{Claim}[section]
\newtheorem{corollary}{Corollary}[section]
\newtheorem{definition}{Definition}[section]
\newtheorem{example}{Example}[section]

\renewenvironment{proof}{{\bf Proof:}}{\hfill\rule{2mm}{2mm}}
\newenvironment{proofsketch}{{\bf Proof Sketch:}}{\hfill\rule{2mm}{2mm}}





%% #1 => label, #2 => caption
\newenvironment{LFigure}[2]{
\begin{figure}
\hrule \smallskip
\newcommand{\Caption}{\caption{#2}\label{#1}}
\singlespacing
}{
\smallskip \hrule
\Caption
\end{figure}
}
\newenvironment{Figure}[2]{
\begin{LFigure}{#1}{#2}
\centering
}{\end{LFigure}}

\newenvironment{UFigure}[2]{
\begin{figure}
\newcommand{\Caption}{\caption{#2}\label{#1}}
\singlespacing
\centering
}{
\smallskip
\Caption
\end{figure}
}
\newenvironment{FigureHere}[2]{
\begin{figure}[h]
\hrule \smallskip
\newcommand{\Caption}{\caption{#2}\label{#1}}
\singlespacing
\centering
}{
\smallskip \hrule
\Caption
\end{figure}
}

\newenvironment{PFigure}[2]{
\begin{figure}[p]
\hrule \smallskip
\newcommand{\Caption}{\caption{#2}\label{#1}}
\singlespacing
\centering
}{
\smallskip \hrule
\Caption
\end{figure}
}




