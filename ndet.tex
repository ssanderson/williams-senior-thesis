\chapter{NDET-Machines and the Class NDETP}
\label{chap:ndet}

\todo{This is meant to be generalizing nondeterministic Turing
  Machines.  Should we give a short intro to them here and/or in the
  background section?}

Although efficient verifiability in terms of witness strings is
arguably the most important and intuitive characterization of NP in
the classical theory, there are a number of other equivalent
definitions.  Another important definition, from which the name NP
derives, comes from the notion of nondeterministic computation (NP
stands for \emph{nondeterministic polynomial time}).  In the classical
theory, a nondeterministic Turing Machine computes in the same manner
as a standard Turing Machine, except there are multiple possible next
configurations associated with each pair of state and tape symbol.  A
nondeterministic TM accepts its input if there is \emph{any} valid
path of configurations that accepts the input.  Somewhat surprisingly,
the functions which can be computed in polynomial time by
nondeterministic Turing Machines turn out to be precisely the same
functions that can be efficiently verified via witness strings.  To
our knowledge, there has been little work done to investigate whether
a similar relationship holds in the generalized theory we have been
investigating.

Here we present a model of nondeterministic computation that naturally
generalizes from the Turing Machine model to the Blum model of
computation.  The basic idea is that each branch node is allowed to
have multiple \textbf{Yes} and \textbf{No} nodes, and an accepting
computation is any computational path through the machine that
respects the branching fufnctions and terminates at a distinguished
accept node.  In keeping with the literature, when we wish to refer to
the machines discussed in the previous section, we use the expression
``BSS Machine'' or ``standard machine''.

\todo{The machines in this section are all decision machines, and the
  theory is a fair amount cleaner if we have accept/reject nodes
  rather than output nodes. Maybe talk about decision vs. computation
  machines in the background section?}

\section{Nondeterministic Computation}

\begin{definition} \textbf{NDET-Machine over R}

  An NDET-Machine over a ring R is a finite, directed, connected graph
  with three types of nodes: \emph{computation}, \emph{shift} and
  \emph{branch}, along with distinguished \emph{input}, \emph{accept}, 
  and \emph{reject} nodes.\\

  Associated with each computation node, $\eta$, is a next node,
  $\beta_{\eta}$, and a polynomial (rational for fields) computation
  map, \functype{g_\eta}{R^\infty}{R^\infty}.\\

  Associated with each shift node, $\eta$ is a next node $\beta_\eta$
  and a shift map, $\sigma_l$ or $\sigma_r$ : $R_\infty \rightarrow
  R_\infty$. Here $\sigma_l$ and $\sigma_r$ are defined in the same
  manner as in the standard machine case.\\

  Associated with each branch node, $\eta$, is a nonzero polynomial
  (rational for fields) function \functype{g_\eta}{R^\infty}{R}, a set
  $\set{\beta^+_1, \beta^+_2, \ldots \beta^+_i}$ of \emph{Yes Nodes}
  and a set $\set{\beta^-_1, \beta^-_2, \ldots \beta^-_j}$ of \emph{No
    Nodes}.  We refer to the max of $i$ and $j$ as the
  \textbf{branching degree} of $\eta$.\\
\end{definition}

\note{The only difference between an NDET Machine and a standard BSS
  machine is that each branch node of an NDET Machine can have
  multiple Yes and No nodes.\\}

The computation of an NDET-machine is defined analogously to that of a
BSS machine, with the caveat that there may be multiple valid
paths continuing from a branch node, and we now accept any input for which
there exists some accepting path through the machine.  More formally,
for each configuration pair $z = (\eta, x)$, we define the
\textbf{next configuration set}, $H(z)$, to be the set of all pairs,
$\set{(\eta', x')}$, such that:

\begin{enumerate}
\item If $\eta$ is a computation node, then $\eta' = \beta_\eta$, and
  $x' = g(x)$.
\item If $\eta$ is a shift node, then $\eta' = \beta_\eta$, and
  $x' = \sigma(x)$.
\item If $\eta$ is a branch node, $x' = x$ and if $g(x) > 0$, $\eta'$
  is a yes-node of $\eta$.  If $g(x) < 0$, $\eta'$ is a no-node of
  $\eta$.
\end{enumerate}

\note{From the definitions above, it is clear that if $z = (\eta, x)$
  is a configuration of some machine $M$ with maximum branching degree
  $k$, then $|H(z)| \leq 1$ if $\eta$ is a non-branch node, and
  $|H(z)| \leq k$ if $\eta$ is a branch node.}

Given a machine $M$ and input $x$, we say that the sequence $z_1 =
(\eta_1, x_1), z_2 = (\eta_2, x_2), \ldots, z_k = (\eta_k, x_k)$ of
configurations of $M$ is \textbf{valid} if $x_1 = x$ and, for all $2
\leq i \leq k$, $z_i \in H(z_{i-1})$.  Such a sequence is said to be
\textbf{accepting} if, additionally, $x_k$ is the accept node of $M$.
It is said to be \textbf{rejecting} if $x_k$ is the reject node of
$M$.  The sequence is said to be \textbf{halting} if it is either
accepting or rejecting.

We denote the (possibly infinite) set of all valid paths for a machine
$M$ on input $x$ by $\allpaths(M, x)$, the set of all accepting paths
for $M$ on $x$ by $\accpaths(M, x)$, and the set of all rejecting
paths for $M$ on $x$ by $\rejpaths(M, x)$.

\begin{definition} \textbf{NDET-Machine Halting, Acceptance and Rejection}

  An NDET-Machine is said to \textbf{halt} on input $x$ if
  $\allpaths(M, x)$ is finite.  It \textbf{accepts} $x$ if it halts
  and $\accpaths(M, x)$ is nonempty.  It \textbf{rejects} if it halts
  and does not accept.
  
\end{definition}

\section{Computation Cost and the Class NDETP}

\begin{definition} \textbf{Nondeterministic Computation Cost}

  Let $\compute = z_1, z_2, \ldots, z_k$ be a halting path for an
  NDET-Machine $M$ on input $x$.  The \textbf{cost} of $\compute_M(x)$ is
  defined in the same manner as in :

  $$\cost(\compute) = k*ht_{max}(x)$$ where $ht_{max}$ is defined as in 
  \textbf{\ref{def:size}}.

  Let $M$ be an NDET-Machine over a ring $R$ with height function
  $ht_R$, and let $M$ halt on input $x$.  The cost of the computation
  of $M$ on $x$ is given by:
  
  $$\max_{\compute \in \allpaths}(\cost(\compute_M(x)))$$

\end{definition}

\begin{definition} $\ndetp$
  
  A language, $S$ is in $\ndetp$ if there exists a machine $M$ and a
  constant $p$ such that, $\forall w \in S$, $M$ accepts w with cost
  $O(size(w)^p)$.
  
\end{definition}

\begin{definition} $\dtime_{f(x)}$ and $\ndettime_{f(x)}$
  A decision problem $S, S_{yes}$ is said to belong to
  $\dtime_{f(x)}$ (deterministic time $f(x)$) if there exists a BSS
  machine that decides $S$ with computation cost bounded by
  $f(size(x))$ for all $x$ in $S_{yes}$.  \\n
  
  A decision problem $S, S_{yes}$ is said to belong to
  $\ndettime_{f(x)}$ (deterministic time $f(x)$) if there exists an
  NDET machine that decides $S$ with computation cost bounded by
  $f(size(x))$ for all $x$ in $S_{yes}$.  \\
\end{definition}

\todo{Use DTIME and NDETTIME to define $\p$, $\np$, $\ndetp$, etc.}

\section{$\ndetp$ and Deterministic Complexity}

In this section we present some results relating the class $\ndetp$
with the deterministic time complexity classes described in Chapter 2.
Most of the proofs presented here are straightforward generalizations
of results from classical complexity theory.\\

Our first result generalizes the classical result that every
nondeterministic time complexity class contains its deterministic
equivalent.  This is unsurprising, since adding the power of
nondeterminism can only make computation of a problem more efficient.
As in the classical case, our proof consists in simply noting that
every deterministic machine is also a valid nondeterministic machine
that happens not to take advantage of the additional power afforded by
nondeterminism.

\todo{Is this even worth listing as a proposition, or should it just
  be a note?  If we keep it as a proposition, should the proof be made
  more formal (e.g. we could proceed by induction on the length of
  $\gamma_x$), or is this acceptable?}

\proposition{Let $R$ be a ring with any height function, and let $(S,
  S_{yes})$ be a decision problem over $R$. If $M$ is a BSS machine
  that decides $S$ in time $t(size(x))$, then $M$ also decides $S$ in
  time $t(size(x))$ when interpreted as an NDET machine.\\}

\proof{ We first note that the only difference between the definitions
  of NDET machines and standard BSS machines is that every branch node
  $\eta$ of BSS machine has a single $\beta_\eta^+$ and a single
  $\beta_{\eta}^-$, whereas each branch node of an NDET machine may
  have multiple yes and no nodes. Thus we can regard standard BSS
  machines as a special case of NDET machines for which every node has
  branching degree 1, and it makes sense to talk about interpreting
  $M$ as an NDET machine.  \\

  Let $x$ be an element of $S$, and let $M$ be a BSS machine that
  decides $S$, now interpreted as an NDET machine. Since $M$ has
  maximum branching degree 1, every configuration of $M$ has at most 1
  one valid next configuration, which implies that there is exactly
  one halting path in $\allpaths(M, x)$.  Moreover, that path is
  precisely $\compute_M(x)$, the sequence of configurations computed
  by $M$ when interpreted as a standard BSS machine.  It follows that
  $\accpaths(M, x)$ (interpreting $M$ as an NDET machine) is nonempty
  if and only if $M$ accepts $x$ when interpreted as a standard
  machine, and under either interpretation the computation cost of $M$
  on $x$ is the cost of $\compute_M(x)$.}

\corollary{Over any ring $R$ with any height function, we
  have: $$\dtime \subseteq \ndettime$$. In particular, we have: $$\p
  \subseteq \ndetp$$.}

Our next result generalizes the classical proof that any
nondeterministic Turing Machine can be simulated by a deterministic
Turing Machine with an exponential slowdown in runtime.  In light of
results from \cite{B98} showing that there are rings with cost
measures over which $\np$ contains undecidable problems, a nontrivial
corollary for us is that every problem in $NTIME_{f(x)}$ is decidable.

\proposition{Let $R$ be a ring with any height function, and let $(S,
  S_{yes})$ be a decision problem over $R$.  If $M$ is an NDET Machine
  that decides $S$ with cost bounded by $t(size(x))$ for all $x$ in
  $S$, then there exists a standard BSS machine, $M'$ that decides $S$
  in time $O(d^{t(size(x))})$, where $d$ is the maximum branching degree of
  any node in $M$.}

\proof{ Algorithm \ref{alg:Simulate-NDET} uses the existence of the
  NDET machine $M$ to decide $S$ with the given bound.  The
  essential idea is that on input $x$ we can breadth-first search
  through the configuration space of $M$, checking all possible
  paths of length less than $t(size(x))$.

  \begin{algorithm} 
    \caption{Simulate-NDET} \label{alg:Simulate-NDET}
    \begin{algorithmic}
      \Require $x \in R_\infty$
      \State $i = 0$
      \State $maxpaths = t(size(x))$
      \State Add configuration $(x, \eta_1)$ to $queue$
      \While{True}
      \State $z$ = $queue.next()$
      \State $i = i+1$
      \If{$i$ > $pathMax$} 
      \State \textbf{reject}
      \EndIf
      \State compute $H_M(z)$
      \For{configuration $(x, \eta)$ in $H_M(z)$} 
      \If{$\eta$ = $\eta_{accept}$}
      \State \textbf{accept}
      \Else add $(x, \eta)$ to $queue$
      \EndIf
      \EndFor
      \EndWhile
    \end{algorithmic}
  \end{algorithm}}  

\todo{Runtime analysis of queue implementation.}

Since every node in $M$ has branching degree $d$ or less, there are
at most $d$ valid next configurations succeeding any configuration,
$z$.  Thus there are at most $d^{t(size(x))}$ paths of length
$t(size(x))$.  In the worst case, which occurs whenever we reject,
we search all such paths, so our worst case runtime is
$O(d^{t(size(x))})$.

\corollary{For any ring $R$ and any cost measure, $\ndettime_{f(x)}
  \subseteq \exptime$}

\corollary{Every problem decidable by an NDET machine is decidable in
  the standard BSS sense.}

The above results shows that nondeterminism does not grant us any more
power as far as computability goes; it ``merely'' gains us at most an
exponential decrease in runtime for decision algorithms.  While this
agrees with the classical case, we note that it is not true in general
for $\np$ over all rings and cost functions.

\section{$\ndetp$ and $\np$}

We now wish to consider the relationship between the $\ndetp$ and the
$\np$ as it has been developed by Blum et al and others.  Since these
classes generalize notions that are equivalent in the classical model,
the natural question to ask is whether the equivalence holds in the
generalized model.  Blum et al. show in \cite{B89} that $\parg{\Z{2}}$
with unit cost and $\parg{\integers}$ are both equivalent to the
classical $\p$, with similar results obtaining for the analogous
$\mathbf{NP}$ classes.  Thus we expect that $\ndetp$ and $\np$ should
coincide for $R = \Z{2}$.  We prove the slightly more general result
that $\ndetp = \np$ for any ring $R$ with height function $ht_R$ such
that, for any $k \in R$ there are at most finitely many elements of
$R$ Our proof follows the classical case by showing that $\ndetp$
machines can match the computational power of $\np$ machines by simply
generating all possible witnesses.  This argument depends crucially on
the fact that for a finite ring there can be at most exponentially
many witnesses of a given length.  Since this is no longer the case
for infinite rings, it becomes less clear under what conditions the
equality still holds.

\todo{Better transition.}

\theorem{When considered with unit cost, $\nparg{\integers}
  \not\subset \decarg{\integers}$}

\proof{ It is easy to formulate Hilbert's Tenth Problem, which asks if
  an arbitrary diophantine equation has solutions over the integers,
  as a decision problem over $\integers$ It is also easy to show that
  this problem is in $\np$, since a witness for a solvable diophantine
  equation is the solution in queston. Since it is well known that
  Hilbert's Tenth Problem is undecidable we have our desired result.
  For more details see \cite{B98}, Chapter 6.}

The above result shows that the classical equivalence of efficient
verifiability and nondeterministic computability is not necessarily
valid for rings other than $\integers_2$.  Two natural questions
arise concerning the relationship between $\ndetp$ and $\np$.  

\begin{itemize}
\item Is it always the case that $\ndetp \subseteq \np$.
\item Under what circumstances is it the case that $\ndetp = \np$?
\end{itemize}

We give an affirmative answer to the first question.  This is
because, given an NDET Machine $M$ that decides a set $S$, we can
produce an NP Machine $M'$ that also decides $S$ by labelling each
node in $M$ by a unique value (or series of values if necessary)
from $R$ and replacing each branch node of $M$ with a subroutine
that reads the next unread value in the witness $w$, and transitions
to the node labelled by that integer if it is a valid next node for
the current configuration. For such a machine, an accepting witness
is one that encodes the sequence of nondeterministic choices made by
$M$ on some accepting path.

\theorem{Let $S$ be a decision problem over a ring $R$ with height
  function $ht_R$, and let $M$ be an NDET machine that decides $M$ in
  time bounded by $O(T(\size(x)))$.  Then there exists an NP machine,
  $M'$ which also decides $S$ in time $O(T(\size(x)))$}

\todo{I think this might work better written out in algorithm style
  like the one above.  Also, should the statement about assuming
  branching degree 2 be broken out into a Lemma?}

\proof{Let $S$ and $M$ be as stated above.  We assume without loss of
  generality that $M$ has maximum branching degree 2, since we can
  always replace a branch node of higher branching degree with a chain
  of branch nodes of degree 2, and doing so increases the runtime of
  $M$ by at most a multiplicative constant factor of $d$.  We
  construct the NP machine $M'$ by starting with $M$ and replacing
  each branch node $\eta$ of $M$ with a subroutine that first computes
  the branching check associated with $\eta$, then queries the witness
  string $w$ and decides which of the possible next nodes to
  transition into based on the value stored in the first coordinate of
  $w$.}

Toward answering the second question, we show that a sufficient
condition for $\ndetp = \np$ is that all values in $R^\infty$ of
size $s$ or less be enumerable by an NDET machine in
nondeterministic time $O(log s)$.  We conjecture that this condition
is also sufficient.  

\theorem{Let $R$ be a ring with associated height function
  satisfying the property that there exists an NDET Machine that
  enumerates every value in $R^\infty$ with size at most $s$ in
  nondeterministic time $O(\log s)$.

  \proof{
    Let $S \in \np$.  Then there exists a machine $M$ such that for
    every $x$ in $S$ there exists a $w$ such that $\computefn_M(x, w)
    = 1$ and $cost_M(x, w) \leq p(size(x))$, where p is some
    polynomial function.  It immediately follows that $w$ must have
    size less than $p(size(x))$, or else computing a single step with
    $w$ in a register of $\statespace_M$ would render the cost of
    computing a single step to be greater than the given bound on the
    runtime of $M$.  Thus we can construct an NDET-Machine which, on
    input $x$, nondeterministically generates all possible witnesses
    of length at most $p(size(x))$, and then for each possible witness
    $w$ simulates $M$ on input $w$.
  }

\todo{Section on digital nondeterminism equivalence and DNP-Complete problems.}