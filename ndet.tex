\chapter{NDET-Machines and the Class NDETP}

Though efficient verifiability in terms of witness strings is arguably
the most important and intuitive characterization of NP in the
classical theory, there are a number of other equivalent definitions.
Another important definition, from which the name NP derives, comes
from the notion of nondeterministic computation (NP stands for
\emph{nondeterministic polynomial time}).  In the classical theory, a
nondeterministic Turing Machine computes in the same manner as a
standard Turing Machine, except there are multiple possible next
configurations associated with each pair of state and tape symbol.  A
nondeterministic TM accepts its input \todo{talk about decision
  machines and accepting/rejecting in the background section?} if
there is \emph{any} valid path of configurations that accepts the
input.  Somewhat surprisingly, the functions which can be computed in
polynomial time by nondeterministic Turing Machines turn out to be
precisely the same functions that can be efficiently verified via
witness strings.  To our knowledge, there has been little work done to
investigate whether a similar relationship holds in the generalized
theory we have been investigating.

Here we present a model of nondeterministic computation that naturally
generalizes from the Turing Machine model to the Blum model of
computation.  The basic idea is that each branch node is allowed to
have multiple \textbf{Yes} and \textbf{No} nodes, and an accepting
computation is any computational path through the machine that
respects the branching functions.  In keeping with the literature,
when we wish to refer to the machines discussed in the previous
section, we use the expression ``BSS Machine''.

\section{Nondeterministic Computation}

\begin{definition} \textbf{NDET-Machine over R}

  An NDET-Machine over a ring R is a finite, directed, connected graph
  with three types of nodes: \emph{computation}, \emph{shift} and
  \emph{branch}, along with distinguished \emph{input}, \emph{accept}, 
  and \emph{reject} nodes.\\

  Associated with each computation node, $\eta$, is a next node,
  $\beta_{\eta}$, and a polynomial (rational for fields) computation
  map, \functype{g_\eta}{R^\infty}{R^\infty}.

  Associated with each shift node, $\eta$ is a next node $\beta_\eta$
  and a shift map, $\sigma_l$ or $\sigma_r$ : $R_\infty \rightarrow
  R_\infty$.  $\sigma_l$ and $\sigma_r$ are defined in the same manner
  here as they are in the standard machine case.

  Associated with each branch node, $\eta$, is a nonzero polynomial (rational)
  function \functype{g_\eta}{R^\infty}{R}, a set $\set{\beta^+_1,
    \beta^+_2, \ldots \beta^+_i}$ of \emph{Yes Nodes} and a set
  $\set{\beta^-_1, \beta^-_2, \ldots \beta^-_j}$ of \emph{No Nodes}.
  We refer to the max of $i$ and $j$ as the \textbf{degree} of $\eta$.
\end{definition}

The computation of an NDET-machine is defined analogously to that of a
Blum machine, with the caveat that we now accept any input for which
there exists some accepting path through the machine.  More formally,
for each configuration pair $z = (\eta, x)$, we define the
\textbf{next configuration set}, $H(z)$, to be the set of all pairs,
$\set{(\eta', x')}$, such that:

\begin{enumerate}
\item If $\eta$ is a computation node, then $\eta' = \beta_\eta$, and
  $x' = g(x)$.
\item If $\eta$ is a shift node, then $\eta' = \beta_\eta$, and
  $x' = sigma(x)$.
\item If $\eta$ is a branch node, $x' = x$ and if $g(x) > 0)$, $\eta'$
  is a yes-node of $\eta$.  If $g(x) < 0$, $\eta'$ is a no-node of
  $\eta$.
\end{enumerate}

Given a machine $M$ and input $x$, we say that the sequence $z_1 =
(\eta_1, x_1), z_2 = (\eta_2, x_2), \ldots z_k = (\eta_k, x_k)$ of
configurations of $M$ is \textbf{valid} if $x_1 = x$ and, for all $2
\leq i \leq k$, $x_i \in H(x_{i-1})$.  Such a sequence is said to be
\textbf{accepting} if, additionally, $x_k$ is the accept node of $M$.
It is said to be \textbf{rejecting} if $x_k$ is the reject node of
$M$.  It is said to be \textbf{halting} if it is either accepting or
rejecting.

We denote the (possibly infinite) set of all valid paths for a machine
$M$ on input $x$ by $\allpaths(M, x)$, the set of all accepting paths
for $M$ on $x$ by $\accpaths(M, x)$, and the set of all rejecting paths 
for $M$ on $x$ by $\rejpaths(M, x)$.

\begin{definition} \textbf{NDET-Machine Halting, Acceptance and Rejection}

  An NDET-Machine is said to \textbf{halt} the input $x$ if
  $\allpaths(M, x)$ is finite.  It \textbf{accepts} $x$ if it halts
  and $\accpaths(M, x)$ is nonempty.  It \textbf{rejects} if it halts
  and does not accept.
  
\end{definition}

\section{Computation Cost and the Class NDETP}

\begin{definition} \textbf{Nondeterministic Computation Cost}

  Let $\gamma = z_1, z_2, \ldots, z_k$ be a halting path for an
  NDET-Machine $M$ on input $x$.  The \textbf{cost} of $\gamma$ is
  given by:

  $$cost(\gamma) = k*ht_{max}(x)$$ where $ht_{max}$ is defined as in 
  \textbf{(2.6.4)}.

  Let $M$ be an NDET-Machine over a ring $R$ with height function
  $ht_R$, and let $M$ halt on input $x$.  The \textbf{cost} of the
  computation of $M$ on $x$ is given by:
  
  $$\max_{\gamma \in \allpaths}(cost(\gamma))$$

\end{definition}

\begin{definition} $\ndetp$
  A language, $S$ is in $\ndetp$ if there exists a machine $M$ and a
  constant $p$ such that, $\forall w \in S$, $M$ accepts w with cost
  $O(size(w)^p)$.
\end{definition}

\section{Results and Open Questions Concerning $\ndetp$}

\proposition{For any ring $R$ and any height function, $\p \subseteq
  \ndetp$} 

\proof{Every standard BSS Machine that acts as a decider
  can be trivially converted into an NDET Machine in which every
  branch node has degree 1. Such a machine has only one valid path of
  any given length, and thus has runtime cost exactly equal to the
  cost of the original machine for all inputs.  Thus every problem in
  $\p$ is also in $\ndetp$.}

\begin{itemize}
\item \textbf{Question:} Under what conditions is it the case that $\ndetp = \np$? 
  
  \textbf{Status:} Known for finite rings and $\integers$ with both
  bit and unit costs.  Unclear for $\rationals$ and uncountably
  infinite rings.
  
  \proposition{Let $R$ be a finite ring or field with $N$
    elements. $\ndetp = \np$ with respect to unit cost.}

  \proof{We first show that $\np \subseteq \ndetp$.
    
    Let $S \in \np$.  Then there exists a machine $M$ such that for
    every $x$ in $S$ there exists a $w$ such that $\computefn_M(x, w)
    = 1$ and $cost_M(x, w) \leq p(size(x))$, where p is some
    polynomial function.  It immediately follows that $w$ must have
    size less than $p(size(x))$, or else computing a single step with
    $w$ in a register of $\statespace_M$ would render the cost of
    computing a single step to be greater than the given bound on the
    runtime of $M$.  Thus we can construct an NDET-Machine which, on
    input $x$, nondeterministically generates all possible witnesses
    of length at most $p(size(x))$, and then for each possible witness
    $w$ simulates $M$ on input $w$.

    \todo{Figure out how to insert diagram of nondeterministic witness
      generation submachine.}

    \todo{Backwards Direction Details: Idea is that given an ndet
      machine, we should be able to construct an NP Machine that uses
      the state projection of an accepting path as a witness.}  
  }
  
  \proposition{Let $R$ = $\integers$ with bit cost.  $\ndetp = \np$.}
  \proof{\todo{See notes}}

  \proposition{Let $R$ = $\integers$ with unit cost.  $\ndetp \subset
    \np$}

  \proof{The containment construction still goes through, but we have
    that $HN_{\integers} \in \np$, but $HN_{\integers}$ is
    undecidable by standard machines.  Relies on the next
    proposition.}

  \proposition{For all $R$, $\ndetp \in \mathbb{EXPTIME}$}

  \proof{We can simulate an NDET-Machine with a standard BSS machine
    with an exponential blowup.  \todo{Fill in details}}

\item If it is not the case that $\ndetp = \np$ for some underlying
  rings, what is their relationship?  For example, is it always the
  case that $\ndetp \subseteq \np$?
\item Are there conditions for which it is the case that $\ndetp = \p$?
\item Under what conditions do we have the existence of complete
  problems for $\ndetp$?
\end{itemize}

\note{The classical theory answers all of these questions for the
  special case R = $\integers_n$ with unit cost and R = $\integers$
  with bit cost.  We are interested in investigating how the classical
  results generalize to the algebraic case.  An important
  consideration in this work will be attending to elements of the
  classical theory which depend either on the representation of inputs
  (since the Blum model says nothing about representation), on the
  countability/enumerability of the set of possible inputs to a Turing
  Machine, or on the finite number of possible elements of a given
  size in the classical model.  All of these assumptions may be
  violated in the more generalized theory.}

