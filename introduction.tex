\chapter{Introduction}

Classical Theory of Computation provides a framework in which we can
pose mathematically rigorous questions about the possibility and cost
of solving problems via algorithmic procedures.  Historically central
to this theory has been the Turing Machine model of computation, in
which the memory of of an ideal computing machine is represented by an
``infinitely long'' tape containing finitely many symbols at any given
point in time.  One shortcoming of this computing model is that it
does not allow us to describe computational problems that are defined
over uncountable sets such $\mathbb{R}$ or $\mathbb{C}$.  Examples of
such problems include analyzing the expected runtime of Newton's
Method, determining membership in the Mandelbrot Set, Linear
Programming with real coefficients, and the problem of determining
whether a finite set of polynomials in $n$ variables share a common
root.\\

To address the above issues and others, in \cite{B89} Blum et
al. develop an algebraic theory of computation that generalizes the
Turing Machine model to computation over arbitrary rings and fields.
In \cite{B98} the same authors present a wealth of results related to
their model, which we refer to as the BSS Machine. In this thesis, we
present a basic development of their model as well as a sample of
their major results.  In particular, we show how Blum et al. define
the relativized complexity classes, $\p$ and $\np$, which generalize
the well-known classes $\mathbf{P}$ and $\mathbf{NP}$ to machines that
compute with values from an arbitrary ring $R$.  We then develop a
third complexity class, $\ndetp$, which we present as an alternative
generalization of the classical $\mathbf{NP}$.  Our major results are
that, over any ring $R$, $\p \subseteq \ndetp \subseteq \np$ and
$\ndetp \subseteq \exptime$. From these containments it follows that
if $\p \subset \ndetp$ or $\ndetp \subset \np$, then $\p \neq \np$.
We also demonstrate the existence of at least one ring over which
$\ndetp \subset \np$.  (It is already known that $\p \neq \np$ for
this case.)  We conclude by proving that that $R$ being finite is a
sufficient condition for the equality $\ndetp = \np$ to hold, and we
conjecture that this condition is also necessary.

%%% Local Variables: 
%%% mode: latex
%%% TeX-master: "main"
%%% End: 