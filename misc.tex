\chapter{Notes and Miscellanea}

\section{Additional Work Pursued this Term}

In the course of investigating some of the open problems outlined
above, we have encountered results/question which are edifying in
their own right and, as far as we can tell, previously unpublished.

\subsection*{Decidability and Recognizability of $\integers$ and $\rationals$ over $\reals$}

\begin{theorem}{$\integers$ is decidable over the ordered $\reals$.
    However, for any machine $M$ that decides $\integers$ and any time
    bound, $T$, there exists an integer $n$ such that $M$ does not
    decide $n$ in time $T$.  Thus, while $\integers$ is decidable, it
    is not decidable in any bounded amount of time.}
\end{theorem}

\begin{proof}
  The decidability of $\integers$ can be shown constructively.  Given
  an input $n$, we can initialize two cells in our state space to 0,
  and then repeatedly add 1 to the first cell and subtract 1 from the
  second cell.  After each round of addition and subtraction, we
  perform an equality and an order comparison against our input.  If
  we ever have equality, output 1.  If the order comparison ever fails
  (indicating that we have checked all integers with absolute value
  less than our input) output 0.

  The unboundedness result follows from the path decomposition
  theorem, which states that the time-T halting sets of any machine
  consist of finite unions of finitely many distinct connected
  components of $R^\infty$.  Since $\integers$ has infinitely many
  distinct connected components, it follows that no finite number of
  time-t halting sets contains all of $\integers$.
\end{proof}

\note{The above construction can also be used to show that $\integers$
  is recognizable over $\reals$ without order comparisons.}

\note{One salient feature of the above construction is that is takes
  advantage of the fact that $\integers$ can be expressed as the set
  of elements generated by the element 1 under addition and
  subtraction.  We can generalize this feature of the construction to
  construct a recognizer for any language which can be generated by
  iterates of a finite set of computable functions on a finite set of
  inputs.} 

\subsection*{Recognizability of $\rationals$ over R}

\begin{theorem}{$\rationals$ is recogizable over $R$}
\end{theorem}
\begin{proof}
  We can enumerate values in the rationals using a Cantor
  diagonalization scheme by maintaining counters for integer values
  $p$ and $q$ to represent the rational $\frac{p}{q}$. We can then
  simply iterate through the enumeration and check for equality
  against our input, outputting 1 if the equality check ever succeeds.
\end{proof}

\subsubsection*{Open Questions}

\begin{enumerate}
\item Is $\integers$ decidable over the unordered reals?
\item Is $\rationals$ deciable over the reals, ordered or unordered?
\item Can every decidable (recognizable) set be described in terms of
  iterates of a finite set of computable functions over a finite set
  of inputs?
\item Are there efficient $\epsilon$-approximation methods for $\integers$
  or $\rationals$ over $\reals$?

\end{enumerate}