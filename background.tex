\chapter{Introduction}

Classical Theory of Computation provides a framework in which we can
pose mathematically rigorous questions about the possibility and cost
of solving problems via algorithmic procedures.  Historically central
to this theory has been the Turing Machine model of computation, in
which the memory of of an ideal computing machine is represented by an
``infinitely long'' tape containing finitely many symbols at any given
point in time.  One shortcoming of this computing model is that it
does not allow us to describe computational problems that are defined
over uncountable sets such $\mathbb{R}$ or $\mathbb{C}$.  Examples of
such problems include analyzing the expected runtime of Newton's
Method, determining membership in the Mandelbrot Set, Linear
Programming with real coefficients, and the problem of determining
whether a finite set of polynomials in $n$ variables share a common
root.\\

To address the above issues and others, in \cite{B89} Blum et
al. develop an algebraic theory of computation that generalizes the
Turing Machine model to computation over arbitrary rings and fields.
In \cite{B98} the same authors present a wealth of results related to
their model, which we refer to as the BSS Machine. In this thesis, we
present a basic development of their model as well as a sample of
their major results.  In particular, we show how Blum et al. define
the relativized complexity classes, $\p$ and $\np$, which generalize
the well-known classes $\mathbf{P}$ and $\mathbf{NP}$ to machines that
compute with values from an arbitrary ring $R$.  We then develop a
third complexity class, $\ndetp$, which we present as an alternative
generalization of the classical $\mathbf{NP}$.  Our major results are
that, over any ring $R$, $\p \subseteq \ndetp \subseteq \np$ and
$\ndetp \subseteq \exptime$. From these containments it follows that
if $\p \subset \ndetp$ or $\ndetp \subset \np$, then $\p \neq \np$.
We also demonstrate the existence of at least one ring over which
$\ndetp \subset \np$.  (It is already known that $\p \neq \np$ for
this case.)  We conclude by proving a sufficient condition under which
the equality $\ndetp = \np$ holds, and we conjecture that this
condition is also necessary.

\chapter{Background}

In this chapter, we present a basic development of an algebraic model
of computation due to Blum et al.  The material presented here draws
heavily from \cite{B89} and \cite{B98}.  Since the computing machines
developed by Blum et al. are essentially a generalization of the
classical Turing Machine, we begin with a brief review of Turing
Machines.  For a more thorough introductory reference see \cite{S06}.
For Turing's original article introducing the concept of a computing
machine, see \cite{T36}.

\section{Classical Turing Machines}

A Turing Machine can be thought of as an infinitely long 1-way tape
divided into cells, on each of which can be written a single symbol.
Such a machine is outfitted with a tape head which can move left and
right as well as read and write symbols to and from the tape. A
machine's ``program'' consists of a set of states along with a
transition function that provides instructions for how the tape head
should behave given a symbol and a current machine state.  These ideas
are made precise as follows:

\begin{definition}{Turing Machine}

  A Turing Machine is a 7-tuple, $(Q, \Sigma, \Gamma, \delta, q_0, q_{accept}, q_{reject})$, where:

  \begin{itemize}
  \item $Q$ is a set of machine states.
  \item $\Sigma$ is an alphabet of input characters, not containing
    the blank symbol, $\blank$.
  \item $\Gamma$ is the tape alphabet, where $\blank \in \Gamma$, and
    $\Sigma \subset \Gamma$.
  \item \functype{\delta}{Q \times \Sigma}{Q \times \Gamma \times
      \set{\leftarrow, \rightarrow}} is a transition function.
  \item $q_0$ is the initial state of the machine.
  \item $q_{accept}$ is the machine's accept state.
  \item $q_{reject}$ is the machine's reject state.
  \end{itemize}

  The transition function $\delta$ determines how the machine passes
  between \textbf{configurations}, which are triples in $\Sigma^*
  \times Q \times \naturals$ encoding the symbols currently written on
  the tape, the machine's current state, and the position of the tape
  head.  At each computation step, the machine reads the symbol
  $\sigma$ on the cell currently occupied by the tape head.  If the
  machine is in state $q$, it then calculates $\delta(q, \sigma)$ =
  $(q', \sigma', d \in \set{\leftarrow, \rightarrow})$, transitions
  into state $q'$, writes $\sigma'$ to the current cell, and
  increments or decrements the location of the tape head depending on
  the value of $d$.  In the case that the machine head would move left
  when on the leftmost cell, it stays in place.  Computation ends when
  a machine transitions into either $q_{accept}$ or $q_{reject}$, at
  which point the machine is said to either \textbf{accept} or
  \textbf{reject} its input, respectively.  A Turing Machine is said
  to \textbf{halt} on input $x$ if it either accepts or rejects $x$.
  Upon halting, the \textbf{output} of a machine is the string left
  behind on the machine's tape.
  
\end{definition}

\begin{example}
  The following Turing Machine reads in a binary string $s$ and
  outputs the string given by replacing each $0$ in $s$ with a $1$ and
  each $1$ with a $0$.  It accepts on all inputs.

  \begin{itemize}
  \item $Q = \set{q_0, q_{accept}, q_{reject}}$
  \item $\Gamma = \set{\blank, 0, 1}$
  \item $\Sigma = \set{0,1}$
  \item $\delta(q_0, 0) = (q_0, 1, \rightarrow)$
  \item $\delta(q_0, 1) = (q_0, 0, \rightarrow)$
  \item $\delta(q_0, \blank) = (q_{accept}, \blank, \leftarrow)$
  \end{itemize}

  Figure \ref{fig:turing-example} illustrates the computation of our
  example machine on the input $1001$. In each computation step but
  the last, the machine flips the parity of the symbol beneath the
  tape, then moves to the right and ``transitions'' from $q_0$ to
  $q_o$.  When the machine reads the blank symbol $\blank$, it leaves
  the same symbol on the tape, transitions to the accept state and
  moves the tape head to the left before exiting with output $0110$.

  \begin{figure}[p]
    \begin{center}
      \tape[1]{$q_0$}{{1,0,0,1,\blank,$\cdots$}} \vspace{1mm}
      \tape[2]{$q_0$}{{0,0,0,1,\blank,$\cdots$}} \vspace{1mm}
      \tape[3]{$q_0$}{{0,1,0,1,\blank,$\cdots$}} \vspace{1mm}
      \tape[4]{$q_0$}{{0,1,1,1,\blank,$\cdots$}} \vspace{1mm}
      \tape[5]{$q_0$}{{0,1,1,0,\blank,$\cdots$}} \vspace{1mm}
      \tape[4]{$q_{accept}$}{{0,1,1,0,\blank, $\cdots$}} \vspace{1mm}
    \end{center}
    \caption{An Example Turing Machine Computation}
    \label{fig:turing-example}
  \end{figure}
\end{example}

\section{Finite Dimensional Machines}

A BSS Machine is in essence a Turing Machine with several key features
abstracted to a more general setting. We replace the classical finite
tape alphabet with elements of a ring/field, and we replace the
classical operations of reading and writing to individual cells with
order/equality comparisons and computations of componentwise
polynomial/rational maps.
 
To build intuition about how Blum machines operate and how we can
analyze their properties, we first present Blum et al's definition of
a \emph{Finite Dimensional Machine}, which corresponds in the
classical theory to a Turing Machine with a tape of finite length.
Having built some intuition with the finite dimensional model, we then
show how we can generalize to the infinite dimensional case, allowing
us to consider machines that can process arbitrarily large inputs.

In what follows, let $R$ be a commutative ring with unity, and let $l,
m$, and $n$ be positive integers.

\begin{definition}{\textbf{Finite Dimensional Machine} over a ring
    \textbf{R}}
  
  A \textbf{Finite Dimensional Machine} (FDM) $M$ over $R$ is a finite
  directed graph with node set $\nodes$.  The collection $\nodes$
  contains three distinguished nodes: the \textbf{input node},
  $\eta_1$; the \textbf{accept node}, $\eta_{accept}$; and the
  \textbf{reject node}, $\eta_{reject}$. The input node has indegree 0;
  the accept and reject nodes have outdegree 0.  The remaining
  elements of $\nodes$ are divided into two types: \emph{computation}
  and \emph{branch}.  Computation nodes have outdegree 1; branch nodes
  have outdegree 2.  We use the expression ``next node(s)'' of a node
  $\eta$ to refer to the node(s) to which $\eta$ has outgoing
  edges.\footnote{The definition used here departs slightly from that
    used by Blum et al. in \cite{B98} and \cite{B89}.  The main
    difference is our replacement of a single output node with the
    pair of accept and reject nodes, which reflects on decision
    problems instead of computable functions.}\\
  
  An FDM has three associated spaces: an input space $\inspace_M
  \subseteq R^n$, a state space $\statespace_M \subseteq R^m$, and an
  output space $\outspace_M \subseteq R^l$.  (Recall that $R^k$
  denotes the set of $k$-tuples containing elements in $R$).\\

  Each node has an associated map and a set of next nodes, which vary
  in structure in accordance with the node's type.

  \begin{enumerate}
  \item Associated with the input node is a linear map
    \functype{I}{\inspace_M}{\statespace_M}. We denote the input node
    by $\eta_1$, and we denote the unique next node of $\eta_1$ by
    $\beta_{\eta_1}$ or just $\beta_1$.
    
  \item Associated with each computation node $\eta$ is a
    componentwise polynomial map
    \functype{g_\eta}{\statespace_M}{\statespace_M}. We denote the
    unique next node of $\eta$ by $\beta_\eta$.
    
  \item Associated with each branch node $\eta$ is a nonzero
    polynomial function \functype{h_\eta}{\statespace_M}{R}.  We
    associate the condition $h_\eta(x) \geq 0$ with one of the next
    nodes of $\eta$, called the \textbf{Yes Node} of $\eta$.  We
    symbolically denote the Yes Node by $\beta_\eta^+$. We associate
    the condition $h_\eta(p) < 0$ with the other next node of $b$,
    which we call the \textbf{No Node} and denote $\beta_\eta^-$.
    
  \item Associated with the accept and reject nodes is a linear map
    \functype{O}{\statespace_M}{\outspace_M}.
  \end{enumerate}
\end{definition}

\textbf{Note: }The above definition is subject to modification
depending on whether the ring $R$ is a field and whether it is endowed
with a natural order.

In the case that $R$ is a field, we may want to allow our machines to
take advantage of the natural division operation by allowing rational
maps rather than polynomial maps for $g_\eta$ and $h_\eta$.  In such a
case, care needs to be taken to ensure that division is always
well-defined by preventing divisions by $0$.  We do so by adopting the
convention that each node that computes a rational function is
preceded by a subroutine that computes all relevant denominators,
entering an infinite loop if any of them are zeros.  This allows us to
assume that computations are always well-defined.

When $R$ is unordered, when $R = \complexes$ for example, we replace
the branching conditions $h(x) > 0$ and $h(x) < 0$ with $h(x) = 0$ and
$h(x) \neq 0$, respectively.\\

\textbf{Example: } Figures \ref{fig:newton-high} and
\ref{fig:newton-low} show how one might implement an FDM to carry out
Newton's Method for determining approximate zeros of a polynomial.
Figure \ref{fig:newton-high} gives a ``high-level'' view of what each
node does logically, while the latter demonstrates the underlying
implementation.  Because we are currently considering only finite
dimensional machines, we restrict ourselves to polynomials of degree
5.  We will show later how we can modify our definitions to allow for
uniform machines that can process inputs of arbitrary size.

\begin{figure}[p]
  \centering

  \begin{tikzpicture}[node distance=4cm]

    \node (init)   [input, draw] {Input: $p \in \reals[x]$, $z \in \reals$, $\epsilon \in \reals$};
    \node (comp)   [compute, draw, below of=init]   {$z = z - \frac{p(z)}{p'(z)}$};
    \node (check)  [branch, draw, below of=comp]    {$|p(x)| - \epsilon < 0$};
    \node (accept) [accept, draw, below of=check]   {Output $z$};
    
    \draw [->, thick] (init) --                                      (comp);
    \draw [->, thick] (comp) --                                      (check);
    \draw [->, thick] (check) to [out=0, in=0] node[auto, swap] {No} (comp);
    \draw [->, thick] (check) to node[auto] {Yes}                    (accept);

  \end{tikzpicture}

  \caption{Newton Machine - High Level}
  \label{fig:newton-high}
\end{figure}

\begin{figure}[p]
  \centering
  
  \begin{tikzpicture}[node distance=4cm, label distance=.3cm]

    \node(init) [input, draw, label=left:\Large $\eta_1$] 
    {Input: $x = (x_1, x_2, x_3, x_4, x_5, x_6, x_7, x_8)$;};
    \node(comp) [compute, draw, label=left:\Large $\eta_2$, below of=init] {
      $g(x) = (x_1 - 
      \frac{x_3x_1^5 + x_4x_1^4+ x_5x_1^3 + x_6x_1^2 + x_7x_1 + x_8}
      {5x_3x_1^4 + 4x_4x_1^3 + 3x_5x_1^2 + 2x_6x_1 + x_7}, 
      x_2, x_3, x_4, x_5, x_6, x_7, x_8)$};
    \node(check)[branch, draw, label=left:\Large $\eta_3$, below of=comp] {$h(x) = x_1^2 - x_2^2$};
    \node(accept)[accept, draw, label=left:\Large $\accept$,below of=check] {$O(x) = x_1$};
    \node(reject)[reject, draw, label=above:\Large $\reject$,left of=accept,
                  node distance= 6cm] 
    {$O(x) = x_1$};

    \draw [->, thick] (init) --                                             (comp);
    \draw [->, thick] (comp) --                                            (check);
    \draw [->, thick] (check) to [out=0, in=0] node[auto, swap] {$h(x) > 0$}(comp);
    \draw [->, thick] (check) to node[auto] {$h(x) < 0$}                  (accept);

    \node (legend) [rectangle, draw, fill=black!10, right of=accept, node distance=6cm] {
      \makecell[l]{
        $\nodes = \set{\eta_1, \eta_2, \eta_3, \accept, \reject}$\\
        $\inspace_M = \statespace_M = \outspace_M = \reals^8$\\
        Computation Nodes: $\set{\eta_2}$\\
        Branch Nodes: $\set{\eta_3}$
      }
    };
    
  \end{tikzpicture}

  \caption{Newton Machine - Low Level}
  \label{fig:newton-low}
\end{figure}


\begin{definition}{\textbf{Configuration}}  

  A \textbf{configuration} of an FDM $M$ is a pair $(\eta, x) \in
  \nodes \times \statespace_M$. The set $\nodes \times \statespace_M$
  is called the \textbf{configuration space} of $M$.  A configuration
  is said to be \textbf{terminal} if $\eta \in \set{\accept,
    \reject}$, and \textbf{non-terminal} otherwise.

\end{definition}

For each non-terminal configuration $z = (\eta, x)$, a machine $M$
defines a unique next configuration $z' = (\eta', x')$ as follows:

\begin{itemize}
\item If $\eta$ is the input node, $z' = (\beta_{1}, x)$.
\item If $\eta$ is a computation node, $z' = (\beta_{\eta}, g_\eta(x))$.
\item If $\eta$ is a branch node, $z' = \twopartdef{(\beta_\eta^+,
    x)}{h_\eta(x) \geq 0}{(\beta_\eta^-, x)}{h_\eta(x) < 0}$.
\end{itemize}

It will occasionally be useful to us to have a compact notation for
writing ``$z'$ is the next configuration of $z$.''  We do so using the
expression $z \yields z'$, which we read as $z$ \textbf{yields} $z'$.\\

We are now in a position to define what it means for a machine $M$ to
perform a computation on some input, $x$.  The basic idea is that a
computation is a path through $\nodes \times \statespace$ that begins
with $(\eta_1, I_M(x))$ and proceeeds by following next
configurations, halting upon reaching either the accept or reject node
of $M$.  The ``run-time'' of a computation is given by the length of
this path, and the output value of a computation is the result of
$O_m$ called on whatever is left in $\statespace_M$ when a terminal
configuration is reached.\\

\begin{definition}{\textbf{Halting Computation and time-t Halting Set}}

  A \textbf{halting computation} performed by an FDM $M$ on input $x$
  is a sequence of configurations: 
  $$\compute_M(x) = (z_0, z_1, \ldots, z_t)$$ such that
  
  \centerline{$z_0 = (\eta_1, I(x)$, $z_t \in \set{(\accept, x_t),
      (\reject, x_t)}$, and $z_i \yields z_{i+1}$ for $0 \leq i < k$}
  
  \vspace{\baselineskip}
  
  The \textbf{output} $\computefn_M(x)$ of a halting computation is
  the value given by $O(x_t)$.  If a halting computation of length $t$
  exists for an input $x$, we say that $x$ is in the \textbf{time-t
    Halting Set} of M and we write $x \in \halting_t$.  \\

  The \textbf{Halting Set} $\halting_M$ of a machine $M$ is the union
  over all $t > 0$ of the time-t halting sets for $M$, i.e., the set of
  values for which $M$ eventually halts.
\end{definition}

\note{If a halting computation exists for a machine $M$ on input $x$,
  it is unique.  However, not every input defines a halting
  computation.  It is easy enough, for example, to construct machines
  which will enter infinite loops without ever reaching the output
  node.  In such a case, we define the computation of a machine $M$ to
  be the infinite sequence $$z_0, z_1, z_2, \ldots, $$ such that   

  \centerline{$z_0 = (\eta_1, I(x))$, and $z_i =  z_{i-1}'$ for $0 < i$.}
}

\vspace{\baselineskip}

Intuitively, we can think about the computation of an FDM, $M$ as the
sequence defined by ``following the flow'' of the node graph while
applying the appropriate transformations to $\statespace_M$ at
computation nodes. Given an input $x$, $M$ starts in configuration
$\eta_1, I(x)$. At each computation node $\eta$, it computes
$g_\eta(x)$, where $x$ is the current value of stored in the machine's
state space.  $g_\eta(x)$ then becomes the new value of the state
space, and the current node becomes $\beta_{\eta}$.  At each branch
node $\eta$, $M$ computes $h_\eta(x)$, where again $x$ is the
currently stored value.  If $h_\eta(x) \geq 0$, the new node is
$\beta_\eta+$.  Otherwise it is $\beta_\eta^-$.  The machine halts
when it eventually reaches the output $\accept$ or $\reject$, at which
point it applies the function $O(x)$ to its stored value, which is
taken as the machine's output.\\

\textbf{Example: } Figure \ref{fig:q-recog} shows a machine over
$\reals$ which, on input $x \in \rationals^+ $ returns a pair of
integers, $(p,q)$ such that $x = \frac{p}{q}$.  If $x$ is not a
positive rational, the machine does not halt.

\begin{figure}[p]
  \centering
  \begin{tikzpicture}[node distance=4.8cm]

    \node(init) [input, draw, label=left:\Large $\eta_1$] 
    {$I(x) = (x, 0, 1)$};
    \node(test) [branch, draw, below of=init, label=45:\Large $\eta_2$] 
    {$h(x) = x_1 - \frac{x_2}{x_3}$};
    \node(flip) [compute, draw, left of=test, label=left:\Large $\eta_3$,]
    {$g(x)= (x_1, 1, x_2+1)$};
    \node(fliptest) [branch, draw, below of=flip, label=left:\Large $\eta_4$] 
    {$h(x) = x_3$};
    \node(inc)  [compute, draw, below of=fliptest, label=left:\Large $\eta_5$,] 
    {$g(x) = (x_1, x_2+1, x_3-1)$};
    \node(accept)   [accept, draw, below of=test, right of=test, label=left:\Large $\accept$] 
    {$O(x) = (x_2, x_3)$};
    \node(reject)   [reject, draw, below of=accept, label=left:\Large $\reject$, node distance=2.4cm] 
    {$O(x) = (x_2, x_3)$};

    \draw[->, thick] (init) -- (test);
    \draw[->, thick] (flip) -- (test);
    \draw[->, thick] (test) to [out=270, in=0] node [auto, swap] {$h(x) \neq 0$} (inc);
    \draw[->, thick] (test) to [out=0, in=90] node[auto] {$h(x)=0$} (accept);
    \draw[->, thick] (fliptest) to [out=0, in= 225] node[auto] {$h(x) = 0$} (test);
    \draw[->, thick] (fliptest) to [out=90, in= 270] node[auto] {$h(x) \neq 0$}(flip);
    \draw[->, thick] (inc) -- (fliptest);

    \node (legend) [rectangle, draw, fill=black!10, below of=reject, node distance=3.0cm] {
      \makecell[l]{
        $\nodes = \set{\eta_1, \eta_2, \eta_3, \eta_4, \eta_5, \accept, \reject}$\\
        $\inspace_M = \reals, \statespace_M = \reals^3, \outspace_M = \reals^2$\\
        Computation Nodes: $\set{\eta_3, \eta_5}$\\
        Branch Nodes: $\set{\eta_2, \eta_4}$\\
        $\halting_M = \rationals^+$\\
        $\computefn_M(x) = (p,q) \mid x = \frac{p}{q}, \text{ gcd}(p,q) = \, 1$
      }
    };

  \end{tikzpicture}

  \caption{Recognizer for $\rationals$ over $\reals$}
  \label{fig:q-recog}
\end{figure}

\begin{definition}{\textbf{Computation Path} and \textbf{State Trajectory}}\\
  
  The \textbf{Computation Path} $\computepath_M(x)$ of $M$ on
  input $x$ is the projection of $\compute_M(x)$ under the map:
  $$\mfunctype{\pathprojection}{\fullstate}{\nodes} \mid \pathprojection(\eta, x) = \eta$$
  
  The \textbf{State Trajectory} of $M$ on input $x$ is the projection
  of $\compute_M(x)$ under the map:
  $$\mfunctype{\stateprojection}{\fullstate}{\nodes} \mid \stateprojection(\eta, x) = x$$
\end{definition}

\note{The computation path and state trajectory of a machine encode,
  respectively, the sequence of nodes a machine traverses during a
  computation, and the sequence of values stored in the machine's
  state space during a computation.}\\

\textbf{Example: } Let $M$ be the machine described in Figure
\ref{fig:q-recog}. For the input $x = 2$, we have:
\begin{eqnarray*}
  \phi_M(x) &=& (\eta_1, (2,0,1)), (\eta_2, (2,0,1)), (\eta_5, (2,0,1)), (\eta_4, (2,1,0)), (\eta_3, (2,1,0)),\\
            &&  (\eta_2, (2,1,2)), (\eta_5, (2,1,2)), (\eta_4, (2,2,1)), (\eta_2, (2,2,1)), (\accept, (2,2,1))\\
            && \\
  \computepath_M(x) &=& \eta_1,\, \eta_2,\, \eta_5,\, \eta_4,\, \eta_3,\, \eta_2,\, \eta_5,\, \eta_4,\, \eta_2,\, \accept
\end{eqnarray*}

The State Trajectory of $M$ on input $x$ is given by:
$$(2,0,1), (2,0,1), (2,0,1), (2,1,0), (2,1,0), (2,1,2), (2,1,2), (2,2,1), (2,2,1), (2,2,1)$$

\begin{definition}{\textbf{Finite-Dimensionally Computable Function}}
  
  A function \functype{\varphi}{X}{R^m}, $X \subset R^n$ is
  \textbf{f.d. computable} over $R$ if there exists a
  finite-dimensional machine, $M$ over $R$ such that 
  $\varphi(x) = \compute_M(x) \mid \forall x \in X$.
  
\end{definition}

\begin{definition}{\textbf{Decidable, Recognizable, and Co-Recognizable Sets}}
  
  A set $S \subset R^n$ is \textbf{finite-dimensionally decidable} if
  there exists an FDM $M$ with $\inspace_M = R^n$ such that $M$
  accepts $x$ if and only if $s \in S$, and $M$ rejects $x$ if and
  only if $s \notin S$.\\

  A set $S \subset R^n$ is \textbf{finite-dimensionally recognizable}
  if there exists an FDM $M$ with $\inspace_M = R^n$ such that $M$
  accepts $x$ if and only if $x \in S$.\\

  A set $S \subset R^n$ is \textbf{finite-dimensionally
    co-recognizable} if there exists an FDM $M$ with $\inspace_M =
  R^n$ such that $M$ accepts $x$ if and only if $x \in S$.
\end{definition}
  
\textbf{Note:} Here we again depart slightly from the development in
\cite{B89} and \cite{B98}.  Blum et al. do not explicitly appeal to
notions of acceptance and rejection.  Instead they consider machines
that output $1$ on some inputs and $0$ on other inputs.  For
completeness, we also present the BSS characterization of
decidability.  It is straightforward to verify that classes of
decidable, recognizable, and co-recognizable languages are the same
under either set of definitions.

\begin{definition}{\textbf{Decidable, Recognizable, and Co-Recognizable Sets (BSS)}}

  A set $S \subset R^n$ is \textbf{(f.d.) decidable} over $R$ if the
  function \functype{\chi_S}{R^n}{R} is (f.d) computable over $R$,
  where:
  
  $$\chi_S(x) =  \twopartdef{1}{x \in S}{0}{x \notin S}$$
  \vspace{\baselineskip}
  
  A set $S \subset R^n$ is \textbf{(f.d) recognizable} or
  \textbf{(f.d) semi-decidable} over $R$ if the function
  \functype{\chi_S}{R^n}{R} is (f.d.) computable over $R$, where:
  
  $$\chi_S(x) = \twopartdef{1}{x \in S}{0 \text{\emph{ or undefined}}}{x \notin S}$$
  \vspace{\baselineskip}
  
  A set $S \subset R^n$ is \textbf{(f.d.) co-recognizable} over $R$ if
  the function \functype{\chi_S}{R^n}{R} is (f.d.) computable over $R$,
  where:
  
  $$\chi_S(x) =  \twopartdef{1}{x \notin S}{0 \text{\emph{ or undefined}}}{x \in S}$$
  \vspace{\baselineskip}
  
\end{definition}

It is fairly straightforward to see that the recognizable sets over
$R$ are precisely the halting sets over R and the co-recognizable sets
are those whose complements are recognizable. Less obviously, the
decidable sets are precisely those that are both recognizable and
co-recognizable.

\begin{proposition}{A set $S \subset R^n$ is decidable if and only if it is
  both recognizable and co-recognizable.}
\end{proposition}
\begin{proofsketch}
  
  Given semidecision machines, $M$ and $M'$ for both $S$ and $R^n -
  S$, we can contruct a decision machine for $S$ by simulating $M$ and
  $M'$ in parallel, outputting 1 if $M$ halts and 0 if $M'$ halts.
  Since one of $M$ and $M'$ must halt eventually, we are guaranteed
  that our constructed machine will halt for all inputs.

  \note{The parallelization construction generalizes nicely to any
    integral number of machines. See \cite{B98} for concrete details.}

\end{proofsketch}

\subsection{Notes and Further Properties of Machines}

\begin{itemize}

\item The behavior of a machine $M$ is completely determined by the
  values of the coefficients in each map associated with the nodes,
  $\nodes_M = \set{1, 2, \ldots, N}$.  We refer to the collection of
  these coefficients as the \textbf{machine constants} of $M$,
  written: $\mathcal{C}_M = \set{c_{11}, c_{21}, \ldots, c_{Nk}
    \subset R}$.  It is relatively straightforward to provide an
  unambiguous coding of $\mathcal{C}_M$ that is linear in the size of
  $\nodes_M$.

\item Associated with a machine $M$ is its degree, $D_M$, the maximum
  of the degrees of all maps associated with $M$.
\end{itemize}

\section{Characterization of (f.d.) Decidable Sets}

Here we sketch some results that characterize the finite dimensionally
decidable sets over $R$.  Our first theorem shows that any computable
set is the projection of a countable union of basic semi/quasi
algebraic sets.  Before we proceed, we need a few more definitions.

  \begin{definition}{\textbf{Semi-algebraic} and \textbf{Quasi-algebraic Sets}}
    
    A set $S \subset R$ is \textbf{basic semi-algebraic} over an
    ordered ring $R$ if $S$ is the set of elements in $R$ satisfying
    a finite system of polynomial equalities and inequalities. A set
    is \textbf{semi-algebraic} over $R$ if it is a finite union of basic
    semi-algebraic sets.

    A set $S \subset R$ is \textbf{basic quasi-algebraic} over an
    unordered ring $R$ if $S$ is the set of elements in $R$ satisfying
    a finite system of polynomial equalities. A set is
    \textbf{quasi-algebraic} over $R$ if it is a finite union of basic
    quasi-algebraic sets.
    
  \end{definition}

  \textbf{Example: } The upper half of the unit disc in $\reals^2$ is
  a basic semi-algebraic set defined by solutions to the equations:
  \begin{eqnarray*}
    x^2 + y^2 < 0\\
    y > 0
  \end{eqnarray*}

  In $\reals^2$, the perimeter of a square centered at the origin with
  side length 1 is a quasi-algebraic set defined by solutions to the
  equations:

  \begin{eqnarray*}
    x = \frac{1}{2}\\
    x = -\frac{1}{2}\\
    y = \frac{1}{2}\\
    y = -\frac{1}{2}\\
  \end{eqnarray*}

  \begin{definition}{\textbf{Initial computation path} and \textbf{time-T halting path set}}

    Let $M$ be a machine over $R$, and let $x$ be an element of
    $\inspace_M$, and let $\computepath_M(x) = (\eta_0, \eta_1, \ldots
    \eta_k, \eta_{k+1}, \ldots)$ be the computation path of $M$ on
    $x$.
    
    The \textbf{initial computation path} on $\computepath_M(x)$ of
    length $k$ is given by: 
    $$\computepath_M^k(x) = (\eta_0, \eta_1, \ldots,\eta_k)$$.

    The \textbf{time-T path set}, $\haltpaths{T}$, of $M$ is given by:
    $$\haltpaths_M(T) = \set{\computepath_M^T(x) \mid T(x) \leq T, x \in \inspace_M}$$.

  \end{definition}

  Recall that $\computepath_M(x)$ encodes the sequence of nodes
  traversed by $M$ on input $x$, so $\computepath_M^k(x)$ encodes the
  first $k$ nodes traversed by $M$ on input $x$, and $\haltpaths_M(T)$
  is the set containing all possible initial paths of length less than
  or equal to $T$.  \\

  Given an initial path $\computepath$ of a machine $M$, we may be
  interested to know which values in $\inspace_M$ cause $M$ to
  initially traverse $\computepath$.  We call te set of such values
  the \textbf{coincidence set} of $\computepath$, which we now
  formally define.
  
  \begin{definition}{\textbf{Coincidence set}}

    Let $M$ be a machine over a ring $R$, and let $\computepath =
    \computepath_{x}(k)$ be the initial computation path of length $k$
    for some $x \in \inspace_M$. The \textbf{coincidence set}
    $\coincidence{\computepath}$ of $M$ with respect to $\computepath$
    is the set of elements in $\inspace_M$ whose initial computation
    path of length $k$ is $\gamma$. It is defined by:
    $$\coincidence{\computepath} = 
    \set{p' \in \inspace_M \mid \computepath_{x'}(k) = \computepath}$$
    
    \emph{\note{It is not hard to see that belonging to the same
        coincidence set of length $k$ defines an equivalence relation
        on members of $\inspace_M$.  Thus, for each $k > 0$, the
        coincidence sets of $M$ on paths of length $k$ partition
        $\inspace_M$ into disjoint equivalence classes.}}

  \end{definition}

  \begin{theorem}{\textbf{Path Decomposition Theorem}}
    For any machine $M$ over $R$, we have the following:

    \begin{enumerate}
    \item For any $T > 0$, $\halting_T$ is a disjoint union of basic
      semi-algebraic (quasi-algebraic in the unordered case) sets.

    \item $\halting_M$ is a countable union of disjoint basic
      semi-algebraic sets.

    \item For $\computepath \in \haltpaths{T}$,
      $\computefn_M|\coincidence{\computepath}$ is a polynomial map.
    \end{enumerate}
  \end{theorem}

  \begin{proofsketch}
    
    The main idea is that the coincidence sets of a given computation
    path form basic semi(quasi)-algebraic sets.  We can partition
    each time-halting set into finitely many coincidence sets, and
    there are countably many time-halting sets.

    \begin{itemize}
    \item \textbf{(1)} follows from the fact that there are finitely
      many computation paths of length T.  Each computation path
      corresponds to a sequence of branch nodes that corresponds to a
      sequence of polynomial equalities and inequalities, so the
      coincidence set of a computation path is basic
      semi(quasi)-algebraic. Since we can partition $\halting_T$ into
      coincidence sets, each of which corresponds to a basic
      semi(quasi)-algebraic, $\halting_T$ is a union of basic
      semi/quasi-algebraics.
    \item \textbf{(2)} follows from the fact that $\halting_M$ is just
      the union over all $T$ of $\halting_M$, and each $\halting_M$ is
      an equivalence class on $\inspace_M$.
    \item \textbf{(3)} follows from the fact that $\computepath$
      corresponds to exactly one sequence of computation nodes, so
      $\compute_M|\coincidence{\computepath}$ computes the composition of
      $g_{\eta}$ for $\eta \in \computepath$.
    \end{itemize}
  \end{proofsketch}
  
  \begin{corollary}
    Any computable set is a countable union of disjoint basic
    semi(quasi)-algebraic sets.
  \end{corollary}

  \section{Generalization of FDMs over R to the Infinite Dimensional Case}

  So far, the machines we have considered can only access a
  finite-length region of ``memory'', which has size bounded by the
  dimension of $\statespace_M$.  An important feature of the classical
  model, however, is that a machine can store an arbitrarily large
  (though still finite) amount of information on its tape.  This
  feature is necessary because it allows for the definition of single
  machines that can accept inputs of arbitrary size. We incorporate
  this feature into our model by expanding a machine's natural state
  space to $R^\infty$ and adding new ``shift'' nodes, which move all
  values stored in the state space to the left or right.

  We denote by $R^\infty$ the disjoint union, $\bigcup_{n \geq 0} R^n$.\\

  We denote by $R_\infty$ the \emph{bi-infinite direct sum} space over
  $R$, which consists of elements of the form $(\ldots x_{-2}, x_{-1}, x_0,
  x_1, x_2, \ldots)$.
  
  \begin{definition}{\textbf{Machine over R}}

    A machine $M$ over a ring $R$ is a finite, directed, connected
    graph with five types of nodes: \emph{input, computation, branch,
      output} and \emph{shift}.  $\inspace_M = \outspace_M =
    R^\infty$, and $R_\infty = \statespace_M$ The first four node
    types are defined in the same manner as the finite dimensional
    case.  Shift nodes behave similarly to computation nodes, in that
    they have a unique next node and an associated map that transforms
    the state space of the machine by shifting all stored values one
    cell to the left or right. Shift nodes are allowed to have one of
    two maps: \emph{shift left} $\sigma_l$, and \emph{shift
      right} $\sigma_r$, where:\\

    $$\sigma_l(x)_i = x_{i+1}$$ and $$\sigma_l(x)_i = x_{i+1}$$
  \end{definition}
  
  \begin{theorem}{\textbf{Path Decomposition Theorem (Infinite Dimensional Case)}}
    
    For any machine M and positive integers n and T, we have the
    following:

    \begin{enumerate}
    \item $\halting_T^n = \bigcup_{\computepath \in \haltpaths{T}}
      \coincidence{\computepath}^n.$ That is, the n-dimensional
      component of the time-T halting set M of is a finite disjoint union
      of basic semi-algebraic sets.
    \item $\halting_M^n = \bigcup_{\computepath \in \haltpaths{}}
      \coincidence{\computepath}^n$.  That is, the n-dimensional
      component of the halting of M is a countable disjoint union of
      basic semi-algebraic sets.
      \end{enumerate}
  \end{theorem}

  \begin{proofsketch}
    
    Most of the proof for the infinite dimensional case follows the
    development for the finite case.  One cause for care is that the
    shift node maps are not polynomial when considered as maps from
    $R_\infty \rightarrow R_\infty$.  This difficulty is resolved by
    noting that the shift maps can be reconstructed as polynomial maps
    for any finite-dimensional subset, $R^n \subset R_\infty$, and by
    noting that in time T the machine can only modify a finite number
    of registers, since all the computation maps are injections of
    finite polynomial maps from $R^m \rightarrow R^n$ into the state
    space, $R_\infty \rightarrow R_\infty$, and we can traverse at
    most T shift nodes in time T.  Since we can bound the total number
    of registers accessible by a given machine in time T, the
    computation performed by a machine at any node is still
    polynomial. Thus for any particular size input and any particular
    halting time, there exists a finite dimensional machine that
    performs the same computations as M.
    
  \end{proofsketch}

  \section{Register Equations}

  One of the most important tools for investigating the behavior of
  BSS machines is the system of polynomial equations known as the
  \textbf{Register Equations}, which provide necessary and sufficient
  conditions for a machine $M$ to accept an input $w$ in time $t$.
  These follow directly from the definition of computation for a BSS
  machine. 

  \begin{definition}{\textbf{Register Equations}}
  \end{definition}
  
  \section{Decision Problems}

  The treatment of complexity questions in our model is much the same
  as in classical complexity theory.  We frame problems as languages,
  ie, as sequences of symbols drawn from some input domain.  In
  classical theory, this means binary encodings of strings -- for us,
  it means subsets of $R^\infty$.  The complexity/decidability of a
  problem then becomes a question about the complexity/decidability of
  testing an input for membership in a language.

  \subsection{Definitions}
  
  \begin{definition}{\textbf{Decision Problem}}
      
    A \textbf{Decision Problem} over a ring $R$ is a set, $S \subseteq
    R^\infty$.  A \textbf{Structured Decision Problem} is a set $S$,
    partitioned into subsets, $S_{yes}$ and $S_{no}$, respectively
    called the \textbf{Yes Instances} and \textbf{No Instances} of 
    $S$.
 
  \end{definition}

  \begin{definition}{\textbf{Characteristic Function}}

    Let $S = S_{yes} \cup S_{no} \subseteq R^{\infty}$.  The
    characteristic function,
    \functype{\charfunc_S}{S}{\set{0,1}}, is given by:

    $$\charfunc_S(x) = \twopartdef{1}{x \in S_{yes}}{0}{x \in S_{no}}$$
  \end{definition}

  \subsection{Notable Decision Problems}

  \begin{itemize}
    \subsubsection{Decision Problems over Arbitrary Rings}
    \bolditem{HN} - (Hilbert Nullstellensatz) Does some finite system
    of polynomial equations have a shared solution?  
    \bolditem{QUAD} - Does a finite system of quadratic equations have a solution?
    \bolditem{4-FEAS} - Does a finite system of degree-4 polynomials have a solution?  
    \bolditem{QA-FEAS} - Does a set of quasi-algebraic equations describe the empty set?  
    \bolditem{KP} - Knapsack Problem
    
    \subsubsection{Decision Problems over Ordered Rings}
    \bolditem{SA-FEAS} - Does a set of semi-algebraic equations describe the empty set?
    \bolditem{LPF} - Feasibility of Linear Programming
    \bolditem{TSP} - Travelling Salesman Problem
    
    \subsubsection{Decision Problems over $\integers$}
    \bolditem{IPF} - Feasibility of Integer Programming Problems

    \subsubsection{Decision Problems over $\integers_2$}
    \bolditem{SAT} - Boolean formula satisfiability

    \todo{Add bounded versions? (See \cite{B98} 84-91.)}

  \end{itemize}

  \subsection{Known Decidability Results}

  All of the above are known to be decidable (when well-defined) over
  $\complexes$, $\reals$, and $\integers_2$.  Over $\integers$,
  \textbf{IPF}, \textbf{TSP}, and \textbf{KP} are decidable, but
  \textbf{4-FEAS}, \textbf{QUAD}, \textbf{HN}, \textbf{QA-FEAS}, and
  \textbf{SA-FEAS} are not.  Over $\rationals$, \textbf{LPF},
  \textbf{TSP}, and \textbf{KP} are decidable.  It is an open problem
  whether \textbf{QUAD} and \textbf{HN} are decidable over $\rationals$.

  \section{The Class P}

The complexity class $P$ is meant to abstract a notion of problems
that are computationally tractable.  In general the class $P$ refers
to those decision problems which can be computable in some number of
steps that is polynomial in some relevant measure of the size of the
problem's input.  Depending on context and interpretation of inputs,
the relevant measure of size varies. In the classical theory of
computation, size refers to the length of the machine's binary input.
Since the binary representation length of an integer is logarithmic in
its magnitude, this definition is equivalent to the set of problems
that can be computed in time polylogarithmic in the magnitude of the
machine's input interpreted as an integer.  In other cases, we find it
more appropriate to interpret the input to an algorithm as a set of
distinct inputs, each of which is bounded in size, rather than a
single input.  (This is how we generally conceptualize the inputs to
sorting algorithms, for example).  In such a case, the relevant
measure of cost is taken to be the number of inputs, rather than the
magnitude of any single input.  In more rare cases, we may care about
both magnitude and number of inputs, in which case our relevant measure
of size may involve some product of the two measures.\\

We extend these intuitions to arbitrary rings by defining two
different measures of the magnitude of an input to a machine: the
\textbf{length} (corresponding to the number of distinct inputs to a
machine) and the \textbf{height} corresponding to the magnitude of
each input.  The overall \textbf{size} of an input is then taken to be
product of the length of an input with the maximum height of any
element of the input.  This naturally abstracts the case where we
interpret our input as a single value with unbounded magnitude
($length = 1$, $height = n$) as well as the case where we regard the
input as an array of bounded values ($length = n$, $height = 1$).

\subsection{Heights, Sizes,  and Costs}

\begin{definition}{\textbf{Height Function} over $R$}

  We define a \textbf{Height Function} on a ring $R$ to be a map:
  $$\mfunctype{ht_R}{R^\infty}{\reals}$$
\end{definition}

\note{As we shall see, the height function on a ring effectively
  defines what we consider to the the cost of performing a computation
  on an element of the ring.  Of particular note are the \emph{unit
    height} function, $ht_R(x) = 1$, and the \emph{bit height}
  function, $ht_R(x) = \log(|x| + 1)$.  We use the unit height
  function in cases where we are concerned with the algebraic
  complexity of an algorithm (ie, the number of algebraic operations
  required to compute the algorithm, independent of the values on
  which the operations are performed).  We use the bit height function
  when we want to capture the classical theory of computation over
  $\integers$.  Over $\rationals$, we give an alternative definition
  of bit height, $ht_\rationals(x) = (max(ht_\integers(p),
  ht_\integers(q))$, where $x = \frac{p}{q}$, $p$ and $q$ relatively
  prime.  This captures the bit cost of representing a rational $x$ as
  a pair of integers.  Unless noted otherwise, we generally use bit
  cost for computation over $\integers$ and $\rationals$, and unit
  cost otherwise.}

\begin{definition}{\textbf{Input Length}}
  
  The \textbf{length} of an input $x \in R^n$ to a machine over $R$ is
  given by: $$\len(x) = n$$
\end{definition}

\begin{definition}{\textbf{Size}}
  
  Let $R$ be a ring with height function $ht_R$.  The \textbf{size} of
  a polynomial $x \in R^n$ is given by:
  $$\size(x) = \len(x)\max_{1 \leq i \leq n}(ht_R(x_i))$$.
  \note{In the case that $ht_R$ is the unit height function, the size of
  the input to a machine reduces to just the length of the input.}
\end{definition}

Now that we have a definition of input size, we want to consider the
cost of performing a computation for a given machine on a given input.
We define this quantity to be the number of steps the machine takes
before halting times the maximum size of any value stored in the
machine's state space during computation.

\begin{definition}{\textbf{Computation Cost}}
  \label{def:size}
  
  Let $x \in R^n$ be an input in $\halting_M$ to a machine $M$ over
  $R$ with height function $ht_R$, and let $\compute_M(x) = (z_0, z_1,
  z_2, \ldots, z_T)$ The \textbf{cost} of $\compute_M(x)$ is given by:
  $$\cost(\compute_M(x)) = T*ht_{max}(\compute_M(x))$$
  where $ht_{max}(\compute_M(x))$ is given by:
  $$\max_{x_i \in \stateprojection(\compute_M(x))}(\size(x_i))$$
  (Recall that $\stateprojection(\compute_M(x))$ is the projection of
  $\compute_M(x)$ onto the first coordinate, which contains values in
  $\statespace_M$.)\\

  \todo{This is technically wrong for infinite dimensional machines,
    since size is only well-defined for finite-length vectors.  What
    we really want is a projection from $\statespace_M$ to $R^m$,
    where $k = \max(K_M, n) + T(x)$, which is the maximum number of
    coordinates we can write to in time $T$. See \cite{B98} pg 87.}

  Note that the above definition is only valid if $M$ halts on $x$,
  when this is not the case, we say that $\cost(\compute_M(x)) = \infty$.

  \note{The above definition departs slightly from the one given in
    Chapter 4 of \cite{B98}.  Blum et al. use the notation
    $\cost_M(x)$ rather than $\cost(\compute_M(x))$.  We do this to
    emphasize that cost is entirely determined by a height function
    combined with a path through the configuration space of $M$.  This
    will be important for us in Chapter 3 when we consider machines
    that can traverse multiple paths on a single input.}
  
\end{definition}

\begin{definition}{\textbf{Polynomial Time Machine}}

  A machine $M$ over $R$ is a \textbf{polynomial time} machine on $X
  \subset \inspace_M$ if there exist positive integers $c$ and $q$
  such that:
  $$cost_M(x) \leq c(size(x))^q \mid \forall x \in X$$
\end{definition}

\begin{definition}{\textbf{p-morphism}}
  
  A function, \functype{\varphi}{X}{Y \cup R^\infty} is called
  \textbf{polynomial-time computable} over $R$, or a
  \textbf{p-morphism} over $R$ if there exists a polynomial time
  machine over $R$ that computes $\varphi$.

\end{definition}

\todo{Rework this using decision machines instead of characteristic
  functions, since the former generalizes more naturally to NDET
  Machines.}

\begin{definition}{Membership in $P$}

  A decision problem $S \subseteq R^\infty$ is said to be in class $P$
  if its characteristic function, $\charfunc_S$ is a p-morphism.
\end{definition}

\section{Polynomial Time Reductions}

\begin{definition}{\textbf{p-reducibility}}

  A decision problem $S$ is \textbf{p-reducible} to a decision problem
  $S'$ if there exists a p-morphism,
  \functype{\varphi}{R^\infty}{R^\infty} such that $\varphi(S_{yes})
  \subseteq S'_{yes}$ and $\varphi(S_{no}) \subseteq S'_{no}$.  The
  map $\varphi$ is called a \textbf{polynomial-time reduction} from
  $S$ to $S'$.
n
  We denote the sentence \emph{$S$ is p-reducible to $S'$} by writing:

  $$S \preduce S'$$
  \end{definition}

  \note{ From the above definition it follows immediately that if
    $\varphi$ is a polynomial-time reduction from $S$ to $S'$, then $x
    \in S_{yes}$ if and only if $\varphi(x) \in S'_{yes}$.  Moreover,
    since compositions of polynomials are still polynomials, it should
    be clear that if a decision problem $S$ is p-reducible to a
    decision problem $S' \in P$, then $S \in P$ as well.}

\section{Notable Prior Results in Reducibility}

\begin{theorem}{\textbf{Known Reductions}}
  \begin{enumerate}
  \item Over any ring or field $R$, $HN \preduce QUAD$.
  \item Over any ordered ring or field $R$, $HN \preduce 4-FEAS$.
  \item Over $(\integers_2, >)$, 
    \begin{enumerate}
    \item $HN \preduce QUAD \preduce 4FEAS$
    \item $QUAD \preduce IP \preduce IPF$
    \end{enumerate}
  \end{enumerate}
\end{theorem}

\section{The Class NP}

The definition of NP over R given by Blum et al is a generalization of
the verifiability definition of NP given in classical complexity theory.

\begin{definition}{\textbf{Class NP} over $R$}

  A decision problem $S \subseteq R^\infty$ is in the class NP over a
  ring $R$ if there exist positive integers $c$ and $q$ and a a
  machine $M$ over R with $\inspace_M = R^\infty \times R^\infty$ such
  that:

  \begin{enumerate}
  \item if $x \in S$ then there exists a \emph{witness}, $w
    \in R^\infty$, such that $\computefn_M(x,w) = 1$ and
    $cost_M(x,w) \leq c(size(x))^q$.
  \item if $x \notin S$, then there is no such $w$.
  \end{enumerate}
\end{definition}

A classic simple example of a problem in NP is SUBSETSUM, which asks,
given a set of values $X = {x_1, x_2, x_3, \ldots x_k}$ and a target
$T$, is there a subset $S \subseteq X$ such that $\sum\limits_{x_i \in
  S} x_i = T$?  Though it is not clear how we might go about
efficiently finding such a subset, it is clear that if we are given a
candidate subset as a witness, we can quickly verify that it is, in
fact, a solution by simply adding the values and checking if the sum
is equal to $T$.\\

An important feature of the class $NP$ in the classical theory is the
existence of \textbf{NP Complete} problems, which are in a sense the
``hardest'' problems in $NP$.  More precisely, NP Complete problems
are those problems which lie in $NP$, and to which every problem in
$NP$ is reducible in polynomial time.  These problems can be shown to
exist in the generalized algebraic theory as well.

\begin{definition}{\textbf{NP Hard over R}}
  A decision problem $S$ is said to be \textbf{NP Hard} over R if
  every $S' \in NP_R$ is p-reducible to $S$.
\end{definition}

\begin{definition}{\textbf{NP Complete over R}}
  A decision problem $S$ is said to be \textbf{NP Complete} over R it
  is both NP Hard and in NP.
\end{definition}

\section{Known NP-Complete Problems}

The following results hold with respect to either unit or bit cost.
See \cite{B98} for details.  \todo{Should we present proofs or proof
  sketches here?}

\begin{itemize}
\item QA-FEAS is NP-Complete over any unordered ring or field R.
\item HN and QUAD are NP-Complete over (F, =) for any field F.
\item SA-FEAS is NP-Complete over any ordered ring or field R.
\item HN, QUAD, and 4-FEAS are NP-Complete over $\integers$,
  $\rationals$ and $\reals$, all considered with order.
\end{itemize}

\todo{Add bounded versions?}

%%% Local Variables: 
%%% mode: latex
%%% TeX-master: "main"
%%% End: 
