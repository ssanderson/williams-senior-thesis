\chapter{Background}

In this chapter, we present a basic development of an algebraic model
of computation due to Blum et al.  The material presented here draws
heavily from \cite{B89} and \cite{B98}.  Since the computing machines
developed by Blum et al. are essentially a generalization of the
classical Turing Machine, we begin with a brief review of Turing
Machines.  For a more thorough introductory reference to Turing
Machines, see \cite{S06}.  For Turing's original article introducing the
concept of a computing machine, see \cite{T36}.

\section{Classical Turing Machines}

A Turing Machine can be thought of as an infinitely long 1-way tape
divided into cells, on each of which can be written a single symbol.
Such a machine is outfitted with a tape head which can move left and
right as well as read and write symbols to and from the tape. A
machine's ``program'' consists of a set of states along with a
transition function that provides instructions for how the tape head
should behave given a symbol and a current machine state.  These ideas
are made precise as follows:

\begin{definition}{\textbf{Turing Machine}}

  A Turing Machine is a 7-tuple, $(Q, \Sigma, \Gamma, \delta, q_0, q_{accept}, q_{reject})$, where:

  \begin{itemize}
  \item $Q$ is a set of machine states.
  \item $\Sigma$ is an alphabet of input characters, not containing
    the blank symbol, $\blank$.
  \item $\Gamma$ is the tape alphabet, where $\blank \in \Gamma$, and
    $\Sigma \subset \Gamma$.
  \item \functype{\delta}{Q \times \Sigma}{Q \times \Gamma \times
      \set{\leftarrow, \rightarrow}} is a transition function.
  \item $q_0$ is the initial state of the machine.
  \item $q_{accept}$ is the machine's accept state.
  \item $q_{reject}$ is the machine's reject state.
  \end{itemize}

  The transition function $\delta$ determines how the machine passes
  between \textbf{configurations}, which are triples in $\Sigma^*
  \times Q \times \naturals$ encoding the symbols currently written on
  the tape, the machine's current state, and the position of the tape
  head.  At each computation step, the machine reads the symbol
  $\sigma$ on the cell currently occupied by the tape head.  If the
  machine is in state $q$, it then calculates $\delta(q, \sigma)$ =
  $(q', \sigma', d \in \set{\leftarrow, \rightarrow})$, transitions
  into state $q'$, writes $\sigma'$ to the current cell, and
  increments or decrements the location of the tape head depending on
  the value of $d$.  In the case that the machine head would move left
  when on the leftmost cell, it stays in place.  Computation ends when
  a machine transitions into either $q_{accept}$ or $q_{reject}$, at
  which point the machine is said to either \textbf{accept} or
  \textbf{reject} its input, respectively.  A Turing Machine is said
  to \textbf{halt} on input $x$ if it either accepts or rejects $x$.
  Upon halting, the \textbf{output} of a machine is the string left
  behind on the machine's tape.
  
\end{definition}

\begin{example}{}

  The following Turing Machine reads in a binary string $s$ and
  outputs the string given by replacing each $0$ in $s$ with a $1$ and
  each $1$ with a $0$.  It accepts on all inputs.

  \begin{itemize}
  \item $Q = \set{q_0, q_{accept}, q_{reject}}$
  \item $\Gamma = \set{\blank, 0, 1}$
  \item $\Sigma = \set{0,1}$
  \item $\delta(q_0, 0) = (q_0, 1, \rightarrow)$
  \item $\delta(q_0, 1) = (q_0, 0, \rightarrow)$
  \item $\delta(q_0, \blank) = (q_{accept}, \blank, \leftarrow)$
  \end{itemize}

  Figure \ref{fig:turing-example} illustrates the computation of our
  example machine on the input $1001$. In each computation step but
  the last, the machine flips the parity of the symbol beneath the
  tape, then moves to the right and ``transitions'' from $q_0$ to
  $q_o$.  When the machine reads the blank symbol $\blank$, it leaves
  the same symbol on the tape, transitions to the accept state and
  moves the tape head to the left before exiting with output $0110$.

  \begin{figure}[p]
    \begin{center}
      \tape[1]{$q_0$}{{1,0,0,1,\blank,$\cdots$}} \vspace{1mm}
      \tape[2]{$q_0$}{{0,0,0,1,\blank,$\cdots$}} \vspace{1mm}
      \tape[3]{$q_0$}{{0,1,0,1,\blank,$\cdots$}} \vspace{1mm}
      \tape[4]{$q_0$}{{0,1,1,1,\blank,$\cdots$}} \vspace{1mm}
      \tape[5]{$q_0$}{{0,1,1,0,\blank,$\cdots$}} \vspace{1mm}
      \tape[4]{$q_{accept}$}{{0,1,1,0,\blank, $\cdots$}} \vspace{1mm}
    \end{center}
    \caption{An Example Turing Machine Computation}
    \label{fig:turing-example}
  \end{figure}
\end{example}

\section{Finite Dimensional Machines}

A BSS Machine is in essence a Turing Machine with several key features
abstracted to a more general setting. We replace the classical finite
tape alphabet with elements of a ring/field, and we replace the
classical operations of reading and writing to individual cells with
order/equality comparisons and computations of componentwise
polynomial/rational maps.
 
To build intuition about how BSS machines operate and how we can
analyze their properties, we first present Blum et al's definition of
a \emph{Finite Dimensional Machine}, which corresponds in the
classical theory to a Turing Machine with a tape of finite length.
Having built some intuition with the finite dimensional model, we then
show how we can generalize to the infinite dimensional case, allowing
us to consider machines that can process arbitrarily large inputs.

In what follows, let $R$ be a commutative ring with unity, and let $l,
m$, and $n$ be positive integers.

\begin{definition}{\textbf{Finite Dimensional Machine} over a ring
    \textbf{R}}
  
  A \textbf{Finite Dimensional Machine} (FDM) $M$ over $R$ is a finite
  directed graph with node set $\nodes$.  The collection $\nodes$
  contains three distinguished nodes: the \textbf{input node}
  $\eta_1$, the \textbf{accept node} $\eta_{accept}$, and the
  \textbf{reject node} $\eta_{reject}$. The input node has indegree 0
  and outdegree 1; the accept and reject nodes have outdegree 0.  The
  remaining elements of $\nodes$ are divided into two types:
  \emph{computation} and \emph{branch}.  Computation nodes have
  outdegree 1; branch nodes have outdegree 2.  We use the expression
  ``next node(s)'' of a node $\eta$ to refer to the node(s) to which
  $\eta$ has outgoing edges.\footnote{The definition used here departs
    slightly from that used by Blum et al. in \cite{B98} and
    \cite{B89}.  The main difference is our replacement of a single
    output node with the pair of accept and reject nodes, which
    reflects our emphasis on decision problems instead of computable
    functions.}\\
  
  An FDM has three associated spaces: an input space $\inspace_M
  \subseteq R^n$, a state space $\statespace_M \subseteq R^m$, and an
  output space $\outspace_M \subseteq R^l$.  (Recall that $R^k$
  denotes the set of $k$-tuples containing elements in $R$).\\

  Each node has an associated map and a set of next nodes, which vary
  in structure in accordance with the node's type.

  \begin{enumerate}
  \item Associated with the input node $\eta_1$ is a linear map
    \functype{I}{\inspace_M}{\statespace_M}. We denote the input node
    by $\eta_1$, and we denote the unique next node of $\eta_1$ by
    $\beta_{\eta_1}$ or just $\beta_1$.
    
  \item Associated with each computation node $\eta$ is a
    componentwise polynomial map
    \functype{g_\eta}{\statespace_M}{\statespace_M}. We denote the
    unique next node of $\eta$ by $\beta_\eta$.
    
  \item Associated with each branch node $\eta$ is a nonzero
    polynomial function \functype{h_\eta}{\statespace_M}{R}.  We
    associate the condition $h_\eta(x) \geq 0$ with one of the next
    nodes of $\eta$, called the \textbf{Yes Node} of $\eta$.  We
    symbolically denote the Yes Node by $\beta_\eta^+$. We associate
    the condition $h_\eta(p) < 0$ with the other next node of $b$,
    which we call the \textbf{No Node} and denote $\beta_\eta^-$.
    
  \item Associated with the accept and reject nodes is a linear map
    \functype{O}{\statespace_M}{\outspace_M}.
  \end{enumerate}
\end{definition}

\textbf{Note: }The above definition is subject to modification
depending on whether the ring $R$ is a field and whether it is endowed
with a natural order.

In the case that $R$ is a field, we may want to allow our machines to
take advantage of the natural division operation by allowing rational
maps rather than polynomial maps for $g_\eta$ and $h_\eta$.  In such a
case, care needs to be taken to ensure that division is always
well-defined by preventing divisions by $0$.  We do so by adopting the
convention that each node that computes a rational function is
preceded by a subroutine that computes all relevant denominators,
entering an infinite loop if any of them are zeros.  This allows us to
assume that computations are always well-defined.

When $R$ is unordered, when $R = \complexes$ for example, we replace
the branching conditions $h(x) > 0$ and $h(x) < 0$ with $h(x) = 0$ and
$h(x) \neq 0$, respectively.\\

\begin{example}
  Figures \ref{fig:newton-high} and \ref{fig:newton-low} show how one
  might implement an FDM to carry out Newton's Method for determining
  approximate zeros of a polynomial.  Figure \ref{fig:newton-high}
  gives a ``high-level'' view of what each node does logically, while
  Figure \ref{fig:newton-low} demonstrates the underlying
  implementation.  Because we are currently considering only finite
  dimensional machines, we restrict ourselves to polynomials of degree
  5.  We will show later how we can modify our definitions to allow
  for uniform machines that can process inputs of arbitrary size.
\end{example}
\begin{figure}[p]
  \centering

  \begin{tikzpicture}[node distance=4cm]

    \node (init)   [input, draw] {Input: $p \in \reals[x]$, $z \in \reals$, $\epsilon \in \reals$};
    \node (comp)   [compute, draw, below of=init]   {$z = z - \frac{p(z)}{p'(z)}$};
    \node (check)  [branch, draw, below of=comp]    {$|p(x)| - \epsilon < 0$};
    \node (accept) [accept, draw, below of=check]   {Output $z$};
    
    \draw [->, thick] (init) --                                      (comp);
    \draw [->, thick] (comp) --                                      (check);
    \draw [->, thick] (check) to [out=0, in=0] node[auto, swap] {No} (comp);
    \draw [->, thick] (check) to node[auto] {Yes}                    (accept);

  \end{tikzpicture}

  \caption{Newton Machine - High Level}
  \label{fig:newton-high}
\end{figure}

\begin{figure}[p]
  \centering
  
  \begin{tikzpicture}[node distance=4cm, label distance=.3cm]

    \node(init) [input, draw, label=left:\Large $\eta_1$] 
    {Input: $x = (x_1, x_2, x_3, x_4, x_5, x_6, x_7, x_8)$;};
    \node(comp) [compute, draw, label=left:\Large $\eta_2$, below of=init] {
      $g(x) = (x_1 - 
      \frac{x_3x_1^5 + x_4x_1^4+ x_5x_1^3 + x_6x_1^2 + x_7x_1 + x_8}
      {5x_3x_1^4 + 4x_4x_1^3 + 3x_5x_1^2 + 2x_6x_1 + x_7}, 
      x_2, x_3, x_4, x_5, x_6, x_7, x_8)$};
    \node(check)[branch, draw, label=left:\Large $\eta_3$, below of=comp] {$h(x) = x_1^2 - x_2^2$};
    \node(accept)[accept, draw, label=left:\Large $\accept$,below of=check] {$O(x) = x_1$};
    \node(reject)[reject, draw, label=above:\Large $\reject$,left of=accept,
                  node distance= 6cm] 
    {$O(x) = x_1$};

    \draw [->, thick] (init) --                                             (comp);
    \draw [->, thick] (comp) --                                            (check);
    \draw [->, thick] (check) to [out=0, in=0] node[auto, swap] {$h(x) > 0$}(comp);
    \draw [->, thick] (check) to node[auto] {$h(x) < 0$}                  (accept);

    \node (legend) [rectangle, draw, fill=black!10, right of=accept, node distance=6cm] {
      \makecell[l]{
        $\nodes = \set{\eta_1, \eta_2, \eta_3, \accept, \reject}$\\
        $\inspace_M = \statespace_M = \outspace_M = \reals^8$\\
        Computation Nodes: $\set{\eta_2}$\\
        Branch Nodes: $\set{\eta_3}$
      }
    };
    
  \end{tikzpicture}

  \caption{Newton Machine - Low Level}
  \label{fig:newton-low}
\end{figure}

\begin{definition}{\textbf{Configuration}}  

  A \textbf{configuration} of an FDM $M$ is a pair $(\eta, x) \in
  \nodes \times \statespace_M$. The set $\nodes \times \statespace_M$
  is called the \textbf{configuration space} of $M$.  A configuration
  is said to be \textbf{terminal} if $\eta \in \set{\accept,
    \reject}$, and \textbf{non-terminal} otherwise.

\end{definition}

For each non-terminal configuration $z = (\eta, x)$, $M$ defines a
unique next configuration $z' = (\eta', x')$ as follows:

\begin{itemize}
\item If $\eta$ is the input node, $z' = (\beta_{1}, x)$.
\item If $\eta$ is a computation node, $z' = (\beta_{\eta}, g_\eta(x))$.
\item If $\eta$ is a branch node, $z' = \twopartdef{(\beta_\eta^+,
    x)}{h_\eta(x) \geq 0}{(\beta_\eta^-, x)}{h_\eta(x) < 0}$.
\end{itemize}

It will occasionally be useful to us to have a compact notation for
writing ``$z'$ is the next configuration of $z$.''  We do so using the
expression $z \yields z'$, which we read as $z$ \textbf{yields} $z'$.\\

We are now in a position to define what it means for a machine $M$ to
perform a computation on some input, $x$.  The basic idea is that a
computation is a path through $\nodes \times \statespace$ that begins
with $(\eta_1, I_M(x))$ and proceeeds by following next
configurations, halting upon reaching either the accept or reject node
of $M$.  The output value of a computation is the result of $O_m$
applied to whatever is left in $\statespace_M$ when a terminal
configuration is reached.\\

\begin{definition}{\textbf{Halting Computation and time-t Halting Set}}

  A \textbf{halting computation} performed by an FDM $M$ on input $x$
  is a sequence of configurations: 
  $$\compute_M(x) = (z_0, z_1, \ldots, z_t)$$ such that
  \begin{align*}
    z_0 &= (\eta_1, I(x))\\
    z_t &\in \set{(\accept, x_t), (\reject, x_t)}\\
    z_i &\yields z_{i+1} \text{ for } 0 \leq i < t
  \end{align*}
  
  If a halting computation of length $t$ exists for an input $x$, we
  can write the following:

  \begin{align*}
    x \in \halting_M(t)\\
    T(x) = t\\
    \computefn_M(x) = O(x_t)
  \end{align*}

  In the above expressions, $x_t$ is the state space component of the
  terminal configuration of $\compute_M(x)$.  The symbol
  $\halting_M(t)$ denotes the \textbf{time-t Halting Set} of $M$.  The
  function \functype{T}{\inspace_M}{\naturals} is the \textbf{Halting
    Time} function, and
  \functype{\computefn_M}{\inspace_M}{\outspace_M} is the
  \textbf{computation function} of $M$.  

  The \textbf{Halting Set} $\halting_M$ of a machine $M$ is the union
  over all $t > 0$ of the time-t halting sets for $M$.

\end{definition}

\note{Since each configuration of an FDM yields a unique next
  configuration, if a halting computation exists for a machine $M$ on
  input $x$, it is unique.  However, not every input defines a halting
  computation.  It is easy enough, for example, to construct machines
  which will enter infinite loops without ever reaching the output
  node.  In such a case, we define $\compute_M(x)$ to be the infinite
  sequence $$z_0, z_1, z_2, \ldots, $$ such that

  \centerline{$z_0 = (\eta_1, I(x))$, and $z_i \yields z_{i+1}$ for $i \geq 0$}
}

\vspace{\baselineskip}

Intuitively, we can think about the computation of an FDM, $M$ as the
sequence defined by ``following the flow'' of the node graph while
applying the appropriate transformations to $\statespace_M$ at
computation nodes. Given an input $x$, $M$ starts in configuration
$\eta_1, I(x)$. At each computation node $\eta$, it computes
$g_\eta(x)$, where $x$ is the current value of stored in the machine's
state space.  $g_\eta(x)$ then becomes the new value of the state
space, and the current node becomes $\beta_{\eta}$.  At each branch
node $\eta$, $M$ computes $h_\eta(x)$, where again $x$ is the
currently stored value.  If $h_\eta(x) \geq 0$, the new node is
$\beta_\eta+$.  Otherwise it is $\beta_\eta^-$.  The machine halts
when it eventually reaches the $\accept$ or $\reject$, at which point
it applies the function $O(x)$ to its stored value, which is
taken as the machine's output.\\

\begin{example}
  Figure \ref{fig:q-recog} shows a machine over $\reals$ which, on
  input $x \in \rationals^+ $ returns the pair of relatively prime
  integers $(p,q)$ such that $x = \frac{p}{q}$.  If $x$ is not a
  positive rational, the machine does not halt.
\end{example}

\begin{figure}[p]
  \centering
  \begin{tikzpicture}[node distance=4.8cm]

    \node(init) [input, draw, label=left:\Large $\eta_1$] 
    {$I(x) = (x, 0, 1)$};
    \node(test) [branch, draw, below of=init, label=45:\Large $\eta_2$] 
    {$h(x) = x_1 - \frac{x_2}{x_3}$};
    \node(flip) [compute, draw, left of=test, label=left:\Large $\eta_3$,]
    {$g(x)= (x_1, 1, x_2+1)$};
    \node(fliptest) [branch, draw, below of=flip, label=left:\Large $\eta_4$] 
    {$h(x) = x_3$};
    \node(inc)  [compute, draw, below of=fliptest, label=left:\Large $\eta_5$,] 
    {$g(x) = (x_1, x_2+1, x_3-1)$};
    \node(accept)   [accept, draw, below of=test, right of=test, label=left:\Large $\accept$] 
    {$O(x) = (x_2, x_3)$};
    \node(reject)   [reject, draw, below of=accept, label=left:\Large $\reject$, node distance=2.4cm] 
    {$O(x) = (x_2, x_3)$};

    \draw[->, thick] (init) -- (test);
    \draw[->, thick] (flip) -- (test);
    \draw[->, thick] (test) to [out=270, in=0] node [auto, swap] {$h(x) \neq 0$} (inc);
    \draw[->, thick] (test) to [out=0, in=90] node[auto] {$h(x)=0$} (accept);
    \draw[->, thick] (fliptest) to [out=0, in= 225] node[auto] {$h(x) = 0$} (test);
    \draw[->, thick] (fliptest) to [out=90, in= 270] node[auto] {$h(x) \neq 0$}(flip);
    \draw[->, thick] (inc) -- (fliptest);

    \node (legend) [rectangle, draw, fill=black!10, below of=reject, node distance=3.0cm] {
      \makecell[l]{
        $\nodes = \set{\eta_1, \eta_2, \eta_3, \eta_4, \eta_5, \accept, \reject}$\\
        $\inspace_M = \reals, \statespace_M = \reals^3, \outspace_M = \reals^2$\\
        Computation Nodes: $\set{\eta_3, \eta_5}$\\
        Branch Nodes: $\set{\eta_2, \eta_4}$\\
        $\halting_M = \rationals^+$\\
        $\computefn_M(x) = (p,q) \mid x = \frac{p}{q}, \text{ gcd}(p,q) = \, 1$
      }
    };

  \end{tikzpicture}
  \caption{Recognizer for $\rationals$ over $\reals$}
  \label{fig:q-recog}
\end{figure}
  
\begin{figure}[p]
  \centering
  
  \begin{tikzpicture}[node distance=4.8cm]

    \node(init)     [input, draw, label=left:\Large $\eta_1$] {$I(x) = (x, 0)$};
    \node(magcheck) [branch, draw, below of=init, label=135:\Large $\eta_2$] {$h(x) = x_1 - x_2$};
    \node(equal)    [branch, draw, right of= magcheck, node distance= 5.2cm, label=above:\Large $\eta_3$] {$h(x) = x_2 - x_1$};
    \node(inc)      [compute, draw, below of=magcheck, label=right:\Large $\eta_4$] {$g(x) = (x_1, x_2+1)$};
    \node(accept)   [accept, draw, below of=equal, right of=equal, label=left:\Large $\accept$] 
    {$O(x) = x_2$};
    \node(reject)   [reject, draw, above of=equal, right of=equal, label=left:\Large $\reject$] 
    {$O(x) = x_2$};

    \draw[->, thick] (init) -- (magcheck);
    \draw[->, thick] (magcheck) to node[auto] {$h(x) < 0$} (inc);
    \draw[->, thick] (inc) to [out=180, in=180] (magcheck);
    \draw[->, thick] (magcheck) to node [auto] {$h(x) \geq 0$} (equal);
    \draw[->, thick] (equal) to [out=-45, in=90]  node  [auto] {$h(x) \geq 0$} (accept);
    \draw[->, thick] (equal) to [out=45, in=270] node  [auto, swap] {$h(x) < 0$} (reject);

  \end{tikzpicture}

  \caption{Decision Machine for $\integers^+$}
  \label{fig:z-decide}
\end{figure}

\begin{definition}{\textbf{Computation Path} and \textbf{State Trajectory}}
  
  The \textbf{Computation Path} $\computepath_M(x)$ of $M$ on
  input $x$ is the projection of $\compute_M(x)$ under the map:
  $$\mfunctype{\pathprojection}{\fullstate}{\nodes} \mid \pathprojection(\eta, x) = \eta$$
  
  The \textbf{State Trajectory} of $M$ on input $x$ is the projection
  of $\compute_M(x)$ under the map:
  $$\mfunctype{\stateprojection}{\fullstate}{\statespace} \mid \stateprojection(\eta, x) = x$$

\end{definition}

\note{The computation path and state trajectory of a machine encode,
  respectively, the sequence of nodes a machine traverses during a
  computation, and the sequence of values stored in the machine's
  state space during a computation.  In Section \ref{sec:measuring} we
  will also want to have handy the \textbf{modified state trajectory}
  of dimension $k$, which encodes the sequence of values stored in the
  first $k$ coordinates of $\statespace_M$ by the sequence.  We denote
  this sequence by $\stateprojection^k$.}

\begin{example}
  Let $M$ be the machine described in Figure \ref{fig:q-recog}. For
  the input $x = 2$, we have:
  \begin{eqnarray*}
    \phi_M(x) &=& (\eta_1, (2,0,1)), (\eta_2, (2,0,1)), (\eta_5, (2,0,1)), (\eta_4, (2,1,0)), (\eta_3, (2,1,0)),\\
    &&  (\eta_2, (2,1,2)), (\eta_5, (2,1,2)), (\eta_4, (2,2,1)), (\eta_2, (2,2,1)), (\accept, (2,2,1))\\
    && \\
    \computepath_M(x) &=& \eta_1,\, \eta_2,\, \eta_5,\, \eta_4,\, \eta_3,\, \eta_2,\, \eta_5,\, \eta_4,\, \eta_2,\, \accept
  \end{eqnarray*}

  The State Trajectory of $M$ on input $x$ is given by:
  $$(2,0,1), (2,0,1), (2,0,1), (2,1,0), (2,1,0), (2,1,2), (2,1,2), (2,2,1), (2,2,1), (2,2,1)$$
\end{example}

\begin{definition}{\textbf{Finite-Dimensionally Computable Function}}
  
  A function \functype{\varphi}{X}{R^m}, $X \subset R^n$ is
  \textbf{f.d. computable} over $R$ if there exists a
  finite-dimensional machine, $M$ over $R$ such that $\varphi(x) =
  \compute_M(x) \mid \forall x \in X$.
  
\end{definition}

\begin{definition}{\textbf{Decidable, Recognizable, and Co-Recognizable Sets}}
  
  A set $S \subset R^n$ is \textbf{finite-dimensionally decidable} if
  there exists an FDM $M$ with $\inspace_M = R^n$ such that $M$
  accepts $x$ if and only if $s \in S$, and $M$ rejects $x$ if and
  only if $s \notin S$.  Such a machine $M$ is said to \textbf{decide}
  $S$.\\

  A set $S \subset R^n$ is \textbf{finite-dimensionally recognizable}
  if there exists an FDM $M$ with $\inspace_M = R^n$ such that $M$
  accepts $x$ if and only if $x \in S$.  Such a machine is said to
  \textbf{recognize} $M$.\\

  A set $S \subset R^n$ is \textbf{finite-dimensionally
    co-recognizable} if there exists an FDM $M$ with $\inspace_M =
  R^n$ such that $M$ accepts $x$ if and only if $x \in S$.
\end{definition}
  
\textbf{Note:} Here we again depart slightly from the development in
\cite{B89} and \cite{B98}.  Blum et al. do not explicitly appeal to
notions of acceptance and rejection.  Instead they consider machines
that output $1$ on some inputs and $0$ on other inputs.  For
completeness, we also present the BSS characterization of
decidability.  It is straightforward to verify that classes of
decidable, recognizable, and co-recognizable languages are the same
under either set of definitions.

\begin{definition}{\textbf{Decidable, Recognizable, and Co-Recognizable Sets (BSS)}}

  A set $S \subset R^n$ is \textbf{(f.d.) decidable} over $R$ if the
  function \functype{\chi_S}{R^n}{R} is (f.d) computable over $R$,
  where:
  
  $$\chi_S(x) =  \twopartdef{1}{x \in S}{0}{x \notin S}$$
  
  A set $S \subset R^n$ is \textbf{(f.d) recognizable} or
  \textbf{(f.d) semi-decidable} over $R$ if the function
  \functype{\chi_S}{R^n}{R} is (f.d.) computable over $R$, where:
  
  $$\chi_S(x) = \twopartdef{1}{x \in S}{0 \text{\emph{ or undefined}}}{x \notin S}$$
  \vspace{\baselineskip}
  
  A set $S \subset R^n$ is \textbf{(f.d.) co-recognizable} over $R$ if
  the function \functype{\chi_S}{R^n}{R} is (f.d.) computable over $R$,
  where:
  
  $$\chi_S(x) =  \twopartdef{1}{x \notin S}{0 \text{\emph{ or undefined}}}{x \in S}$$
  
\end{definition}

\begin{example}
  The machine presented in Figure \ref{q-recog} shows that
  $\rationals$ is finite-dimensionally recognizable over $\reals$.
  The machine presented in Figure \ref{z-decide} shows that
  $\integers$ is decidable over $\reals$.
\end{example}

It is fairly straightforward to see that the recognizable sets over
$R$ are precisely the halting sets over R and the co-recognizable sets
are those whose complements are recognizable. Less obviously, the
decidable sets are precisely those that are both recognizable and
co-recognizable.

\begin{proposition}{A set $S \subset R^n$ is decidable if and only if it is
  both recognizable and co-recognizable.}
\label{prop:dec=rec+corec}
\end{proposition}
\begin{proofsketch}
  
  A machine that acts as a decider for a set $S$ is by definition a
  recognizer for both $S$ and $S'$.  Conversely, if $M$ and $M'$ are
  recognizers for $S$ and $R^n - S$, we can contruct a decision
  machine for $S$ by simulating $M$ and $M'$ in parallel, outputting 1
  if $M$ halts and 0 if $M'$ halts.  Since one of $M$ and $M'$ must
  halt eventually, we are guaranteed that our constructed machine will
  halt for all inputs.

  \note{The parallelization construction generalizes nicely to any
    integral number of machines. See \cite{B98} for concrete details.}

\end{proofsketch}

\section{Characterization of Finite-Dimensional Halting Sets}

  \begin{definition}{\textbf{Semi-algebraic} and \textbf{Quasi-algebraic Set}}
    
    A set $S \subset R$ is \textbf{basic semi-algebraic} over an
    ordered ring $R$ if $S$ is the set of elements in $R$ satisfying
    a finite system of polynomial equalities and inequalities. A set
    is \textbf{semi-algebraic} over $R$ if it is a finite union of basic
    semi-algebraic sets.
    
    A set $S \subset R$ is \textbf{basic quasi-algebraic} over an
    unordered ring $R$ if $S$ is the set of elements in $R$ satisfying
    a finite system of polynomial equalities. A set is
    \textbf{quasi-algebraic} over $R$ if it is a finite union of basic
    quasi-algebraic sets.
    
  \end{definition}

  \begin{example}
    The upper half of the unit disc in $\reals^2$ is a basic
    semi-algebraic set defined by solutions to the equations:
    \begin{eqnarray*}
      x^2 + y^2 < 0\\
      y > 0
    \end{eqnarray*}

    In $\reals^3$, the surface of the unit sphere is a basic
    quasi-algebraic set defined by solutions to the equation:
    \begin{eqnarray*}
      x^2 + y^2 + z^2 = 1
    \end{eqnarray*}
  \end{example}

  \begin{definition}{\textbf{Initial computation path} and \textbf{time-T halting path set}}

    Let $M$ be a machine over $R$, and let $x$ be an element of
    $\inspace_M$, and let $\computepath_M(x) = (\eta_0, \eta_1, \ldots,
    \eta_k, \eta_{k+1}, \ldots)$ be the computation path of $M$ on
    $x$.\\
    
    The \textbf{initial computation path} on $\computepath_M(x)$ of
    length $k$ is given by: 
    $$\computepath_M^k(x) = (\eta_0, \eta_1, \ldots,\eta_k)$$

    The \textbf{time-T halting path set}, $\haltpaths_M^T$, of $M$ is
    given by:
    $$\haltpaths_M^T = \set{\computepath_M(x) \mid x \in \halting_M(T)}$$
  \end{definition}

  Recall that $\computepath_M(x)$ encodes the sequence of nodes
  traversed by $M$ on input $x$, so $\computepath_M^k(x)$ encodes the
  first $k$ nodes traversed by $M$ on input $x$.  The collection
  $\haltpaths_M(T)$ is the set containing all possible initial paths
  of length less than or equal to $T$.\\

  \begin{definition}{\textbf{Coincidence Set}}

    Let $M$ be a machine over a ring $R$, and let $\computepath =
    \computepath_M^k(x)$ be the initial computation path of length $k$
    for some $x \in \inspace_M$. The \textbf{coincidence set}
    $\coincidence_M(\computepath)$ of $\computepath$ is the set of
    elements in $\inspace_M$ whose initial computation path of length
    $k$ is $\gamma$. It is defined by:
    $$\coincidence_M(\computepath) = 
    \set{p' \in \inspace_M \mid \computepath_{x'}(k) = \computepath}$$
  \end{definition}
    
  \note{For each $k > 0$, the coincidence sets on paths of length $k$
    partition $\inspace_M$ into sets of inputs that cause $M$ to
    traverse different sequences of nodes.  Thus belonging to the same
    coincidence set of length defines an equivalence relation on
    members of $\inspace_M$.  Thus, for each $k > 0$, the coincidence
    sets of $M$ on paths of length $k$ partition $\inspace_M$ into
    equivalence classes.}

  \begin{theorem}{\textbf{Path Decomposition Theorem (Finite-Dimensional Version)}}
    
    For any FDM $M$ over $R$, we have the following:

    \begin{enumerate}
    \item For any $T > 0$, the time-T halting set $\halting_T$ is a
      semi-algebraic set if $M$ branches on $>$, and $\halting_T$ is
      quasi-algebraic if $M$ branches on $=$.

    \item The halting set $\halting_M$ is a countable union of
      disjoint semi(quasi)-algebraic sets.

    \item For $\computepath \in \haltpaths_M^T$, $M$ computes a
      polynomial function on elements of $\coincidence_\computepath$.
      In other words, $\computefn_M|\coincidence{\computepath}$ is a
      polynomial map.
    \end{enumerate}
    
    \label{thm:pdt-finite}
  \end{theorem}

  \begin{proofsketch}
    
    The main idea is that the coincidence sets of a given computation
    path form basic semi(quasi)-algebraic sets.  We can partition each
    time-T halting set into finitely many coincidence sets, and there
    are countably many time-halting sets.  A full characterization of
    the algebraic equations defining each $\coincidence(\computepath)$
    can be found in \cite{B98}, Section 2.3.

    \begin{itemize}
    \item \textbf{(1)} follows from the fact that there are finitely
      many halting computation paths paths of length $T$.  The set of
      elements that follow a particular path is determined by the
      outputs of the branching functions along the path, which in turn
      correspond to sequences of polynomial equalities and
      inequalities.  It follows that each coincidence set of a halting
      computation path is basic semi(quasi)-algebraic. Since we can
      partition $\halting_M^T$ into coincidence sets, each of which is
      basic semi(quasi)-algebraic, $\halting_M^T$ is a disjoint union
      of basic semi(quasi)-algebraics.
    \item \textbf{(2)} follows from the fact that $\halting_M$ is just
      the union over all $T \in \naturals$ of $\halting_M^T$.
    \item \textbf{(3)} follows from the fact that $\computepath$
      corresponds to exactly one sequence of computation nodes, so
      $\computefn_M|\coincidence{\computepath}$ computes the
      composition of $g_{\eta}$ for $\eta \in \computepath$.
    \end{itemize}
  \end{proofsketch}
  
  \begin{corollary}
    Every finite-dimensionally recognizable set is a countable union
    of disjoint semi(quasi)-algebraic sets.
  \end{corollary}

  \begin{theorem}
    $\rationals$ is not finite-dimensionally decidable over $\reals$.
  \end{theorem}
  \begin{proof}
    By Proposition \ref{prop:rec+corec}, it suffices to show that
    $\reals - \rationals$ is not recognizable.  It is well-known that
    the semi-algebraic sets are finite unions of either open, closed,
    or half-open intervals (note that closed intervals include single
    points).  Since $\reals - \rationals$ is totally disconnected, it
    contains no open intervals, so the only semi-algebraic subsets of
    $\reals-\rationals$ are finite sets of points.  By Thereom
    \ref{thm:pdt-finite}, if $\reals-\rationals$ is decidable over
    $\reals$, then it can be expressed as a countable union of
    disjoint semi-algebraic sets.  But this would imply that
    $\reals-\rationals$ could be given by a countable union of finite
    sets, implying that $\reals-\rationals$ is countable, a
    contradiction.
  \end{proof}
    
\section{Uniform Machines over $R$}

  So far, the machines we have considered can only access a
  finite-length region of ``memory,'' which has size bounded by the
  dimension of $\statespace_M$.  We have seen that this restriction
  does not impede our ability to solve certain classes of problems,
  such as determining whether an arbitrary real number is an integer
  or a rational.  Some problems, however, are naturally understood as
  ranging over an input space of unbounded length.  For example, in
  the case of our Newton's Method machine (Figure
  \ref{fig:newton-low}), we might like to be able to take as input a
  polynomial of arbitrary degree, rather than have to define different
  machines for different-sized polynomials.  In other words, we would
  like to be able to define \textbf{uniform machines}, which can
  compute on inputs of arbitrary size.\\

  \todo{Talk about Universal Machine as well}

  Allowing for uniform machines requires two major changes to the
  finite-dimensional model.  First, we allow the input, state, and
  output spaces to have unbounded length.  Second, we introduce a new
  node type, which makes it possible for a machine to compute with
  values stored at high indices in the newly expanded state space.\\
  
  \begin{definition}{\textbf{$R^\infty$ and $R_\infty$}}

    We denote by $R^\infty$ the union $\bigcup_{n \geq 0} R^n$. Note
    that although $R^\infty$ contains tuples of arbitrary dimension,
    any particular element of $R^\infty$ is an element of $R^n$ for
    some particular $n$.\\

    We denote by $R_\infty$ the \emph{bi-infinite direct sum} over
    $R$, which consists of elements of the form $$(\ldots x_{-2},
    x_{-1}, x_0.x_1, x_2, \ldots)$$ where the period is a
    distinguished marker between $x_0$ and $x_1$.  Often, when writing
    an element of $R_\infty$, we omit the commas and write simply
    $$\ldots x_{-2}x_{-1}x_0.x_1x_2 \ldots$$
  \end{definition}

  \subsection{Maps on $R^\infty$ and $R_\infty$}
    
  Our uniform machines will perform computations on inputs from
  $R^\infty$, using $R_\infty$ as a state space, but an essential
  feature of our model is that a machine over $R$ can be given a
  finite representation with respect to the elements of $R$.  This
  means that the computation and branching functions of our machines
  will always effectively be functions from $R^n \rightarrow R^m$.  We
  extend these functions to $R^\infty$ and $R_\infty$ as
  follows.\footnote{As was the case for finite-dimensional machines,
    ``polynomial'' can be replaced here with ``rational'' whenever $R$
    is a field.}

  We first consider the extension of functions suitable to serve as
  branching functions.  

  \begin{definition}{\textbf{Polynomial Map: $R_\infty \rightarrow R$}}
    
    Let \functype{h}{R^m}{R} be a polynomial map of degree $d$.  Then
    $h$ defines a related polynomial map
    \functype{\widehat{h}}{R_\infty}{R} of \emph{dimension} $m$ and
    \emph{degree} $d$, which is given by:
    $$\widehat{h}(x) = h(x_1, \ldots, x_m)$$
  \end{definition}

  Functions suitable for use in computation nodes are only slightly
  trickier.  
  
  \begin{definition}{\textbf{Polynomial Map: $R_\infty \rightarrow R_\infty$}}
    
    Let \functype{g}{R^m}{R^n} be a function given by $$g(x) =
    (g_1(x), g_2(x), \ldots , g_n(x))$$ where each $g_i$ is a
    polynomial map of degree at most $d$ from $R_m$ to $R$.  Let $x
    \in R_\infty = \ldots x_{-2}x_{-1}x_0.x_1x_2 \ldots$. We say that
    $g$ defines an associated polynomial map
    \functype{\widehat{g}}{R_\infty}{R_\infty} of \emph{dimension} $m$
    and degree $d$ as follows:
    $$\widehat{g}(x) = 
    \ldots x_{-2}x_{-1}x_0.
    \widehat{g_1}(x)\widehat{g_2}(x)\ldots\widehat{g_n}(x)x_{n+1}x_{n+2}\ldots$$
    where each $\widehat{g_i}(x)$ is defined in the manner of our
    definition of $\widehat{h}$ above.
  \end{definition}
  
  \begin{definition} \textbf{Input and Output Maps}
    
    We define the input map, \functype{I_\infty}{R^\infty}{R_\infty}
    by:
    $$I_\infty(x) = 
    (\ldots, \widehat{n}.x_1,x_2,\ldots,x_n, 0,0,0,\ldots) \text{ for }
    x \in R^n$$ where $\widehat{n}$ denotes a sequence of $n$ $1$s.
    
    Defining the output map, \functype{O_\infty}{R_\infty}{R^\infty},
    is more subtle than defining the input map, since we need some way
    of determining how many coordinates of the state space to output.
    We do so by scanning the negative-indexed coordinates of the state
    space until we find an entry equal to $0$, and we output the same
    number of positive-indexed entries.
    $$O_\infty(\ldots x_{-2}x_{-1}x_0.x_1x_2\ldots x_l \ldots ) = 
    \twopartdef{\mathbf{0}}{l = 0}{(x_1, \ldots, x_l)}{\text{otherwise}}
    $$
    where $l = \min_{i \geq 0} \suchthat x_{-i} = 0$.  
  \end{definition}

  \begin{definition}{\textbf{BSS Machine over R}}

    A machine $M$ over a ring $R$ is a finite directed graph with
    node set $\nodes$.  The collection $\nodes$ contains three
    distinguished nodes: the \textbf{input node} $\eta_1$, the
    \textbf{accept node} $\accept$, and the \textbf{reject node}
    $\reject$.  The input node has indegree 0 and outdegree 1; the
    accept and reject nodes have outdegree 0.  The remaining
    elements of $\nodes$ are divided into three types:
    \emph{computation, branch,} and \emph{shift}.

    The inputs to $M$ are drawn from $R^\infty$.  The state space of
    $M$ is $R_\infty$.  Outputs from $M$ are mapped from the state
    space back to $R^\infty$.\footnote{In this respect, the
      description of a BSS Machine is simpler than that of a
      finite-dimensional machine, since every BSS Machine has the same
      input, output, and state spaces.  Occasionally, when we wish to
      emphasize the use of $R^\infty$ as the input or output space of
      $M$, we use the notation $\inspace_M$ and $\outspace_M$
      developed for FDMs.  The same goes for referring to $R_\infty$
      as $\statespace_M$.}

    As with FDMs, each node of a BSS Machine has an associated map and
    a set of next nodes.\footnote{As with finite dimensional machines,
      references to polynomial maps can be replaced with references to
      rational maps whenever $R$ is a field, and branching conditions
      that depend on order can be replaced with tests for equality
      with $0$ whenever $R$ is to be considered without order.}

    \begin{enumerate}
    \item Associated with the input node $\eta_1$ is the input map
      \functype{I_\infty}{R^\infty}{R_\infty} We denote the input node
      by $\eta_1$, and we denote the unique next node of $\eta_1$ by
      $\beta_{\eta_1}$ or just $\beta_1$.

    \item Associated with each computation node $\eta$ is a polynomial
      map \functype{\widehat{g}_\eta}{R_\infty}{R_\infty}. We again
      denote the unique next node of $\eta$ by $\beta_\eta$.

    \item Associated with each branch node is a polynomial map
      \functype{\widehat{h}_\eta}{R_\infty}{R}.  We associate the
      condition $\widehat{h}_\eta(x) \geq 0$ with the Yes Node of
      $\eta$, which we denote by $\beta_\eta^+$.  We associate the
      condition $\widehat{h}_\eta(x) < 0$ with the No Node
      $\beta_\eta^-$.

    \item Associated with each shift node $\sigma$ is either the
      \textbf{shift left} map $\sigma_l$ or the \textbf{shift right}
      map $\sigma_r$.  The shift maps are given by:
      \begin{align*}
        \sigma_l(x)_i &= x_{i+1}\\
        \sigma_r(x)_i &= x_{i-1}
      \end{align*}
      We denote the unique next node of $\sigma$ by
      $\beta_\sigma$. 
    \end{enumerate}
  \end{definition}

  \begin{definition}{\textbf{Degree and Dimension of a BSS Machine}}
    
    Let $M$ be a BSS Machine with node set $\nodes_M$.  We define the
    \textbf{dimension} $K_M$ of $M$ to be the maximum dimension of any
    polynomial map appearing in a computation or branch node of $M$.\\
    
    Similarly, we define the \textbf{degree} of $M$ to be the maximum
    degree of any polynomial map appearing in a computation or branch
    node of $M$.
    
  \end{definition}

  ``Full'' BSS Machines have much in common with their
  finite-dimensional counterparts.  The principle differences between
  the two models come from the expansion of input, state and output
  spaces and the inclusion of shift nodes.  It is worth noting that
  without the latter inclusion our machines would still be, in
  essence, finite dimensional, since the computation and branch nodes
  only modify a finite region of $\statespace_M$ at any given
  computation step.  The shift nodes allow us to make use of the
  expanded state space by providing a mechanism for moving values at
  high indices into locations where they can be used for computation.

  \begin{example}
    In \cite{B98}, Blum et al. show how to construct a Universal
    Polynomial Evaluator (UPE), a BSS Machine which takes as input an
    encoding of a polynomial $p(x) \in \poly{R}$ and an argument
    $x_0$, outputting the value $p(x_0)$.  Using this machine as a
    subroutine, we can construct an algorithm to run Newton's Method
    for arbitrary polynomial inputs.  Here and in what follows, we use
    the notation $\langle X \rangle$ to denote an \textbf{encoding} of
    some object $X$.

    \begin{algorithm}
      \caption{Newton's Method} \label{alg:newton}
      \begin{algorithmic}
        \Require $\langle (p(x), x_0, \epsilon) \rangle \mid p(x) \in \poly{R}\,; x_0 \in R\,; \epsilon \in R$
        \State Compute and store $\langle p'(x) \rangle$
        \While{$p(x_0) > \epsilon$ }
        \State Compute, using UPE subroutine, the values $p(x_0)$ and $p'(x_0)$
        \State $x_0 = x - \frac{p(x)}{p'(x)}$
        \EndWhile
      \end{algorithmic}
    \end{algorithm}
  \end{example}

  \note{It quickly becomes cumbersome to present a full formal
    construction of BSS Machines that solve even relatively simple
    problems.  This is true of Turing Machines in the classical model
    as well.  In keeping with the classical tradition, we prefer to
    present algorithms in high-level pseudocode in the manner shown
    above.  For a justification of this approach, see \cite{B98}
    Sections 3.5 and 4.4.}

  \section{Characterization of Halting Sets}

  \begin{theorem}{\textbf{Path Decomposition Theorem}}
    
    For any machine M and positive integers n and T, we have the
    following:

    \begin{enumerate}
    \item $\halting_T^n = \bigcup_{\computepath \in \haltpaths{T}}
      \coincidence_\computepath^n.$ That is, the n-dimensional
      component of the time-T halting set M of is a finite disjoint union
      of basic semi-algebraic sets.
    \item $\halting_M^n = \bigcup_{\computepath \in \haltpaths{}}
      \coincidence{\computepath}^n$.  That is, the n-dimensional
      component of the halting set of M is a countable disjoint union
      of basic semi-algebraic sets.
    \end{enumerate}
  \end{theorem}

  \begin{proofsketch}
    
    The proof of the Path Decomposition Theorem proceeds by
    considering the computation of a BSS Machine $M$ on a
    finite-dimensional subset of $R^\infty$.  The essential idea is to
    show that the computation performed by $M$ in time $T$ inputs of
    size at most $n$ is equivalent from the computation performed by
    some finite-dimensional machine.  This follows from the fact that
    we can bound the number of registers used by $M$ under such
    restrictions, because it can perform at most $T$ shift operations,
    and because each computation map is the extension of some map
    \functype{g}{R^m}{R^n}, and therefore modifies only finitely many
    registers.\\
    
    One cause for care is that the shift node maps are not polynomial
    when considered as maps from $R_\infty \rightarrow R_\infty$.
    This difficulty is resolved by noting that the shift maps can be
    reconstructed as polynomial maps for any finite-dimensional
    subset, $R^n \subset R_\infty$, and by noting that in time $T$ the
    machine can only modify a finite number of registers.  This
    follows from the fact that all the computation maps are extensions
    of finite polynomial maps from $R^m \rightarrow R^n$, and from the
    fact that and we can traverse at most T shift nodes in time T.
    Since we can bound the total number of registers accessible by a
    given machine in time $T$, the computation performed by $M$ in
    time $T$ on inputs of size at most $n$ is equivalent to the
    computation of some finite dimensional machine.  The theorem
    statements then follow from Theorem \ref{thm:pdt-finite}.
    
  \end{proofsketch}

  % \section{Register Equations}

  % One of the most important tools for investigating the behavior of
  % BSS machines is the system of polynomial equations known as the
  % \textbf{Register Equations}, which provide necessary and sufficient
  % conditions for a machine $M$ to accept an input $w$ in time $t$.
  % These follow directly from the definition of computation for a BSS
  % machine. 

  % \begin{definition}{\textbf{Register Equations}}
  % \end{definition}
  
  \section{Decision Problems}

  The treatment of complexity questions in our model is much the same
  as in classical complexity theory.  We frame problems as languages,
  ie, as sequences of symbols drawn from some input domain.  In
  classical theory, this means binary encodings of strings -- for us,
  it means subsets of $R^\infty$.  The complexity/decidability of a
  problem then becomes a question about the complexity/decidability of
  testing an input for membership in a language.

  \begin{definition}{\textbf{Decision Problem}}
      
    A \textbf{Decision Problem} over a ring $R$ is a set, $S \subseteq
    R^\infty$.  A \textbf{Structured Decision Problem} is a set $S$,
    partitioned into subsets, $S_{yes}$ and $S_{no}$, respectively
    called the \textbf{Yes Instances} and \textbf{No Instances} of
    $S$.
 
  \end{definition}

  \subsection{Notable Decision Problems}

  \todo{Figure out what to do with this.}

  \begin{itemize}
    \subsubsection{Decision Problems over Arbitrary Rings}
    \bolditem{HN} - (Hilbert Nullstellensatz) Does some finite system
    of polynomial equations have a shared solution?  
    \bolditem{QUAD} - Does a finite system of quadratic equations have a solution?
    \bolditem{4-FEAS} - Does a finite system of degree-4 polynomials have a solution?  
    \bolditem{QA-FEAS} - Does a set of quasi-algebraic equations describe the empty set?  
    \bolditem{KP} - Knapsack Problem
    
    \subsubsection{Decision Problems over Ordered Rings}
    \bolditem{SA-FEAS} - Does a set of semi-algebraic equations describe the empty set?
    \bolditem{LPF} - Feasibility of Linear Programming
    \bolditem{TSP} - Travelling Salesman Problem
    
    \subsubsection{Decision Problems over $\integers$}
    \bolditem{IPF} - Feasibility of Integer Programming Problems

    \subsubsection{Decision Problems over $\integers_2$}
    \bolditem{SAT} - Boolean formula satisfiability

    \todo{Add bounded versions? (See \cite{B98} 84-91.)}

  \end{itemize}

  \subsection{Known Decidability Results}

  All of the above are known to be decidable (when well-defined) over
  $\complexes$, $\reals$, and $\integers_2$.  Over $\integers$,
  \textbf{IPF}, \textbf{TSP}, and \textbf{KP} are decidable, but
  \textbf{4-FEAS}, \textbf{QUAD}, \textbf{HN}, \textbf{QA-FEAS}, and
  \textbf{SA-FEAS} are not.  Over $\rationals$, \textbf{LPF},
  \textbf{TSP}, and \textbf{KP} are decidable.  It is an open problem
  whether \textbf{QUAD} and \textbf{HN} are decidable over
  $\rationals$.

\section{Measuring Complexity}
\label{sec:measuring}
One of the major goals of classical theory of computation is to
provide a framework for analyzing the \emph{difficulty} of solving
problems.  Very generally, providing such a complexity analysis
involves characterizing the resources necessary for an algorithm to
compute a function $f$ or decide a set $S$.  Consideration of
different resources gives rise to different notions of complexity, but
a historically important case has been the notion of \textbf{time
  complexity}.  In the Turing Machine model of computation, analyzing
time complexity means analyzing the number of steps taken by a TM on
input $x$ as a function of length of $x$.  It is easily shown, for
example, that on input $x$, the machine exhibited in
\ref{fig:turing-example} performs exactly $\len(x) + 1$ steps before
halting, where $\len(x)$ is the number of $0$s and $1$s in $x$.\\

Measuring the size of an input string $x$ is simple in the Turing
Machine model; is it always simply the number symbols in $x$.  For
Blum et al., however, input size is slightly more complicated.
Consider for example, the FDM $M$ presented in Figure
\ref{fig:z-decide}.  Every input to $M$ is a single real number.  If
we consider the size of inputs $M$ to be the number of registers used
to store an input (ie, the dimension of the input space,
$\inspace_M$), then every input to $M$ has the same size.  On the
other hand, it is easy to show that the length of any computation
performed by $M$ on input $x \in \reals$ is bounded by $\floor{x}$.
This leads us naturally to consider two distinct notions of size
within the BSS model: the size associated with the dimensionality of
an input, which we call the \textbf{length} of the input, and the size
associated with the value of entries in $\inspace_M$ or
$\statespace_M$.  Drawing on terminology from number theory, we refer
to this latter notion as \textbf{height}.  We then define the overall
\textbf{size} of an input in terms of both its length and the heights
of its entries.

\begin{definition}{\textbf{Height Function} over $R$}

  We define a \textbf{Height Function} on a ring $R$ to be a map:
  $$\mfunctype{ht_R}{R^\infty}{\reals}$$
\end{definition}

\note{Common height functions include the \emph{unit height} function,
  $ht_R(x) = 1$, and the \emph{bit height} function, $ht_R(x) =
  \log(|x| + 1)$.  We use the unit height function in cases where we
  are concerned with the algebraic complexity of an algorithm (ie, the
  number of algebraic operations required to compute the algorithm,
  independent of the values on which the operations are performed).
  We most often use the bit height function when working with
  $\integers$ when we want to capture the classical cost of computing
  with the binary representation of an integer. Similarly, when
  working with $\rationals$, we give an alternative definition of bit
  height, $ht_\rationals(x) = (max(ht_\integers(p), ht_\integers(q))$,
  where $x = \frac{p}{q}$, $p$ and $q$ relatively prime.  This
  captures the cost of representing a rational $x$ as a pair of binary
  integers.}

\begin{definition}{\textbf{Input Length}}
  
  The \textbf{length} of an input $x \in R^n$ to a machine over $R$ is
  given by: $$\len(x) = n$$
\end{definition}

\begin{definition}{\textbf{Size}}
  
  Let $R$ be a ring with height function $ht_R$.  The \textbf{size} of
  a polynomial $x \in R^n$ is given by:
  $$\size(x) = \len(x)\max_{1 \leq i \leq n}(ht_R(x_i))$$
  \note{In the case that $ht_R$ is the unit height function, the size of
  the input to a machine reduces to just the length of the input.}
\end{definition}

Now that we have a definition of input size, we want to consider the
\textbf{cost} of performing a computation for a given machine on a
given input.  We define this quantity to be the number of steps the
machine takes before halting times the maximum height of any value
stored in the machine's state space during computation.

\begin{definition}{\textbf{Computation Cost}}
  
  Let $M$ be a BSS Machine over $R$ with height function $ht_R$, and
  let $x \in \halting_M$, such that 
  $$\compute_M(x) = (z_0, z_1,z_2, \ldots, z_T)$$ 
  
  The \textbf{cost} of $\compute_M(x)$ is given by:
  $$\cost_M(\compute_M(x)) = T*ht_{max}(x)$$
  where $ht_{max}(x)$ is the maximum height of any element occurring
  in the computation of $M$ on input $x$.  More specifically, if the
  state trajectory $\stateprojection(x)$ is given by: \todo{Figure out
    how to center this properly.}
  \begin{eqnarray*}
    \ldots x_{(-2,0)}x_{(-1,0)}x_{(0,0)}.x_{(1,0)}x_{(2,0)} \ldots ,\\ 
    \ldots x_{(-2,1)}x_{(-1,1)}x_{(0,1)}.x_{(1,1)}x_{(2,1)} \ldots ,\\ 
    \ldots x_{(-2,2)}x_{(-1,2)}x_{(0,2)}.x_{(1,2)}x_{(2,2)} \ldots ,\\
    \vdots       
  \end{eqnarray*}

  then $$ht_{max}(x) = \max_{i \in \integers, \; \j \leq T(x)}ht(x_{(i,j)})$$
  In the case that $M$ does not halt on $x$ (i.e., $x \notin
  \halting_M$, we adopt the convention that $\cost_M(x) = \infty$.
\end{definition}

  \note{The above definition departs slightly from the one given in
    Chapter 4 of \cite{B98}.  Blum et al. use the notation
    $\cost_M(x)$, rather than $\cost(\compute_M(x))$.  We do this to
    emphasize that cost is determined by a height function combined
    with a particular path through the configuration space of $M$.
    This will be important for us in Chapter \ref{chap:ndet} when we
    consider machines that traverse multiple paths on a single input.}

\begin{definition}\textbf{$\dtime(O(f(x)))$}

  Let $f(x)$ be a function $\reals \rightarrow \reals$.  A decision
  problem $S$ is said to be a member of the complexity class
  $\dtime(O(f(x)))$ if there exists a machine $M$ and a constant $c
  \in \reals$ such that $M$ decides $S$ and for all $x \in \inspace_M$
  we have:
  $$\cost_M(\compute_M(x)) \leq cf(x)$$
\end{definition}

\begin{example}
  \todo{Good example for this}
\end{example}

\begin{definition}{\textbf{Polynomial Time Machine}}

  A machine $M$ over $R$ is a \textbf{polynomial time} machine on $X
  \subset \inspace_M$ if there exist positive integers $c$ and $q$
  such that:
  $$cost_M(x) \leq c(size(x))^q \mid \forall x \in X$$
\end{definition}

\begin{definition}{\textbf{p-morphism}}
  
  A function, \functype{\varphi}{X}{Y \cup R^\infty} is called
  \textbf{polynomial-time computable} over $R$, or a
  \textbf{p-morphism} over $R$ if there exists a polynomial time
  machine over $R$ that computes $\varphi$.

\end{definition}

\todo{Rework this using decision machines instead of characteristic
  functions, since the former generalizes more naturally to NDET
  Machines.}

\begin{definition}{Membership in $P$}

  A decision problem $S \subseteq R^\infty$ is said to be in class $P$
  if its characteristic function, $\charfunc_S$ is a p-morphism.
\end{definition}

\section{Polynomial Time Reductions}

\begin{definition}{\textbf{p-reducibility}}

  A decision problem $S$ is \textbf{p-reducible} to a decision problem
  $S'$ if there exists a p-morphism,
  \functype{\varphi}{R^\infty}{R^\infty} such that $\varphi(S_{yes})
  \subseteq S'_{yes}$ and $\varphi(S_{no}) \subseteq S'_{no}$.  The
  map $\varphi$ is called a \textbf{polynomial-time reduction} from
  $S$ to $S'$.

  We denote the sentence \emph{$S$ is p-reducible to $S'$} by writing:

  $$S \preduce S'$$
  \end{definition}

  \note{ From the above definition it follows immediately that if
    $\varphi$ is a polynomial-time reduction from $S$ to $S'$, then $x
    \in S_{yes}$ if and only if $\varphi(x) \in S'_{yes}$.  Moreover,
    since compositions of polynomials are still polynomials, it should
    be clear that if a decision problem $S$ is p-reducible to a
    decision problem $S' \in P$, then $S \in P$ as well.}

\section{Notable Prior Results in Reducibility}

\begin{theorem}{\textbf{Known Reductions}}
  \begin{enumerate}
  \item Over any ring or field $R$, $HN \preduce QUAD$.
  \item Over any ordered ring or field $R$, $HN \preduce 4-FEAS$.
  \item Over $(\integers_2, >)$, 
    \begin{enumerate}
    \item $HN \preduce QUAD \preduce 4FEAS$
    \item $QUAD \preduce IP \preduce IPF$
    \end{enumerate}
  \end{enumerate}
\end{theorem}

\section{The Class NP}

The definition of NP over R given by Blum et al is a generalization of
the verifiability definition of NP given in classical complexity theory.

\begin{definition}{\textbf{Class NP} over $R$}

  A decision problem $S \subseteq R^\infty$ is in the class NP over a
  ring $R$ if there exist positive integers $c$ and $q$ and a a
  machine $M$ over R with $\inspace_M = R^\infty \times R^\infty$ such
  that:

  \begin{enumerate}
  \item if $x \in S$ then there exists a \emph{witness}, $w
    \in R^\infty$, such that $\computefn_M(x,w) = 1$ and
    $cost_M(x,w) \leq c(size(x))^q$.
  \item if $x \notin S$, then there is no such $w$.
  \end{enumerate}
\end{definition}

A classic simple example of a problem in NP is SUBSETSUM, which asks,
given a set of values $X = {x_1, x_2, x_3, \ldots x_k}$ and a target
$T$, is there a subset $S \subseteq X$ such that $\sum\limits_{x_i \in
  S} x_i = T$?  Though it is not clear how we might go about
efficiently finding such a subset, it is clear that if we are given a
candidate subset as a witness, we can quickly verify that it is, in
fact, a solution by simply adding the values and checking if the sum
is equal to $T$.\\

An important feature of the class $NP$ in the classical theory is the
existence of \textbf{NP Complete} problems, which are in a sense the
``hardest'' problems in $NP$.  More precisely, NP Complete problems
are those problems which lie in $NP$, and to which every problem in
$NP$ is reducible in polynomial time.  These problems can be shown to
exist in the generalized algebraic theory as well.

\begin{definition}{\textbf{NP Hard over R}}
  A decision problem $S$ is said to be \textbf{NP Hard} over R if
  every $S' \in NP_R$ is p-reducible to $S$.
\end{definition}

\begin{definition}{\textbf{NP Complete over R}}
  A decision problem $S$ is said to be \textbf{NP Complete} over R it
  is both NP Hard and in NP.
\end{definition}

\section{Known NP-Complete Problems}

The following results hold with respect to either unit or bit cost.
See \cite{B98} for details.  \todo{Should we present proofs or proof
  sketches here?}

\begin{itemize}
\item QA-FEAS is NP-Complete over any unordered ring or field R.
\item HN and QUAD are NP-Complete over (F, =) for any field F.
\item SA-FEAS is NP-Complete over any ordered ring or field R.
\item HN, QUAD, and 4-FEAS are NP-Complete over $\integers$,
  $\rationals$ and $\reals$, all considered with order.
\end{itemize}

\todo{Add bounded versions?}

%%% Local Variables: 
%%% mode: latex
%%% TeX-master: "main"
%%% End: 
